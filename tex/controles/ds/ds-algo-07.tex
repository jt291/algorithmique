% ds-algo-07.tex

%-------------------------------------------------------------------------
\documentclass[11pt,a4paper]{article}
%-------------------------------------------------------------------------

%-------------------------------------------------------------------------
\input{ds-algo-preambule.tex}
%-------------------------------------------------------------------------

%-----------------------------------------------------------------------------
\begin{document}
%-----------------------------------------------------------------------------
\entete

%-----------------------------------------------------------------------------
\section{Calcul de $\pi$}
%-----------------------------------------------------------------------------
Définir une fonction qui calcule $\pi$ à l'ordre $n$ selon la formule :
	$$\pi = 2\cdot
	\frac{4}{3}\cdot\frac{16}{15}\cdot\frac{36}{35}\cdot\frac{64}{63}\cdots =
	      2\prod_{k=1}^n\frac{4k^2}{4k^2 - 1}$$

$$\framebox[14.5cm]{$\rule{0cm}{16cm}$}$$

%-----------------------------------------------------------------------------
\section{Conversion décimal $\rightarrow$ base $b$}
%-----------------------------------------------------------------------------
Définir une fonction qui calcule le code $t$ en base $b$ sur $k$ chiffres
d'un entier $n$.
Exemples pour $n=23$ : 
\begin{tabular}[t]{ll@{\ $\rightarrow$\ }l}
{\tt b = 2}  & {\tt k = 7} & {\tt t = [0, 0, 1, 0, 1, 1, 1]}\\
{\tt b = 5}  & {\tt k = 5} & {\tt t = [0, 0, 0, 4, 3]}\\
{\tt b = 21} & {\tt k = 2} & {\tt t = [1, 2]}\\
{\tt b = 25} & {\tt k = 6} & {\tt t = [0, 0, 0, 0, 0, 23]}
\end{tabular}

$$\framebox[14.5cm]{$\rule{0cm}{18cm}$}$$

%-----------------------------------------------------------------------------
\section{Quinconce}
%-----------------------------------------------------------------------------
Définir une procédure qui dessine $n\times m$ cercles de rayon $r$
disposés en quinconce sur $n$ rangées de $m$ cercles chacune.
On utilisera les instructions de tracé {\em à la Logo}.

$$\framebox[14.5cm]{$\rule{0cm}{17cm}$}$$


%-----------------------------------------------------------------------------
\section{Coefficients de Kreweras}
%-----------------------------------------------------------------------------
\begin{minipage}[t]{7cm}
On consid\`ere la fonction {\tt g} ci-contre :
\begin{enumerate}
\item Calculer toutes les valeurs possibles
	de $g(n,m)$ pour $n \in [0,6]$.
\item Vérifier que {\tt 12.*g(5,5)/g(6,6)} est une bonne approximation de
	$\pi$.
\end{enumerate}
\end{minipage}
\hfill
\begin{minipage}[t]{7cm}\footnotesize
\begin{verbatim}
#----------------------------------
def g(n,m):
#----------------------------------
    assert type(n) is int
    assert type(m) is int
    assert 0 <= m and m <= n
    if n == 0 and m == 0:
        c = 1
    else:
        if m == 0: c = 0
        else:
            c = 0
            for i in range(1,m+1):
                c = c + g(n-1,n-i)
    return c
#---------------------------------
\end{verbatim}
\end{minipage}


$$\framebox[14.5cm]{$\rule{0cm}{14cm}$}$$

%-----------------------------------------------------------------------------
\section{Portée des variables}
%-----------------------------------------------------------------------------
On considère les fonctions {\tt f}, {\tt g} et {\tt h} suivantes :
\begin{center}
\begin{py}{4cm}
\begin{verbatim}
def f(x):
    x = 2*x
    print('f', x)
    return x
\end{verbatim}
\end{py}\hspace*{1cm}
\begin{py}{4cm}
\begin{verbatim}
def g(x):
    x = 4*f(x)
    print('g', x)
    return x
\end{verbatim}
\end{py}\hspace*{1cm}
\begin{py}{4cm}
\begin{verbatim}
def h(x):
    x = 3*g(f(x))
    print('h', x)
    return x
\end{verbatim}
\end{py}
\end{center}

Qu'affichent les appels suivants ?
\vspace*{2mm}

\begin{minipage}{7cm}
\begin{enumerate}
\item 

\begin{py}{3cm}
\begin{verbatim}
>>> x = 2
>>> print(x)

>>> y = f(x)
>>> print(x)

>>> z = g(x)
>>> print(x)

>>> t = h(x)
>>> print(x)
\end{verbatim}
\end{py}

\framebox[5.5cm]{$\rule{0cm}{11cm}$}
\end{enumerate}
\end{minipage}
\hfill
\begin{minipage}{7cm}
\begin{enumerate}

\item

\begin{py}{4cm}
\begin{verbatim}
>>> x = 2
>>> print(x)

>>> x = f(x)
>>> print(x)

>>> x = g(x)
>>> print(x)

>>> x = h(x)
>>> print(x)
\end{verbatim}
\end{py}

\framebox[5.5cm]{$\rule{0cm}{11cm}$}

\end{enumerate}
\end{minipage}

%-----------------------------------------------------------------------------
\section{Exécution d'une fonction itérative}
%-----------------------------------------------------------------------------
\begin{minipage}[t]{7cm}
On considère la procédure {\tt f} définie ci-contre.
\begin{enumerate}
\item Qu'affiche l'instruction sui\-vante ?
\begin{verbatim}
>>> for n in range(7): f(n)

\end{verbatim}
\item Que représente {\tt c} à la fin de chaque itération sur {\tt p} ?
\end{enumerate}
\end{minipage}
\hfill
\begin{py}{8cm}
\begin{verbatim}
#----------------------------------------
def f(n):
#----------------------------------------
    for p in range(n+1) :
        num = 1
        den = 1
        for i in range(1,p+1) :
            num = num*(n-i+1)
            den = den*i
        c = num/den
        print(c,end=' ')  #--- affichage
    print()               #--- affichage
    return
#----------------------------------------
\end{verbatim}
\end{py}
\vspace*{3mm}

\noindent\framebox[7cm]{\begin{minipage}{6.5cm}
{\tt c} ? \vspace*{7cm}
\end{minipage}}
\mbox{}\hfill
\framebox[8.25cm]{\begin{minipage}{7.5cm}\tt
>>> for n in range(7): f(n)\\
...
\vspace*{9cm}

\end{minipage}}

%-----------------------------------------------------------------------------
\label{fini}
\end{document}
%-----------------------------------------------------------------------------
