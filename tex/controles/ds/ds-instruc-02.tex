% ds-instruc-02.tex

%-------------------------------------------------------------------------
\documentclass[11pt,a4paper]{article}
%-------------------------------------------------------------------------

%-------------------------------------------------------------------------
%-------------------------------------------------------------------------
% ds-info-S1-preambule.tex
%-------------------------------------------------------------------------

%-------------------------------------------------------------------------
\usepackage{calc}
\usepackage[text={16cm,23cm},centering=true,showframe=false]{geometry}
\usepackage{fancybox,fancyvrb,fancyhdr,lastpage,lineno,import}
\usepackage{longtable,multirow}
\usepackage{xcolor,graphics,xmpmulti,pgf,pgfpages,tikz,wrapfig}
\usepackage{colortbl,color}
\usepackage{amsmath,amssymb,amsfonts}
\usepackage{hyperref,multimedia,rotating,framed,pstricks}
\usepackage{listings,index}
%
%---- pdflatex
%\usepackage[T1]{fontenc}
%\usepackage[utf8]{inputenc}
%---- xelatex
\usepackage{fontspec}
%
%\usepackage[french]{minitoc}
\usepackage[french]{babel}
\usepackage[french]{nomencl}
\usepackage[framed,hyperref,standard]{ntheorem}
\usepackage{eurosym,pifont}
%-------------------------------------------------------------------------

%-------------------------------------------------------------------------
\lstset
{
language=Python,
basicstyle=\ttfamily,
identifierstyle=\ttfamily,
keywordstyle=\color{blue}\ttfamily,
commentstyle=\color{gray}\ttfamily,
stringstyle=\color{green}\ttfamily,
showstringspaces=false,
extendedchars=true,
numbers=left, 
numberstyle=\color{blue}\tiny,
frame=lines,
linewidth=0.95\textwidth,
xleftmargin=5mm
} 
%-------------------------------------------------------------------------

%-------------------------------------------------------------------------
\pgfdeclareimage[width=3cm,interpolate=true]{logo-enib}{logo-enib}
%-------------------------------------------------------------------------

%-------------------------------------------------------------------------
\pagestyle{fancy}
\fancyhead{}
\fancyhead[L]{\hspace*{-3em}\begin{minipage}{3cm}\pgfuseimage{logo-enib}\end{minipage}}
\fancyhead[C]{Informatique S1}
\fancyhead[R]{\thepage/\pageref{LastPage}}
\fancyfoot{}
\fancyfoot[L]{}
\fancyfoot[C]{}
\fancyfoot[R]{}
\setlength{\headheight}{80pt}
\setlength{\footskip}{38pt}
\renewcommand{\headrulewidth}{0pt}
\renewcommand{\footrulewidth}{0pt}
%-------------------------------------------------------------------------

\voffset=-1cm

%-------------------------------------------------------------------------
\def\entete{\noindent\begin{tabular}{|l|l|l|} 
\hline 
 & & \\ 
\makebox[6cm][l]{\bsc{Nom :}} & \makebox[6cm][l]{\bsc{Prénom :}} & \makebox[2.65cm][l]{\bsc{Groupe :}} \\[1mm] 
\hline 
\end{tabular}\\[1mm]
{\footnotesize \textsc{Durée : 90'\hfill Documents, calculettes, téléphones et ordinateurs interdits}}}

\def\notes{\begin{tabular}{|c|c|c|c|}  
\hline 
\makebox[0.5cm]{3} & \makebox[0.5cm]{2} & \makebox[0.5cm]{1} & \makebox[0.5cm]{0} \\  
\hline
\end{tabular}  
} 

\def\autoevaluation{$$\begin{tabular}{|c|c|c|}
\hline
\multicolumn{3}{|c|}{\textbf{Auto-évaluation}} \\
\hline
\textbf{M} & \textbf{V} & \textbf{R} \\
Méthode(s) & Vérification(s) & Résultat(s) \\
\notes & \notes & \notes \\[1mm]
\hline
\end{tabular}$$ $$ $$}

\def\reponse{\mbox{}\hfill \fbox{\huge Réponse page suivante}}
%-------------------------------------------------------------------------

%-------------------------------------------------------------------------
\tikzset{
xmin/.store in=\xmin, xmin/.default=-3, xmin=-3,
xmax/.store in=\xmax, xmax/.default=3,  xmax=3,
ymin/.store in=\ymin, ymin/.default=-3, ymin=-3,
ymax/.store in=\ymax, ymax/.default=3,  ymax=3,
}

\newcommand{\grille}{\draw[color=lightgray] (\xmin,\ymin) grid (\xmax,\ymax);}

\newcommand{\axes}{
	\draw[->] (\xmin,0) -- (\xmax,0);
	\draw[->] (0,\ymin) -- (0,\ymax);
}

\newcommand{\fenetre}{\clip (\xmin,\ymin) rectangle (\xmax,\ymax);}
%-------------------------------------------------------------------------

%-------------------------------------------------------------------------
\def\ga{\textsc{ga}}   
\def\bu{\textsc{bu}} 
\def\zo{\textsc{zo}} 
\def\meu{\textsc{meu}} 
%-------------------------------------------------------------------------

%-------------------------------------------------------------------------
\newenvironment{py}[1]{\begin{minipage}[t]{#1}\footnotesize}{\end{minipage}}
%-------------------------------------------------------------------------

%-------------------------------------------------------------------------
\input{sigle}
%-------------------------------------------------------------------------

\graphicspath{{../../fig/}}



\usepackage{epsfig}
%-------------------------------------------------------------------------

%-----------------------------------------------------------------------------
\begin{document}
%-----------------------------------------------------------------------------
\entete

%-----------------------------------------------------------------------------
\section{Exécution d'une séquence d'instructions}
%-----------------------------------------------------------------------------
Qu'affiche la séquence d'instructions suivante ?
\vspace*{5mm}

\hspace*{5mm}\begin{py}{5cm}
\begin{verbatim}
a = 441
x = 1
z = a
y = 0
t = x
print(a,x,z,t,y)
while x <= a: x = x*4

print(a,x,z,t,y)
t = x
while x > 1:
  x = x/4
  t = t/2 - x
  if t <= z: 
    z = z - t
    t = t + x*2
  y = t/2
  print(a,x,z,t,y)

print(a,x,z,t,y)
\end{verbatim}
\end{py}\hfill
\begin{tabular}[t]{|c|c|c|c|c|}
\hline
\makebox[1.25cm]{a} & \makebox[1.25cm]{x} & \makebox[1.25cm]{z} &
\makebox[1.25cm]{t} & \makebox[1.25cm]{y} \\
\hline
  &   &   &   &   \\[15cm]
\hline
\end{tabular}

%-----------------------------------------------------------------------------
\newpage
\section{Calcul de $\pi$}
%-----------------------------------------------------------------------------
Ecrire un algorithme qui calcule $pi$ à l'ordre $n$ selon la formule :
	$$\frac{\pi^2}{6} = 1 + \frac{1}{4} + \frac{1}{9} + \frac{1}{16} - \cdots + \frac{1}{n^2} = 
	\sum_{k=1}^n \frac{1}{k^2}$$

$$\framebox[14.5cm]{\begin{minipage}{14cm}
\vspace*{17cm}

\end{minipage}}$$ 

%-----------------------------------------------------------------------------

%-----------------------------------------------------------------------------
\newpage
\section{Zéro d'une fonction}
%-----------------------------------------------------------------------------
Dans cette section, on recherche le zéro d'une fonction $f$ continue sur un 
intervalle $[a,b]$ telle que $f(a).f(b) < 0$ 
(il existe donc une racine de $f$ dans $]a,b[$ que nous supposerons 
unique). 

Ecrire un algorithme qui détermine le zéro de $\cos(x)$ dans $[4,5]$
	selon la méthode des tangentes.
	
	Indications : soit $x_n$ une approximation de la racine $c$ recherchée : 
	$f(c) = f(x_n) + (c-x_n)f'(x_n)$; comme $f(c) = 0$, on a : 
	$c = x_n - f(x_n)/f'(x_n)$. Posons $x_{n+1} = x_n - f(x_n)/f'(x_n)$ : 
	on peut considérer que $x_{n+1}$ est une meilleure approximation de $c$ que 
	$x_n$. On recommence le procédé avec $x_{n+1}$  et ainsi de suite jusqu'à ce 
	que $|x_{n+1}-x_n|$ soit inférieur à un certain seuil $s$.

$$\framebox[14.5cm]{\begin{minipage}{14cm}
\vspace*{15cm}

\end{minipage}}$$ 

%-----------------------------------------------------------------------------
\newpage
\section{Tableau d'Ibn al-Banna}
%-----------------------------------------------------------------------------
L'exercice suivant est inspir\'e du premier chapitre du
livre ``Histoire d'algo\-ri\-thmes''\footnote{{\bf Chabert J.-L. et al.}, 
{\em Histoire d'algorithmes : du caillou à la puce, Chapitre 1: algorithmes des op\'erations 
arithm\'etiques}, Editions Belin, Paris, 1994}.
On consid\`ere ici le texte d'Ibn al-Banna concernant la multiplication
\`a l'aide de tableaux.
$$\begin{minipage}{15cm}\footnotesize
Tu construis un quadrilat\`ere que tu subdivises verticalement et
horizontalement en autant de bandes qu'il y a de positions dans les
deux nombres multipli\'es. Tu divises diagonalement les carr\'es
obtenus, \`a l'aide de diagonale allant du coin inf\'erieur gauche au
coin sup\'erieur droit.

Tu places le multiplicande au-dessus du quadrilat\`ere, en faisant 
correspondre chacune de ses positions \`a une colonne\footnote{L'écriture
du nombre s'effectue de droite à gauche (exemple : 352 s'écrira donc 253).}. 
Puis, tu places le multiplicateur \`a gauche ou \`a droite du quadrilat\`ere,
de telle sorte qu'il descende avec lui en faisant correspondre \'egalement 
chacune de ses positions \`a une ligne\footnote{L'écriture
du nombre s'effectue de bas en haut (exemple : {\tiny$\begin{array}{c}3\\5\\2\end{array}$} 
s'écrira donc {\tiny$\begin{array}{c}2\\5\\3\end{array}$}).}. Puis, tu multiplies, 
l'une apr\`es l'autre, chacune des positions du multiplicande du carr\'e 
par toutes les positions du multiplicateur, et tu poses le r\'esultat 
partiel correspondant \`a chaque position dans le carr\'e o\`u se coupent 
respectivement leur colonne et leur ligne, en pla\c{c}ant les unit\'es 
au-dessus de la diagonale et les dizaines en dessous. Puis, tu
commences \`a additionner, en partant du coin sup\'erieur gauche :
tu additionnes ce qui est entre les diagonales, sans effacer, 
en pla\c{c}ant chaque nombre dans sa position, en transf\'erant 
les dizaines de chaque somme partielle \`a la diagonale suivante et
en les ajoutant \`a ce qui y figure. 

La somme que tu obtiendras sera le r\'esultat.
\end{minipage}$$

En utilisant la m\'ethode du tableau d'Ibn al-Banna, calculer $31977\times293$ 
($= 9369261$).
$$\framebox[14.5cm]{\begin{minipage}{14cm}
\vspace*{10cm}

\end{minipage}}$$ 


%-----------------------------------------------------------------------------
\label{fini}
\end{document}
%-----------------------------------------------------------------------------
