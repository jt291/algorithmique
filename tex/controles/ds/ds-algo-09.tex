% ds-algo-09.tex

%-------------------------------------------------------------------------
\documentclass[11pt,a4paper]{article}
%-------------------------------------------------------------------------

%-------------------------------------------------------------------------
%-------------------------------------------------------------------------
% ds-info-S1-preambule.tex
%-------------------------------------------------------------------------

%-------------------------------------------------------------------------
\usepackage{calc}
\usepackage[text={16cm,23cm},centering=true,showframe=false]{geometry}
\usepackage{fancybox,fancyvrb,fancyhdr,lastpage,lineno,import}
\usepackage{longtable,multirow}
\usepackage{xcolor,graphics,xmpmulti,pgf,pgfpages,tikz,wrapfig}
\usepackage{colortbl,color}
\usepackage{amsmath,amssymb,amsfonts}
\usepackage{hyperref,multimedia,rotating,framed,pstricks}
\usepackage{listings,index}
%
%---- pdflatex
%\usepackage[T1]{fontenc}
%\usepackage[utf8]{inputenc}
%---- xelatex
\usepackage{fontspec}
%
%\usepackage[french]{minitoc}
\usepackage[french]{babel}
\usepackage[french]{nomencl}
\usepackage[framed,hyperref,standard]{ntheorem}
\usepackage{eurosym,pifont}
%-------------------------------------------------------------------------

%-------------------------------------------------------------------------
\lstset
{
language=Python,
basicstyle=\ttfamily,
identifierstyle=\ttfamily,
keywordstyle=\color{blue}\ttfamily,
commentstyle=\color{gray}\ttfamily,
stringstyle=\color{green}\ttfamily,
showstringspaces=false,
extendedchars=true,
numbers=left, 
numberstyle=\color{blue}\tiny,
frame=lines,
linewidth=0.95\textwidth,
xleftmargin=5mm
} 
%-------------------------------------------------------------------------

%-------------------------------------------------------------------------
\pgfdeclareimage[width=3cm,interpolate=true]{logo-enib}{logo-enib}
%-------------------------------------------------------------------------

%-------------------------------------------------------------------------
\pagestyle{fancy}
\fancyhead{}
\fancyhead[L]{\hspace*{-3em}\begin{minipage}{3cm}\pgfuseimage{logo-enib}\end{minipage}}
\fancyhead[C]{Informatique S1}
\fancyhead[R]{\thepage/\pageref{LastPage}}
\fancyfoot{}
\fancyfoot[L]{}
\fancyfoot[C]{}
\fancyfoot[R]{}
\setlength{\headheight}{80pt}
\setlength{\footskip}{38pt}
\renewcommand{\headrulewidth}{0pt}
\renewcommand{\footrulewidth}{0pt}
%-------------------------------------------------------------------------

\voffset=-1cm

%-------------------------------------------------------------------------
\def\entete{\noindent\begin{tabular}{|l|l|l|} 
\hline 
 & & \\ 
\makebox[6cm][l]{\bsc{Nom :}} & \makebox[6cm][l]{\bsc{Prénom :}} & \makebox[2.65cm][l]{\bsc{Groupe :}} \\[1mm] 
\hline 
\end{tabular}\\[1mm]
{\footnotesize \textsc{Durée : 90'\hfill Documents, calculettes, téléphones et ordinateurs interdits}}}

\def\notes{\begin{tabular}{|c|c|c|c|}  
\hline 
\makebox[0.5cm]{3} & \makebox[0.5cm]{2} & \makebox[0.5cm]{1} & \makebox[0.5cm]{0} \\  
\hline
\end{tabular}  
} 

\def\autoevaluation{$$\begin{tabular}{|c|c|c|}
\hline
\multicolumn{3}{|c|}{\textbf{Auto-évaluation}} \\
\hline
\textbf{M} & \textbf{V} & \textbf{R} \\
Méthode(s) & Vérification(s) & Résultat(s) \\
\notes & \notes & \notes \\[1mm]
\hline
\end{tabular}$$ $$ $$}

\def\reponse{\mbox{}\hfill \fbox{\huge Réponse page suivante}}
%-------------------------------------------------------------------------

%-------------------------------------------------------------------------
\tikzset{
xmin/.store in=\xmin, xmin/.default=-3, xmin=-3,
xmax/.store in=\xmax, xmax/.default=3,  xmax=3,
ymin/.store in=\ymin, ymin/.default=-3, ymin=-3,
ymax/.store in=\ymax, ymax/.default=3,  ymax=3,
}

\newcommand{\grille}{\draw[color=lightgray] (\xmin,\ymin) grid (\xmax,\ymax);}

\newcommand{\axes}{
	\draw[->] (\xmin,0) -- (\xmax,0);
	\draw[->] (0,\ymin) -- (0,\ymax);
}

\newcommand{\fenetre}{\clip (\xmin,\ymin) rectangle (\xmax,\ymax);}
%-------------------------------------------------------------------------

%-------------------------------------------------------------------------
\def\ga{\textsc{ga}}   
\def\bu{\textsc{bu}} 
\def\zo{\textsc{zo}} 
\def\meu{\textsc{meu}} 
%-------------------------------------------------------------------------

%-------------------------------------------------------------------------
\newenvironment{py}[1]{\begin{minipage}[t]{#1}\footnotesize}{\end{minipage}}
%-------------------------------------------------------------------------

%-------------------------------------------------------------------------
\input{sigle}
%-------------------------------------------------------------------------

\graphicspath{{../../fig/}}



%-------------------------------------------------------------------------

\usepackage{epsfig}

%-----------------------------------------------------------------------------
\begin{document}
%-----------------------------------------------------------------------------
\entete

%-----------------------------------------------------------------------------
\section{Calcul de $\pi$}
Définir une fonction qui calcule $\pi$ à l'ordre $n$ selon la formule :
$$\pi = 2\cdot
	\frac{4}{3}\cdot\frac{16}{15}\cdot\frac{36}{35}\cdot\frac{64}{63}\cdots =
	      2\prod_{k=1}^n\frac{4k^2}{4k^2 - 1}$$

$$\framebox[14.5cm]{$\rule{0cm}{15cm}$}$$

%-----------------------------------------------------------------------------
\section{Conversion base $b$ $\rightarrow$ décimal}
Définir une fonction qui calcule la valeur décimale $n$ d'un entier
positif $t$ codé en base $b$.\\
Exemples : 
\begin{tabular}[t]{ll@{\ $\rightarrow$\ }l}
{\tt b = 2}  & {\tt t = [0, 0, 1, 0, 1, 1, 1]}    & {\tt n = 23}\\
{\tt b = 5}  & {\tt t = [0, 0, 0, 4, 3]}          & {\tt n = 23}\\
{\tt b = 21} & {\tt t = [1, 2]}                   & {\tt n = 23}\\
{\tt b = 25} & {\tt t = [0, 0, 0, 0, 0, 23]}      & {\tt n = 23}
\end{tabular}

$$\framebox[14.5cm]{$\rule{0cm}{19cm}$}$$


%-----------------------------------------------------------------------------
\section{Polygones réguliers}
Définir une fonction {\tt polygone(x0,y0,d,n)} qui trace un polygone régulier à {\tt n} côtés 
de longueur {\tt d} à partir du point de coordonnées $(x_0,y_0)$. 
On utilisera les instructions de tracé {\em à la Logo} 
(voir annexe \ref{logo} page \pageref{logo}).

\begin{minipage}{6cm}
Exemples :

\begin{verbatim}
>>> from turtle import *
>>> polygone(-100,0,50,3)
>>> polygone(0,0,20,7)
>>> polygone(100,0,2,100)
\end{verbatim}
\end{minipage}\hfill
\begin{minipage}{8cm}
\epsfig{figure=polygone.eps, width=8cm}
\end{minipage}

$$\framebox[14.5cm]{$\rule{0cm}{14cm}$}$$


%-----------------------------------------------------------------------------
\section{Spirales rectangulaires}
Définir une fonction {\tt spirale(n,x0,y0,a0,dr)} qui trace une spirale rectangulaire 
à {\tt n} côtés à partir du point de coordonnées $(x_0,y_0)$ et avec une orientation initiale
$a_0$. {\tt dr} représente l'incrément de longueur d'un côté de la spirale à son suivant
immédiat (le premier côté ayant pour longueur {\tt dr}).
On utilisera les instructions de tracé {\em à la Logo} 
(voir annexe \ref{logo} page \pageref{logo}).\vspace*{2mm}


\begin{minipage}{6cm}
Exemples :

\begin{verbatim}
>>> from turtle import *
>>> spirale(10,-100,0,0,8)
>>> spirale(20,0,0,30,3)
>>> spirale(15,100,0,-45,5)
\end{verbatim}
\end{minipage}\hfill
\begin{minipage}{8cm}
\epsfig{figure=spirale.eps, width=8cm}
\end{minipage}

$$\framebox[14.5cm]{$\rule{0cm}{13cm}$}$$

\newpage


%-----------------------------------------------------------------------------
\section{Portée des variables}
On considère les fonctions {\tt f}, {\tt g} et {\tt h} suivantes :
\begin{center}
\begin{py}{4cm}
\begin{verbatim}
def f(x):
    x = 3*x
    print('f', x)
    return x
\end{verbatim}
\end{py}\hspace*{1cm}
\begin{py}{4cm}
\begin{verbatim}
def g(x):
    x = 2*f(x)
    print('g', x)
    return x
\end{verbatim}
\end{py}\hspace*{1cm}
\begin{py}{4cm}
\begin{verbatim}
def h(x):
    x = g(f(x))
    print('h', x)
    return x
\end{verbatim}
\end{py}
\end{center}

Qu'affichent les appels suivants ?
\vspace*{2mm}

\begin{minipage}{7cm}
\begin{enumerate}
\item 

\begin{py}{3cm}
\begin{verbatim}
>>> x = 2
>>> print(x)

>>> y = f(x)
>>> print(x)

>>> z = g(x)
>>> print(x)

>>> t = h(x)
>>> print(x)
\end{verbatim}
\end{py}

\framebox[5.5cm]{$\rule{0cm}{13cm}$}
\end{enumerate}
\end{minipage}
\hfill
\begin{minipage}{7cm}
\begin{enumerate}

\item

\begin{py}{4cm}
\begin{verbatim}
>>> x = 2
>>> print(x)

>>> x = f(x)
>>> print(x)

>>> x = g(x)
>>> print(x)

>>> x = h(x)
>>> print(x)
\end{verbatim}
\end{py}

\framebox[5.5cm]{$\rule{0cm}{13cm}$}

\end{enumerate}
\end{minipage}

%-----------------------------------------------------------------------------
\section{Recherche d'un élément dans un tableau}
Définir une fonction itérative {\tt recherche(t,x,k)} pour la recherche de la 
$\mbox{\tt k}^{\grave eme}$ 
occurrence d'un élément {\tt x} dans un tableau {\tt t} à partir de la fin du
tableau. La fonction retourne
un doublet {\tt (ok,i)} où {\tt ok} est une variable booléenne égale à 
{\tt True} si la $\mbox{\tt k}^{\grave eme}$ occurrence de {\tt x} existe dans 
{\tt t} ({\tt i} est alors le rang de cette occurrence dans {\tt t}); {\tt ok}
vaut {\tt False} sinon (la valeur de {\tt i} n'a alors pas de signification :
elle est sans importance).

Exemples : \begin{py}{7cm}
\begin{verbatim}
>>> recherche([1,2,1,3,4,1,5],1,2)
(True, 2)
>>> recherche([1,2,1,3,4,1,5],1,4)
(False, -1)
\end{verbatim}
\end{py}


$$\framebox[14.5cm]{$\rule{0cm}{17cm}$}$$

%-----------------------------------------------------------------------------
\section{Exécution d'une fonction de tri}
\begin{minipage}[t]{7cm}
Qu'affiche l'appel {\tt f([5,1,4,7,10])} où {\tt f} est la fonction itérative
définie ci-contre ?

$$\framebox[6.75cm]{$\rule{0cm}{15cm}$}$$
\end{minipage}
\hfill
\begin{py}{8cm}
\begin{verbatim}
#----------------------------------------
def f(t):
#----------------------------------------
    assert type(t) is list

    for i in range(len(t)):
        m = i
        for j in range(i+1,len(t)):
            if t[j] > t[m]: m = j
        t[i],t[m] = t[m],t[i]
        print(i, t) #---------- affichage

    return t
#----------------------------------------
\end{verbatim}
\end{py}

%-----------------------------------------------------------------------------
\section{Exécution d'une fonction de sélection}
\begin{minipage}[t]{7cm}
Qu'affiche l'appel {\tt g(h,[5,1,4,7,10])} où {\tt g} et {\tt h}
sont les fonctions définies ci-contre ?

$$\framebox[6.75cm]{$\rule{0cm}{15cm}$}$$
\end{minipage}
\hfill
\begin{py}{8cm}
\begin{verbatim}
#----------------------------------------
def h(x):
#----------------------------------------
    assert type(x) is int
    
    return x%2 == 0
#----------------------------------------


#----------------------------------------
def g(p,t):
#----------------------------------------
    assert type(t) is list

    i = 0
    while i < len(t):
        if p(t[i]) == True:
            del t[i]
        else: i = i + 1
        print(i, t)  #--------- affichage
    return t
#----------------------------------------
\end{verbatim}
\end{py}

%-----------------------------------------------------------------------------
\appendix
\newpage
%-----------------------------------------------------------------------------
\section{Annexe : instructions Logo}\label{logo}
%-----------------------------------------------------------------------------
\begin{quote}\scriptsize\em 
{\em Logo is the name for a philosophy of education and 
a continually evolving family of programming languages that aid in its realization
(Harold Abelson, Apple Logo, 1982).}
This statement sums up two fundamental aspects of Logo and puts them in the proper order. 
The Logo programming environments that have been developed over the past 28 years are 
rooted in constructivist educational philosophy, and are designed to support constructive 
learning. [\ldots]
Constructivism views knowledge as being created by learners in their own minds through 
interaction with other people and the world around them. This theory is most closely 
associated with Jean Piaget, the Swiss psychologist, who spent decades studying and 
documenting the learning processes of young children.\\
\mbox{}\hfill {\rm\bf Logo Foundation :} {\tt http://el.media.mit.edu/logo-foundation}
\end{quote}

On suppose connues les procédures de tracés géométriques {\em à la Logo} :\vspace*{2mm}

\begin{minipage}{14cm}\footnotesize
\begin{description}
\item[\tt degrees()] fixe l'unité d'angle en degrés
\item[\tt radians()] fixe l'unité d'angle en radians
\item[\tt reset()] efface l'écran et réinitialise les variables
\item[\tt clear()] efface l'écran
\item[\tt forward(d)] avance d'une distance $d$
\item[\tt backward(d)] recule d'une distance $d$
\item[\tt left(a)] tourne sur la gauche d'un angle $a$
\item[\tt right(a)] tourne sur la droite d'un angle $a$
\item[\tt up()] lève le crayon 
\item[\tt down()] abaisse le crayon 
\item[\tt goto(x,y)] déplace le crayon à la position $(x,y)$
\item[\tt setheading(a)] oriente la tortue selon un angle $a$ avec l'axe horizontal
\item[\tt towards(x,y)] donne l'angle entre la direction courante et la direction 
	pointant vers le point de coordonnées $(x,y)$
\item[\tt circle(r)] trace un cercle de rayon $r$
\item[\tt circle(r,a)] trace un arc de cercle de rayon $r$ et d'angle au sommet $a$
\end{description}
\end{minipage}

%-----------------------------------------------------------------------------
\label{fini}
\end{document}
%-----------------------------------------------------------------------------
