% ds-algo-01.tex

%-------------------------------------------------------------------------
\documentclass[11pt,a4paper]{article}
%-------------------------------------------------------------------------

%-------------------------------------------------------------------------
%-------------------------------------------------------------------------
% ds-info-S1-preambule.tex
%-------------------------------------------------------------------------

%-------------------------------------------------------------------------
\usepackage{calc}
\usepackage[text={16cm,23cm},centering=true,showframe=false]{geometry}
\usepackage{fancybox,fancyvrb,fancyhdr,lastpage,lineno,import}
\usepackage{longtable,multirow}
\usepackage{xcolor,graphics,xmpmulti,pgf,pgfpages,tikz,wrapfig}
\usepackage{colortbl,color}
\usepackage{amsmath,amssymb,amsfonts}
\usepackage{hyperref,multimedia,rotating,framed,pstricks}
\usepackage{listings,index}
%
%---- pdflatex
%\usepackage[T1]{fontenc}
%\usepackage[utf8]{inputenc}
%---- xelatex
\usepackage{fontspec}
%
%\usepackage[french]{minitoc}
\usepackage[french]{babel}
\usepackage[french]{nomencl}
\usepackage[framed,hyperref,standard]{ntheorem}
\usepackage{eurosym,pifont}
%-------------------------------------------------------------------------

%-------------------------------------------------------------------------
\lstset
{
language=Python,
basicstyle=\ttfamily,
identifierstyle=\ttfamily,
keywordstyle=\color{blue}\ttfamily,
commentstyle=\color{gray}\ttfamily,
stringstyle=\color{green}\ttfamily,
showstringspaces=false,
extendedchars=true,
numbers=left, 
numberstyle=\color{blue}\tiny,
frame=lines,
linewidth=0.95\textwidth,
xleftmargin=5mm
} 
%-------------------------------------------------------------------------

%-------------------------------------------------------------------------
\pgfdeclareimage[width=3cm,interpolate=true]{logo-enib}{logo-enib}
%-------------------------------------------------------------------------

%-------------------------------------------------------------------------
\pagestyle{fancy}
\fancyhead{}
\fancyhead[L]{\hspace*{-3em}\begin{minipage}{3cm}\pgfuseimage{logo-enib}\end{minipage}}
\fancyhead[C]{Informatique S1}
\fancyhead[R]{\thepage/\pageref{LastPage}}
\fancyfoot{}
\fancyfoot[L]{}
\fancyfoot[C]{}
\fancyfoot[R]{}
\setlength{\headheight}{80pt}
\setlength{\footskip}{38pt}
\renewcommand{\headrulewidth}{0pt}
\renewcommand{\footrulewidth}{0pt}
%-------------------------------------------------------------------------

\voffset=-1cm

%-------------------------------------------------------------------------
\def\entete{\noindent\begin{tabular}{|l|l|l|} 
\hline 
 & & \\ 
\makebox[6cm][l]{\bsc{Nom :}} & \makebox[6cm][l]{\bsc{Prénom :}} & \makebox[2.65cm][l]{\bsc{Groupe :}} \\[1mm] 
\hline 
\end{tabular}\\[1mm]
{\footnotesize \textsc{Durée : 90'\hfill Documents, calculettes, téléphones et ordinateurs interdits}}}

\def\notes{\begin{tabular}{|c|c|c|c|}  
\hline 
\makebox[0.5cm]{3} & \makebox[0.5cm]{2} & \makebox[0.5cm]{1} & \makebox[0.5cm]{0} \\  
\hline
\end{tabular}  
} 

\def\autoevaluation{$$\begin{tabular}{|c|c|c|}
\hline
\multicolumn{3}{|c|}{\textbf{Auto-évaluation}} \\
\hline
\textbf{M} & \textbf{V} & \textbf{R} \\
Méthode(s) & Vérification(s) & Résultat(s) \\
\notes & \notes & \notes \\[1mm]
\hline
\end{tabular}$$ $$ $$}

\def\reponse{\mbox{}\hfill \fbox{\huge Réponse page suivante}}
%-------------------------------------------------------------------------

%-------------------------------------------------------------------------
\tikzset{
xmin/.store in=\xmin, xmin/.default=-3, xmin=-3,
xmax/.store in=\xmax, xmax/.default=3,  xmax=3,
ymin/.store in=\ymin, ymin/.default=-3, ymin=-3,
ymax/.store in=\ymax, ymax/.default=3,  ymax=3,
}

\newcommand{\grille}{\draw[color=lightgray] (\xmin,\ymin) grid (\xmax,\ymax);}

\newcommand{\axes}{
	\draw[->] (\xmin,0) -- (\xmax,0);
	\draw[->] (0,\ymin) -- (0,\ymax);
}

\newcommand{\fenetre}{\clip (\xmin,\ymin) rectangle (\xmax,\ymax);}
%-------------------------------------------------------------------------

%-------------------------------------------------------------------------
\def\ga{\textsc{ga}}   
\def\bu{\textsc{bu}} 
\def\zo{\textsc{zo}} 
\def\meu{\textsc{meu}} 
%-------------------------------------------------------------------------

%-------------------------------------------------------------------------
\newenvironment{py}[1]{\begin{minipage}[t]{#1}\footnotesize}{\end{minipage}}
%-------------------------------------------------------------------------

%-------------------------------------------------------------------------
\input{sigle}
%-------------------------------------------------------------------------

\graphicspath{{../../fig/}}



%-------------------------------------------------------------------------

%-----------------------------------------------------------------------------
\begin{document}
%-----------------------------------------------------------------------------
%\entete

%-----------------------------------------------------------------------------
\section{Fonctions numériques}
%-----------------------------------------------------------------------------

%-----------------------------------------------------------------------------
\subsection{Calcul de $\pi$}
Définir une fonction qui calcule $\pi$ à l'ordre $n$ selon l'approximation
suivante :
$$\pi \approx \sum_{k = 0}^n\frac{1}{16^k}
\left(\frac{4}{8k+1} - \frac{2}{8k+4} - \frac{1}{8k+5} - \frac{1}{8k+6}\right)$$
On n'utilisera pas la fonction {\em puissance} ({\tt x**k}).

$$\framebox[14.5cm]{$\rule{0cm}{14cm}$}$$

%-----------------------------------------------------------------------------
\subsection{Conversion décimal $\rightarrow$ base $b$}
Définir une fonction qui code sur $k$ chiffres un entier positif $n$ 
du système décimal au système en base $b$.

$$\framebox[14.5cm]{$\rule{0cm}{20cm}$}$$

%-----------------------------------------------------------------------------
%\entete
\section{Fonctions graphiques}
%-----------------------------------------------------------------------------

%-----------------------------------------------------------------------------
\subsection{Courbes paramétrées}
Ecrire une fonction qui permettent le tracé de courbes paramétrées
($x = f(t)$, $y = g(t)$) en utilisant les instructions {\em à la Logo}.
Les courbes paramétrées seront données sous la forme de fonctions {\tt lambda} 
comme dans l'exemple ci-dessous du cercle de centre $(x_0,y_0)$ et de rayon $r$.
$$\begin{py}{14cm}
\begin{verbatim}
#-----------------------------------------------------------------------
def parametric_circle(x0,y0,r):
#-----------------------------------------------------------------------
    """
    cercle paramétrique : x = x0 + r*cos(t), y = y0 + r * sin(t)
    """
    return lambda(t): x0 + r * cos(t), 
           lambda(t): y0 + r * sin(t)
#-----------------------------------------------------------------------
\end{verbatim}
\end{py}$$

$$\framebox[14.5cm]{$\rule{0cm}{11cm}$}$$


%-----------------------------------------------------------------------------
\subsection{Courbes fractales}
On consid\`ere la proc\'edure {\tt p} ci-contre :\hfill
\begin{py}{7cm}
\begin{verbatim}
def p(n,d):
    assert type(n) is int
    assert n >= 0
    if n == 0: forward(d)
    else:
        p(n-1,d/3.)
        left(60)
        p(n-1,d/3.)
        right(120)
        p(n-1,d/3.)
        left(60)
        p(n-1,d/3.)
    return
\end{verbatim}
\end{py}

\begin{enumerate}
\item On consid\`ere l'appel {\tt p(1,300)} et le crayon initialement en (0,0)
	avec une direction de 0.
	Dessiner le r\'esultat de cet appel.
\item On consid\`ere l'appel {\tt p(3,300)} et le crayon initialement en (0,0)
	avec une direction de 0.
	Dessiner le r\'esultat de cet appel.
\end{enumerate}

$$\framebox[14.5cm]{$\rule{0cm}{12cm}$}$$

%-----------------------------------------------------------------------------
%\entete
\section{Appels de fonctions}
%-----------------------------------------------------------------------------

%-----------------------------------------------------------------------------
\subsection{Portée des variables}
On considère les fonctions {\tt f}, {\tt g} et {\tt h} suivantes :
\begin{center}
\begin{py}{4cm}
\begin{verbatim}
def f(x):
    x = 2*x
    print('f', x)
    return x
\end{verbatim}
\end{py}\hspace*{1cm}
\begin{py}{4cm}
\begin{verbatim}
def g(x):
    x = 2*f(x)
    print('g', x)
    return x
\end{verbatim}
\end{py}\hspace*{1cm}
\begin{py}{4cm}
\begin{verbatim}
def h(x):
    x = 2*g(f(x))
    print('h', x)
    return x
\end{verbatim}
\end{py}
\end{center}

Qu'affichent les appels suivants ?
\vspace*{2mm}

\begin{minipage}{7cm}
\begin{enumerate}
\item 

\begin{py}{3cm}
\begin{verbatim}
>>> x = 5
>>> print(x)

>>> y = f(x)
>>> print(x)

>>> z = g(x)
>>> print(x)

>>> t = h(x)
>>> print(x)
\end{verbatim}
\end{py}

\framebox[5.5cm]{$\rule{0cm}{12cm}$}
\end{enumerate}
\end{minipage}
\hfill
\begin{minipage}{7cm}
\begin{enumerate}

\item

\begin{py}{4cm}
\begin{verbatim}
>>> x = 5
>>> print(x)

>>> x = f(x)
>>> print(x)

>>> x = g(x)
>>> print(x)

>>> x = h(x)
>>> print(x)
\end{verbatim}
\end{py}

\framebox[5.5cm]{$\rule{0cm}{12cm}$}

\end{enumerate}
\end{minipage}

%-----------------------------------------------------------------------------
\subsection{Récursivité}
Définir une fonction récursive pour la recherche de la première occurrence
d'un élément $x$ dans un tableau $t$.

$$\framebox[14.5cm]{$\rule{0cm}{20cm}$}$$

%-----------------------------------------------------------------------------
%\entete
\section{Exécutions de fonctions}
%-----------------------------------------------------------------------------

%-----------------------------------------------------------------------------
\subsection{Exécution d'une fonction itérative}
\begin{minipage}[t]{7cm}
Qu'affiche l'appel {\tt f([3,6,4,5,2,1])} où {\tt f} est la fonction itérative
définie ci-contre ?

$$\framebox[6.75cm]{$\rule{0cm}{15cm}$}$$
\end{minipage}
\hfill
\begin{py}{8cm}
\begin{verbatim}
#----------------------------------------
def f(t):
#----------------------------------------
    assert type(t) is list

    for i in range(len(t)):
        m = i
        for j in range(i+1,len(t)):
            if t[j] < t[m]: m = j
        t[i],t[m] = t[m],t[i]
        print(i, t)  #---------- affichage

    return t
#----------------------------------------
\end{verbatim}
\end{py}

%-----------------------------------------------------------------------------
\subsection{Exécution d'une fonction récursive}
\begin{minipage}[t]{7cm}
Qu'affiche l'appel {\tt f(3,4,5,6)} où {\tt f} est la fonction récursive
définie ci-contre ?

$$\framebox[6.75cm]{$\rule{0cm}{20cm}$}$$
\end{minipage}
\hfill
\begin{py}{8cm}
\begin{verbatim}
#----------------------------------------
def f(n,a,b,c):
#----------------------------------------
    assert type(n) is int 
    assert n >= 0
    assert a != b 
    assert b != c 
    assert a != c
    
    if n > 0:
      f(n-1,a,c,b)
      print(a, c) #------------- affichage
      f(n-1,b,a,c)

    return
#----------------------------------------
\end{verbatim}
\end{py}

%-----------------------------------------------------------------------------
\label{fini}
\end{document}
%-----------------------------------------------------------------------------
