% ds-algo-03.tex

%-------------------------------------------------------------------------
\documentclass[11pt,a4paper]{article}
%-------------------------------------------------------------------------

%-------------------------------------------------------------------------
%-------------------------------------------------------------------------
% ds-info-S1-preambule.tex
%-------------------------------------------------------------------------

%-------------------------------------------------------------------------
\usepackage{calc}
\usepackage[text={16cm,23cm},centering=true,showframe=false]{geometry}
\usepackage{fancybox,fancyvrb,fancyhdr,lastpage,lineno,import}
\usepackage{longtable,multirow}
\usepackage{xcolor,graphics,xmpmulti,pgf,pgfpages,tikz,wrapfig}
\usepackage{colortbl,color}
\usepackage{amsmath,amssymb,amsfonts}
\usepackage{hyperref,multimedia,rotating,framed,pstricks}
\usepackage{listings,index}
%
%---- pdflatex
%\usepackage[T1]{fontenc}
%\usepackage[utf8]{inputenc}
%---- xelatex
\usepackage{fontspec}
%
%\usepackage[french]{minitoc}
\usepackage[french]{babel}
\usepackage[french]{nomencl}
\usepackage[framed,hyperref,standard]{ntheorem}
\usepackage{eurosym,pifont}
%-------------------------------------------------------------------------

%-------------------------------------------------------------------------
\lstset
{
language=Python,
basicstyle=\ttfamily,
identifierstyle=\ttfamily,
keywordstyle=\color{blue}\ttfamily,
commentstyle=\color{gray}\ttfamily,
stringstyle=\color{green}\ttfamily,
showstringspaces=false,
extendedchars=true,
numbers=left, 
numberstyle=\color{blue}\tiny,
frame=lines,
linewidth=0.95\textwidth,
xleftmargin=5mm
} 
%-------------------------------------------------------------------------

%-------------------------------------------------------------------------
\pgfdeclareimage[width=3cm,interpolate=true]{logo-enib}{logo-enib}
%-------------------------------------------------------------------------

%-------------------------------------------------------------------------
\pagestyle{fancy}
\fancyhead{}
\fancyhead[L]{\hspace*{-3em}\begin{minipage}{3cm}\pgfuseimage{logo-enib}\end{minipage}}
\fancyhead[C]{Informatique S1}
\fancyhead[R]{\thepage/\pageref{LastPage}}
\fancyfoot{}
\fancyfoot[L]{}
\fancyfoot[C]{}
\fancyfoot[R]{}
\setlength{\headheight}{80pt}
\setlength{\footskip}{38pt}
\renewcommand{\headrulewidth}{0pt}
\renewcommand{\footrulewidth}{0pt}
%-------------------------------------------------------------------------

\voffset=-1cm

%-------------------------------------------------------------------------
\def\entete{\noindent\begin{tabular}{|l|l|l|} 
\hline 
 & & \\ 
\makebox[6cm][l]{\bsc{Nom :}} & \makebox[6cm][l]{\bsc{Prénom :}} & \makebox[2.65cm][l]{\bsc{Groupe :}} \\[1mm] 
\hline 
\end{tabular}\\[1mm]
{\footnotesize \textsc{Durée : 90'\hfill Documents, calculettes, téléphones et ordinateurs interdits}}}

\def\notes{\begin{tabular}{|c|c|c|c|}  
\hline 
\makebox[0.5cm]{3} & \makebox[0.5cm]{2} & \makebox[0.5cm]{1} & \makebox[0.5cm]{0} \\  
\hline
\end{tabular}  
} 

\def\autoevaluation{$$\begin{tabular}{|c|c|c|}
\hline
\multicolumn{3}{|c|}{\textbf{Auto-évaluation}} \\
\hline
\textbf{M} & \textbf{V} & \textbf{R} \\
Méthode(s) & Vérification(s) & Résultat(s) \\
\notes & \notes & \notes \\[1mm]
\hline
\end{tabular}$$ $$ $$}

\def\reponse{\mbox{}\hfill \fbox{\huge Réponse page suivante}}
%-------------------------------------------------------------------------

%-------------------------------------------------------------------------
\tikzset{
xmin/.store in=\xmin, xmin/.default=-3, xmin=-3,
xmax/.store in=\xmax, xmax/.default=3,  xmax=3,
ymin/.store in=\ymin, ymin/.default=-3, ymin=-3,
ymax/.store in=\ymax, ymax/.default=3,  ymax=3,
}

\newcommand{\grille}{\draw[color=lightgray] (\xmin,\ymin) grid (\xmax,\ymax);}

\newcommand{\axes}{
	\draw[->] (\xmin,0) -- (\xmax,0);
	\draw[->] (0,\ymin) -- (0,\ymax);
}

\newcommand{\fenetre}{\clip (\xmin,\ymin) rectangle (\xmax,\ymax);}
%-------------------------------------------------------------------------

%-------------------------------------------------------------------------
\def\ga{\textsc{ga}}   
\def\bu{\textsc{bu}} 
\def\zo{\textsc{zo}} 
\def\meu{\textsc{meu}} 
%-------------------------------------------------------------------------

%-------------------------------------------------------------------------
\newenvironment{py}[1]{\begin{minipage}[t]{#1}\footnotesize}{\end{minipage}}
%-------------------------------------------------------------------------

%-------------------------------------------------------------------------
\input{sigle}
%-------------------------------------------------------------------------

\graphicspath{{../../fig/}}



%-------------------------------------------------------------------------

%-----------------------------------------------------------------------------
\begin{document}
%-----------------------------------------------------------------------------
\entete

%-----------------------------------------------------------------------------
\section{Calcul de $\pi$}
%-----------------------------------------------------------------------------
Définir une fonction qui calcule $\pi$ à l'ordre $n$ selon la formule :
$$\frac{\pi}{4} = 1 - \frac{1}{3} + \frac{1}{5} - \cdots + (-1)^n\frac{1}{2n+1} = 
	\sum_{k=0}^n(-1)^k \frac{1}{2k+1}$$
On n'utilisera pas la fonction {\em puissance} ({\tt x**n}).
$$\framebox[14.5cm]{$\rule{0cm}{15cm}$}$$

%-----------------------------------------------------------------------------
\section{Conversion base $b$ $\rightarrow$ décimal}
%-----------------------------------------------------------------------------
Définir une fonction qui calcule la valeur décimale $n$ d'un entier
positif $t$ codé en base $b$.\\
Exemples : 
\begin{tabular}[t]{ll@{\ $\rightarrow$\ }l}
{\tt b = 2}  & {\tt t = [0, 0, 1, 0, 1, 1, 1]}    & {\tt n = 23}\\
{\tt b = 5}  & {\tt t = [0, 0, 0, 4, 3]}          & {\tt n = 23}\\
{\tt b = 21} & {\tt t = [1, 2]}                   & {\tt n = 23}\\
{\tt b = 25} & {\tt t = [0, 0, 0, 0, 0, 23]}      & {\tt n = 23}
\end{tabular}

$$\framebox[14.5cm]{$\rule{0cm}{18cm}$}$$

%-----------------------------------------------------------------------------
\section{Polygones réguliers}
%-----------------------------------------------------------------------------
Définir une fonction qui trace un polygone régulier à $n$ côtés de longueur
$d$ à partir du point de coordonnées $(x_0,y_0)$. On utilisera
les instructions de tracé {\em à la Logo}.

$$\framebox[14.5cm]{$\rule{0cm}{17cm}$}$$


%-----------------------------------------------------------------------------
\section{Courbes fractales}
%-----------------------------------------------------------------------------
\begin{minipage}[t]{7cm}
On consid\`ere la fonction {\tt g} ci-contre :
\begin{enumerate}
\item Calculer toutes les valeurs possibles
	de $g(n,m)$ pour $n \in [0,6]$.
\item Vérifier que {\tt 12.*g(5,5)/g(6,6)} est une bonne approximation de
	$\pi$.
\end{enumerate}
\end{minipage}
\hfill
\begin{minipage}[t]{7cm}
\begin{verbatim}
#----------------------------------------
def g(n,m):
#----------------------------------------
    assert type(n) is int
    assert type(m) is int
    assert 0 <= m and m <= n
    if n == 0 and m == 0:
        c = 1
    else:
        if m == 0: c = 0
        else:
            c = 0
            for i in range(1,m+1):
                c = c + g(n-1,n-i)
    return c
#----------------------------------------
\end{verbatim}
\end{minipage}


$$\framebox[14.5cm]{$\rule{0cm}{14cm}$}$$

%-----------------------------------------------------------------------------
\section{Portée des variables}
%-----------------------------------------------------------------------------
On considère les fonctions {\tt f}, {\tt g} et {\tt h} suivantes :
\begin{center}
\begin{py}{4cm}
\begin{verbatim}
def f(x):
    x = 3*x
    print('f', x)
    return x
\end{verbatim}
\end{py}\hspace*{1cm}
\begin{py}{4cm}
\begin{verbatim}
def g(x):
    x = 4*f(x)
    print('g', x)
    return x
\end{verbatim}
\end{py}\hspace*{1cm}
\begin{py}{4cm}
\begin{verbatim}
def h(x):
    x = 2*g(f(x))
    print('h', x)
    return x
\end{verbatim}
\end{py}
\end{center}

Qu'affichent les appels suivants ?
\vspace*{2mm}

\begin{minipage}{7cm}
\begin{enumerate}
\item 

\begin{py}{3cm}
\begin{verbatim}
>>> x = 2
>>> print(x)

>>> y = f(x)
>>> print(x)

>>> z = g(x)
>>> print(x)

>>> t = h(x)
>>> print(x)
\end{verbatim}
\end{py}

\framebox[5.5cm]{$\rule{0cm}{11cm}$}
\end{enumerate}
\end{minipage}
\hfill
\begin{minipage}{7cm}
\begin{enumerate}

\item

\begin{py}{4cm}
\begin{verbatim}
>>> x = 2
>>> print(x)

>>> x = f(x)
>>> print(x)

>>> x = g(x)
>>> print(x)

>>> x = h(x)
>>> print(x)
\end{verbatim}
\end{py}

\framebox[5.5cm]{$\rule{0cm}{11cm}$}

\end{enumerate}
\end{minipage}

%-----------------------------------------------------------------------------
\section{Exécution d'une fonction itérative}
%-----------------------------------------------------------------------------
\begin{minipage}[t]{7cm}
On considère la procédure {\tt f} définie ci-contre.
\begin{enumerate}
\item Qu'affiche l'instruction sui\-vante ?
\begin{verbatim}
>>> for n in range(7): f(n)

\end{verbatim}
\item Que représente {\tt c} à la fin de chaque itération sur {\tt p} ?
\end{enumerate}
\end{minipage}
\hfill
\begin{py}{8cm}
\begin{verbatim}
#----------------------------------------
def f(n):
#----------------------------------------
    for p in range(n+1) :
        num = 1
        den = 1
        for i in range(1,p+1) :
            num = num*(n-i+1)
            den = den*i
        c = num/den
        print(c,end=' ')  #--- affichage
    print()               #--- affichage
    return
#----------------------------------------
\end{verbatim}
\end{py}
\vspace*{3mm}

\noindent\framebox[7cm]{\begin{minipage}{6.5cm}
{\tt c} ? \vspace*{7cm}
\end{minipage}}
\mbox{}\hfill
\framebox[8.25cm]{\begin{minipage}{7.5cm}\tt
>>> for n in range(7): f(n)\\
...
\vspace*{9cm}

\end{minipage}}

%-----------------------------------------------------------------------------
\label{fini}
\end{document}
%-----------------------------------------------------------------------------
