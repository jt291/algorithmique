% ds-algo-10.tex

%-------------------------------------------------------------------------
\documentclass[11pt,a4paper]{article}
%-------------------------------------------------------------------------

%-------------------------------------------------------------------------
\input{ds-algo-preambule.tex}
%-------------------------------------------------------------------------

\usepackage{epsfig}

%-----------------------------------------------------------------------------
\begin{document}
%-----------------------------------------------------------------------------
\entete

%-----------------------------------------------------------------------------
\section{Calcul de $\pi$}
Définir une fonction qui calcule $\pi$ à l'ordre $n$ selon la formule :
$$\frac{\pi}{4} = 1 - \frac{1}{3} + \frac{1}{5} - \cdots + (-1)^n\frac{1}{2n+1} = 
	\sum_{k=0}^n(-1)^k \frac{1}{2k+1}$$
On n'utilisera pas la fonction {\em puissance} ({\tt x**n}).
$$\framebox[14.5cm]{$\rule{0cm}{16cm}$}$$


%-----------------------------------------------------------------------------
\section{Conversion base $b$ $\rightarrow$ décimal}
Définir une fonction qui calcule la valeur décimale $n$ d'un entier
positif $t$ codé en base $b$.\\
Exemples : 
\begin{tabular}[t]{ll@{\ $\rightarrow$\ }l}
{\tt b = 2}  & {\tt t = [0, 0, 1, 0, 1, 1, 1]}    & {\tt n = 23}\\
{\tt b = 5}  & {\tt t = [0, 0, 0, 4, 3]}          & {\tt n = 23}\\
{\tt b = 21} & {\tt t = [1, 2]}                   & {\tt n = 23}\\
{\tt b = 25} & {\tt t = [0, 0, 0, 0, 0, 23]}      & {\tt n = 23}
\end{tabular}

$$\framebox[14.5cm]{$\rule{0cm}{17cm}$}$$


%-----------------------------------------------------------------------------
\section{Spirales rectangulaires}
Définir une fonction {\tt spirale(n,x0,y0,a0,dr)} qui trace une spirale rectangulaire 
à {\tt n} côtés à partir du point de coordonnées $(x_0,y_0)$ et avec une orientation initiale
$a_0$. {\tt dr} représente l'incrément de longueur d'un côté de la spirale à son suivant
immédiat (le premier côté ayant pour longueur {\tt dr}).


\begin{minipage}{6cm}
Exemples :

\begin{verbatim}
>>> from turtle import *
>>> spirale(10,-100,0,0,8)
>>> spirale(20,0,0,30,3)
>>> spirale(15,100,0,-45,5)
\end{verbatim}
\end{minipage}\hfill
\begin{minipage}{8cm}
\epsfig{figure=spirale.eps, width=8cm}
\end{minipage}

$$\framebox[14.5cm]{$\rule{0cm}{10cm}$}$$

\newpage


%-----------------------------------------------------------------------------
\section{Portée des variables}
On considère les fonctions {\tt f}, {\tt g} et {\tt h} suivantes :
\begin{center}
\begin{py}{4cm}
\begin{verbatim}
def f(x):
    x = 3*x
    print('f', x)
    return x
\end{verbatim}
\end{py}\hspace*{1cm}
\begin{py}{4cm}
\begin{verbatim}
def g(x):
    x = 2*f(x)
    print('g', x)
    return x
\end{verbatim}
\end{py}\hspace*{1cm}
\begin{py}{4cm}
\begin{verbatim}
def h(x):
    x = g(f(x))
    print('h', x)
    return x
\end{verbatim}
\end{py}
\end{center}

Qu'affichent les appels suivants ?
\vspace*{2mm}

\begin{minipage}{7cm}
\begin{enumerate}
\item 

\begin{py}{3cm}
\begin{verbatim}
>>> x = 2
>>> print(x)

>>> y = f(x)
>>> print(x)

>>> z = g(x)
>>> print(x)

>>> t = h(x)
>>> print(x)
\end{verbatim}
\end{py}

\framebox[5.5cm]{$\rule{0cm}{9cm}$}
\end{enumerate}
\end{minipage}
\hfill
\begin{minipage}{7cm}
\begin{enumerate}

\item

\begin{py}{4cm}
\begin{verbatim}
>>> x = 2
>>> print(x)

>>> x = f(x)
>>> print(x)

>>> x = g(x)
>>> print(x)

>>> x = h(x)
>>> print(x)
\end{verbatim}
\end{py}

\framebox[5.5cm]{$\rule{0cm}{9cm}$}

\end{enumerate}
\end{minipage}


%-----------------------------------------------------------------------------
\section{Exécution d'une fonction itérative}
%-----------------------------------------------------------------------------

\begin{minipage}[t]{8cm}
On considère la fonction {\tt f} ci-contre.
\begin{enumerate}
\item Que fait cette fonction ?
	$$\framebox[7cm]{$\rule{0cm}{5cm}$}$$
\end{enumerate}
\end{minipage}
\hfill
\begin{py}{7cm}
\begin{verbatim}
def f(t):
    assert type(t) is list
    i = 0
    while i < len(t):
        j = i + 1
        while j < len(t):
            if t[j]%t[i] == 0 : del t[j]
            else: j = j + 1
            print(i, j, t)
        i = i + 1
    return t
\end{verbatim}
\end{py}

\begin{enumerate}\setcounter{enumi}{1}
\item Qu'affiche l'appel {\tt f(range(2,10))} ?
	$$\framebox[10cm]{$\rule{0cm}{10cm}$}$$
\end{enumerate}

%-----------------------------------------------------------------------------
\section{Exécution d'une fonction récursive}
%-----------------------------------------------------------------------------

\begin{minipage}[t]{8cm}
On considère la fonction {\tt f} ci-contre.
\begin{enumerate}
\item Que fait cette fonction ?
	$$\framebox[7cm]{$\rule{0cm}{5cm}$}$$
\end{enumerate}
\end{minipage}
\hfill
\begin{py}{7cm}
\begin{verbatim}
def f(a,b):
    assert type(a) is int and a >= 0
    assert type(b) is int and b >= 0
    if b == 0 : d = a
    else: d = f(b,a%b)
    print(a, b)
    return d
\end{verbatim}
\end{py}

\begin{enumerate}\setcounter{enumi}{1}
\item Qu'affiche l'appel {\tt f(12,18)} ?
	$$\framebox[10cm]{$\rule{0cm}{10cm}$}$$
\end{enumerate}


%-----------------------------------------------------------------------------
\label{fini}
\end{document}
%-----------------------------------------------------------------------------
