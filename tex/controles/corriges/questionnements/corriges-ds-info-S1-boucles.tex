%-------------------------------------------------------------------------
% corriges-ds-info-S1-boucles.tex
%-------------------------------------------------------------------------

%-------------------------------------------------------------------------
\documentclass[11pt,a4paper]{article}
%-------------------------------------------------------------------------

%-------------------------------------------------------------------------
\input{corriges-ds-info-S1-preambule.tex}
%-------------------------------------------------------------------------

%-------------------------------------------------------------------------
\begin{document}
%-------------------------------------------------------------------------
$$\mbox{\textbf{\large Développements limités}}$$
%-------------------------------------------------------------------------

\paragraph{Questions :} 
Ecrire un algorithme qui calcule $y = f(x)$ en fonction du développement en série entière de la fonction $f$ : 
$f(x) = \sum u_k$, en respectant les contraintes suivantes :
\begin{itemize}
\item les calculs seront arrêtés lorsque la valeur absolue du terme $u_k$ ($|u_k|$) 
	sera inférieure à un certain seuil $s$ (avec $0 < s < 1$);
\item on n'utilisera ni la fonction \emph{puissance} ($x^n$) ni la fonction
	\emph{facto\-riel\-le} ($n!$) pour effectuer le calcul du développement.
\end{itemize}
\vspace*{5mm}

\noindent\begin{minipage}{6cm}
D'une manière générale, le code aura l'allure ci-contre.
On peut vérifier en comparant la valeur obtenue pour $y$
au calcul direct de la fonction $f$ : $|y - f(x)|$, et ce 
pour différentes valeurs de $x$ et de $s$.
\end{minipage}
\hfill
\begin{minipage}{9cm}\footnotesize
\begin{Verbatim}[frame=single]
x, s = 0.25, 1.0e-9
# calcul de f(x)
k, u = k0, u0         # initialisation
y = u
while fabs(u) > s :
    u = g(u,k,x)      # relation de récurrence
    y = y + u
    k = k + 1
# vérification
print(fabs(y - f(x)) < s)
\end{Verbatim}
\end{minipage}
\vspace*{5mm}


%1
\begin{enumerate}
\item $\displaystyle\arcsin(x)$
\hfill
\begin{minipage}[t]{9cm}\footnotesize
\begin{Verbatim}
k, u = 0, x
y = u
u = u*x*x*(2*k+1)*(2*k+1)/((2*k+2)*(2*k+3))
\end{Verbatim}
\end{minipage}
\vspace*{3mm}

%2
\item $\displaystyle\arccos(x)$
\hfill
\begin{minipage}[t]{9cm}\footnotesize
\begin{Verbatim}
k, u = 0, -x
y = pi/2 + u
u = u*x*x*(2*k+1)*(2*k+1)/((2*k+2)*(2*k+3))
\end{Verbatim}
\end{minipage}
\vspace*{3mm}

%3
\item $\displaystyle\arctan(x)$
\hfill
\begin{minipage}[t]{9cm}\footnotesize
\begin{Verbatim}
k, u = 0, x
y = u
u = -u*x*x*(2*k+1)/(2*k+3)
\end{Verbatim}
\end{minipage}
\vspace*{3mm}

%4
\item $\displaystyle \frac{1}{1+x}$  
\hfill
\begin{minipage}[t]{9cm}\footnotesize
\begin{Verbatim}
k, u = 0, 1
y = u
u = -u*x
\end{Verbatim}
\end{minipage}
\vspace*{3mm}

%5
\item $\displaystyle \frac{1}{1-x}$
\hfill
\begin{minipage}[t]{9cm}\footnotesize
\begin{Verbatim}
k, u = 0, 1
y = u
u = u*x
\end{Verbatim}
\end{minipage}
\vspace*{3mm}

%6
\item $\displaystyle \frac{1}{1+x^2}$  
\hfill
\begin{minipage}[t]{9cm}\footnotesize
\begin{Verbatim}
k, u = 0, 1
y = u
u = -u*x*x
\end{Verbatim}
\end{minipage}
\vspace*{3mm}

%7
\item $\displaystyle \frac{1}{1-x^2}$
\hfill
\begin{minipage}[t]{9cm}\footnotesize
\begin{Verbatim}
k, u = 0, 1
y = u
u = u*x*x
\end{Verbatim}
\end{minipage}
\vspace*{3mm}

%8
\item $\displaystyle \sqrt{1+x}$
\hfill
\begin{minipage}[t]{9cm}\footnotesize
\begin{Verbatim}
k, u = 0, 1
y = u
u = -u*x*(2*k-1)/(2*(k+1))
\end{Verbatim}
\end{minipage}
\vspace*{3mm}

%9
\item $\displaystyle \frac{1}{\sqrt{1+x}}$
\hfill
\begin{minipage}[t]{9cm}\footnotesize
\begin{Verbatim}
k, u = 0, 1
y = u
u = -u*x*(2*k+1)/(2*k+2)
\end{Verbatim}
\end{minipage}
\vspace*{3mm}

%10
\item $\displaystyle \frac{1}{\sqrt{1-x^2}}$
\hfill
\begin{minipage}[t]{9cm}\footnotesize
\begin{Verbatim}
k, u = 0, 1
y = u
u = u*x*x*(2*k+1)*(2*k+2)/(4*(k+1)*(k+1))
\end{Verbatim}
\end{minipage}
\vspace*{3mm}

%11
\item $\displaystyle \frac{1}{(a-x)^2}$
\hfill
\begin{minipage}[t]{9cm}\footnotesize
\begin{Verbatim}
k, u = 0, 1/(a*a)
y = u
u = u*x*(k+2)/(a*(k+1))
\end{Verbatim}
\end{minipage}
\vspace*{3mm}

%12
\item $\displaystyle \frac{1}{(a-x)^3}$
\hfill
\begin{minipage}[t]{9cm}\footnotesize
\begin{Verbatim}
k, u = 0, 1/(a*a*a)
y = u
u = u*x*(k+3)/(a*(k+1))
\end{Verbatim}
\end{minipage}
\vspace*{3mm}

%13
\item $\displaystyle \frac{1}{(a-x)^5}$
\hfill
\begin{minipage}[t]{9cm}\footnotesize
\begin{Verbatim}
k, u = 0, 1/(a*a*a*a*a)
y = u
u = u*x*(k+5)/(a*(k+1))
\end{Verbatim}
\end{minipage}
\vspace*{3mm}


%14
\item $\displaystyle \exp(x)$
\hfill
\begin{minipage}[t]{9cm}\footnotesize
\begin{Verbatim}
k, u = 0, 1
y = u
u = u*x/(k+1)
\end{Verbatim}
\end{minipage}
\vspace*{3mm}


%15
\item $\displaystyle \exp(-x)$
\hfill
\begin{minipage}[t]{9cm}\footnotesize
\begin{Verbatim}
k, u = 0, 1
y = u
u = -u*x/(k+1)
\end{Verbatim}
\end{minipage}
\vspace*{3mm}


%16
\item $\displaystyle\log(1+x)$
\hfill
\begin{minipage}[t]{9cm}\footnotesize
\begin{Verbatim}
k, u = 1, x
y = u
u = -u*x*k/(k+1)
\end{Verbatim}
\end{minipage}
\vspace*{3mm}

%17
\item $\displaystyle\log(1-x)$
\hfill
\begin{minipage}[t]{9cm}\footnotesize
\begin{Verbatim}
k, u = 1, -x
y = u
u = u*x*k/(k+1)
\end{Verbatim}
\end{minipage}
\vspace*{3mm}


%18
\item $\displaystyle\log\left(\frac{1+x}{1-x}\right)$
\hfill
\begin{minipage}[t]{9cm}\footnotesize
\begin{Verbatim}
k, u = 0, 2*x
y = u
u = u*x*x*(2*k+1)/(2*k+3)
\end{Verbatim}
\end{minipage}
\vspace*{3mm}


%19
\item $\displaystyle\sinh(x)$
\hfill
\begin{minipage}[t]{9cm}\footnotesize
\begin{Verbatim}
k, u = 0, x
y = u
u = u*x*x/((2*k+2)*(2*k+3))
\end{Verbatim}
\end{minipage}
\vspace*{3mm}


%20
\item $\displaystyle\cosh(x)$
\hfill
\begin{minipage}[t]{9cm}\footnotesize
\begin{Verbatim}
k, u = 0, 1
y = u
u = u*x*x/((2*k+1)*(2*k+2))
\end{Verbatim}
\end{minipage}
\vspace*{3mm}


%21
\item $\displaystyle\arg\sinh(x)$  
\hfill
\begin{minipage}[t]{9cm}\footnotesize
\begin{Verbatim}
k, u = 0, x
y = u
u = -u*x*x*(2*k+1)*(2*k+1)/((2*k+2)*(2*k+3))
\end{Verbatim}
\end{minipage}
\vspace*{3mm}


%22
\item $\displaystyle\arg\tanh(x)$  	
\hfill
\begin{minipage}[t]{9cm}\footnotesize
\begin{Verbatim}
k, u = 0, x
y = u
u = u*x*x*(2*k+1)/(2*k+3)
\end{Verbatim}
\end{minipage}
\vspace*{3mm}

%23
\item $\displaystyle\sin(x)$
\hfill
\begin{minipage}[t]{9cm}\footnotesize
\begin{Verbatim}
k, u = 0, x
y = u
u = -u*x*x/((2*k+2)*(2*k+3))
\end{Verbatim}
\end{minipage}
\vspace*{3mm}

%24
\item $\displaystyle\cos(x)$
\hfill
\begin{minipage}[t]{9cm}\footnotesize
\begin{Verbatim}
k, u = 0, 1
y = u
u = -u*x*x/((2*k+1)*(2*k+2))
\end{Verbatim}
\end{minipage}
\vspace*{3mm}

\end{enumerate}

%-------------------------------------------------------------------------
\end{document}
%-------------------------------------------------------------------------
