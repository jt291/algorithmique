%-------------------------------------------------------------------------
% corriges-ds-info-S1-boucles-polygones.tex
%-------------------------------------------------------------------------

%-------------------------------------------------------------------------
\documentclass[11pt,a4paper,colorlinks,breaklinks]{article}
%-------------------------------------------------------------------------


%-------------------------------------------------------------------------
\input{corriges-ds-info-S1-preambule.tex}
%-------------------------------------------------------------------------

%-------------------------------------------------------------------------
\begin{document}
%-------------------------------------------------------------------------
$$\mbox{\textbf{\large Polygones réguliers}}$$


\paragraph{Questions :} 
En utilisant les instructions de la tortue \logo{}
(module \texttt{turtle}), écrire un algorithme qui dessine un motif géométrique
composé de $(n\times m)$ polygones réguliers alignés sur une grille
ou disposés en quinconce sur la grille.

\paragraph{Réponses :} 
D'une manière générale, le code aura la structure suivante selon que 
les polygones sont alignés ou en quinconce :

\noindent
\mbox{}\hfill
\begin{minipage}[t]{6cm}
alignés :\\
\centerline{\includegraphics[width=3cm]{grille-1.pdf}}

\footnotesize
\begin{Verbatim}[frame=single]
# initialisation du motif
dx, dy = 20, 20
n, m = 5, 4
x0, y0 = 0, 0
c, d = 7, 15 
# dessin du motif
for j in range(m) :
    xl = x0
    yl = y0 + j*dy
    # dessin d'une ligne de figures
    for i in range(n) :
        x, y = xl + i*dx, yl
        # dessin d'une figure
        up()
        goto(x,y)
        setheading(0)
        down()
        # tracé d'un polygone
        for i in range(c):
            forward(d)
            left(360/c)
\end{Verbatim}
\end{minipage}
\hfill
\begin{minipage}[t]{6cm}
en quinconce :\\
\centerline{\includegraphics[width=3cm]{grille-2.pdf}}

\footnotesize
\begin{Verbatim}[frame=single]
# initialisation du motif
dx, dy = 20, 20
n, m = 5, 4
x0, y0 = 0, 0
c, d = 7, 15
# dessin du motif
for j in range(m) :
    xl = x0 + dx*(j%2)/2
    yl = y0 + j*dy
    # dessin d'une ligne de figures
    nf = n - (j%2)
    for i in range(nf) :
        x, y = xl + i*dx, yl
        # dessin d'une figure
        up()
        goto(x,y)
        setheading(0)
        down()
        # tracé d'un polygone
        for i in range(c):
            forward(d)
            left(360/c)
\end{Verbatim}
\end{minipage}

%-------------------------------------------------------------------------
\end{document}
%-------------------------------------------------------------------------
