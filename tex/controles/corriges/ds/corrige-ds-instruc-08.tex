% corrige-ds-instruc-08.tex

%-------------------------------------------------------------------------
\documentclass[11pt,a4paper]{article}
%-------------------------------------------------------------------------

%-------------------------------------------------------------------------
\input{ds-instruc-preambule.tex}
\usepackage{epsfig}
%-------------------------------------------------------------------------

%-----------------------------------------------------------------------------
\begin{document}
%-----------------------------------------------------------------------------

%-----------------------------------------------------------------------------
\section{Exécution d'une séquence d'instructions}
%-----------------------------------------------------------------------------

%-----------------------------------------------------------------------------
\section{Calcul de $\pi$}
%-----------------------------------------------------------------------------
Ecrire un algorithme qui calcule $pi$ à l'ordre $n$ selon la formule :
	$$\frac{\pi^2}{6} = 1 + \frac{1}{4} + \frac{1}{9} + \frac{1}{16} - \cdots + \frac{1}{n^2} = 
	\sum_{k=1}^n \frac{1}{k^2}$$

$$\framebox[14.5cm]{\begin{minipage}{14cm}
\vspace*{17cm}

\end{minipage}}$$ 

%-----------------------------------------------------------------------------

%-----------------------------------------------------------------------------
\section{Zéro d'une fonction}
%-----------------------------------------------------------------------------
Dans cette section, on recherche le zéro d'une fonction $f$ continue sur un 
intervalle $[a,b]$ telle que $f(a).f(b) < 0$ 
(il existe donc une racine de $f$ dans $]a,b[$ que nous supposerons 
unique). 

Ecrire un algorithme qui détermine le zéro de $\cos(x)$ dans $[1,2]$
	selon la méthode des cordes.

	Indications : on pose $x_1 = a$, $x2 = b$ et $x = (x_2f(x_1) - x_1f(x_2))/(f(x_1)-f(x_2))$
	le point d'intersection de la corde $AB$ et de l'axe des abscisses. 
	Si $f(x1).f(x) < 0$, la racine est dans $]x_1,x[$ et on pose $x_2 = x$; 
	sinon la racine est dans $]x,x_2[$ et on pose $x_1 = x$. 
	Puis on réitère le procédé. Lorsque $x_1$ et $x_2$ seront suffisamment 
	proches, on décidera que la racine est $x$.
$$\framebox[14.5cm]{\begin{minipage}{14cm}
\vspace*{16cm}

\end{minipage}}$$ 


%-----------------------------------------------------------------------------
\section{Tableau d'Ibn al-Banna}
%-----------------------------------------------------------------------------
L'exercice suivant est inspir\'e du premier chapitre du
livre ``Histoire d'algo\-ri\-thmes''\footnote{{\bf Chabert J.-L. et al.}, 
{\em Histoire d'algorithmes : du caillou à la puce, Chapitre 1: algorithmes des op\'erations 
arithm\'etiques}, Editions Belin, Paris, 1994}.
On consid\`ere ici le texte d'Ibn al-Banna concernant la multiplication
\`a l'aide de tableaux.
$$\begin{minipage}{15cm}\footnotesize
Tu construis un quadrilat\`ere que tu subdivises verticalement et
horizontalement en autant de bandes qu'il y a de positions dans les
deux nombres multipli\'es. Tu divises diagonalement les carr\'es
obtenus, \`a l'aide de diagonale allant du coin inf\'erieur gauche au
coin sup\'erieur droit.

Tu places le multiplicande au-dessus du quadrilat\`ere, en faisant 
correspondre chacune de ses positions \`a une colonne\footnote{L'écriture
du nombre s'effectue de droite à gauche (exemple : 352 s'écrira donc 253).}. 
Puis, tu places le multiplicateur \`a gauche ou \`a droite du quadrilat\`ere,
de telle sorte qu'il descende avec lui en faisant correspondre \'egalement 
chacune de ses positions \`a une ligne\footnote{L'écriture
du nombre s'effectue de bas en haut (exemple : {\tiny$\begin{array}{c}3\\5\\2\end{array}$} 
s'écrira donc {\tiny$\begin{array}{c}2\\5\\3\end{array}$}).}. Puis, tu multiplies, 
l'une apr\`es l'autre, chacune des positions du multiplicande du carr\'e 
par toutes les positions du multiplicateur, et tu poses le r\'esultat 
partiel correspondant \`a chaque position dans le carr\'e o\`u se coupent 
respectivement leur colonne et leur ligne, en pla\c{c}ant les unit\'es 
au-dessus de la diagonale et les dizaines en dessous. Puis, tu
commences \`a additionner, en partant du coin sup\'erieur gauche :
tu additionnes ce qui est entre les diagonales, sans effacer, 
en pla\c{c}ant chaque nombre dans sa position, en transf\'erant 
les dizaines de chaque somme partielle \`a la diagonale suivante et
en les ajoutant \`a ce qui y figure. 

La somme que tu obtiendras sera le r\'esultat.
\end{minipage}$$

En utilisant la m\'ethode du tableau d'Ibn al-Banna, calculer $43973\times177$ 
($= 7783221$).
$$\framebox[14.5cm]{\begin{minipage}{14cm}
\vspace*{10cm}

\end{minipage}}$$ 


%-----------------------------------------------------------------------------
\label{fini}
\end{document}
%-----------------------------------------------------------------------------
