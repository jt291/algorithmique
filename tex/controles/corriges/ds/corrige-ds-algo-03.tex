% corrige-ds-algo-03.tex

%-------------------------------------------------------------------------
\documentclass[11pt,a4paper]{article}
%-------------------------------------------------------------------------

%-------------------------------------------------------------------------
\input{ds-algo-preambule.tex}
%-------------------------------------------------------------------------

%-----------------------------------------------------------------------------
\begin{document}
%-----------------------------------------------------------------------------

%-----------------------------------------------------------------------------
\section{Calcul de $\pi$}
%-----------------------------------------------------------------------------
\lstinputlisting[basicstyle=\footnotesize]{ds-algo-03-calculPi.py}

%-----------------------------------------------------------------------------
\section{Conversion base $b$ $\rightarrow$ décimal}
%-----------------------------------------------------------------------------
\lstinputlisting[basicstyle=\footnotesize]{ds-algo-03-conversion.py}

%-----------------------------------------------------------------------------
\section{Polygones réguliers}
%-----------------------------------------------------------------------------
\lstinputlisting[basicstyle=\footnotesize]{ds-algo-03-polygone.py}


%-----------------------------------------------------------------------------
\section{Coefficients de Kreweras}
%-----------------------------------------------------------------------------
%\lstinputlisting[basicstyle=\footnotesize]{ds-algo-03-kreweras.py}

\begin{minipage}[t]{7cm}
\begin{enumerate}
\item 
\begin{Verbatim}
>>> for n in range(7):
        for m in range(n+1):
            print(g(n,m),end=' ')
        print()

1 
0 1 
0 1 1 
0 1 2 2 
0 2 4 5 5 
0 5 10 14 16 16 
0 16 32 46 56 61 61 
\end{Verbatim}
\end{enumerate}
\end{minipage}
\hfill
\begin{minipage}[t]{7cm}
\begin{enumerate}\setcounter{enumi}{1}
\item  
\begin{Verbatim}
>>> 12*g(5,5)/g(6,6)
3.1475409836065573

\end{Verbatim}
\end{enumerate}
\end{minipage}


%-----------------------------------------------------------------------------
\section{Portée des variables}
%-----------------------------------------------------------------------------
%\lstinputlisting[basicstyle=\footnotesize]{ds-algo-03-portees.py}

\begin{minipage}[t]{7cm}\footnotesize
\begin{verbatim}
>>> x = 2
>>> print(x)
2

>>> y = f(x)
>>> print(x)
f 6
2

>>> z = g(x)
>>> print(x)
f 6
g 24
2

>>> t = h(x)
>>> print(x)
f 6
f 18
g 72
h 144
2
\end{verbatim}
\end{minipage}
\hfill
\begin{minipage}[t]{7cm}\footnotesize
\begin{verbatim}
>>> x = 2
>>> print(x)
2

>>> x = f(x)
>>> print(x)
f 6
6

>>> x = g(x)
>>> print(x)
f 18
g 72
72

>>> x = h(x)
>>> print(x)
f 216
f 648
g 2592
h 5184
5184
\end{verbatim}
\end{minipage}


%-----------------------------------------------------------------------------
\section{Exécution d'une fonction itérative}
%-----------------------------------------------------------------------------
%\lstinputlisting[basicstyle=\footnotesize]{ds-algo-03-binome.py}

\noindent\begin{minipage}[t]{10cm}
\begin{enumerate}
\item Il s'agit du tableau de Pascal des coefficients du binôme $(x+y)^n$
	pour les valeurs de $n$ allant de 0 à 6.
\item {\tt c} représente le $p^{i\grave eme}$ coefficient du binôme $(x+y)^n$ :
	$\displaystyle c = C_n^p = \left(\begin{array}{c}p\\n\end{array}\right) =
	\frac{n!}{p!(n-p)!}$.
\end{enumerate}
\end{minipage}
\hfill
\begin{minipage}[t]{5cm}
\begin{verbatim}

>>> for n in range(7): 
        f(n)

1 
1 1 
1 2 1 
1 3 3 1 
1 4 6 4 1 
1 5 10 10 5 1 
1 6 15 20 15 6 1 
\end{verbatim}
\end{minipage}

%-----------------------------------------------------------------------------
\label{fini}
\end{document}
%-----------------------------------------------------------------------------
