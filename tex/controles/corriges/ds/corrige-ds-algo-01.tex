% corrige-ds-algo-01.tex

%-------------------------------------------------------------------------
\documentclass[11pt,a4paper]{article}
%-------------------------------------------------------------------------

%-------------------------------------------------------------------------
%-------------------------------------------------------------------------
% ds-info-S1-preambule.tex
%-------------------------------------------------------------------------

%-------------------------------------------------------------------------
\usepackage{calc}
\usepackage[text={16cm,23cm},centering=true,showframe=false]{geometry}
\usepackage{fancybox,fancyvrb,fancyhdr,lastpage,lineno,import}
\usepackage{longtable,multirow}
\usepackage{xcolor,graphics,xmpmulti,pgf,pgfpages,tikz,wrapfig}
\usepackage{colortbl,color}
\usepackage{amsmath,amssymb,amsfonts}
\usepackage{hyperref,multimedia,rotating,framed,pstricks}
\usepackage{listings,index}
%
%---- pdflatex
%\usepackage[T1]{fontenc}
%\usepackage[utf8]{inputenc}
%---- xelatex
\usepackage{fontspec}
%
%\usepackage[french]{minitoc}
\usepackage[french]{babel}
\usepackage[french]{nomencl}
\usepackage[framed,hyperref,standard]{ntheorem}
\usepackage{eurosym,pifont}
%-------------------------------------------------------------------------

%-------------------------------------------------------------------------
\lstset
{
language=Python,
basicstyle=\ttfamily,
identifierstyle=\ttfamily,
keywordstyle=\color{blue}\ttfamily,
commentstyle=\color{gray}\ttfamily,
stringstyle=\color{green}\ttfamily,
showstringspaces=false,
extendedchars=true,
numbers=left, 
numberstyle=\color{blue}\tiny,
frame=lines,
linewidth=0.95\textwidth,
xleftmargin=5mm
} 
%-------------------------------------------------------------------------

%-------------------------------------------------------------------------
\pgfdeclareimage[width=3cm,interpolate=true]{logo-enib}{logo-enib}
%-------------------------------------------------------------------------

%-------------------------------------------------------------------------
\pagestyle{fancy}
\fancyhead{}
\fancyhead[L]{\hspace*{-3em}\begin{minipage}{3cm}\pgfuseimage{logo-enib}\end{minipage}}
\fancyhead[C]{Informatique S1}
\fancyhead[R]{\thepage/\pageref{LastPage}}
\fancyfoot{}
\fancyfoot[L]{}
\fancyfoot[C]{}
\fancyfoot[R]{}
\setlength{\headheight}{80pt}
\setlength{\footskip}{38pt}
\renewcommand{\headrulewidth}{0pt}
\renewcommand{\footrulewidth}{0pt}
%-------------------------------------------------------------------------

\voffset=-1cm

%-------------------------------------------------------------------------
\def\entete{\noindent\begin{tabular}{|l|l|l|} 
\hline 
 & & \\ 
\makebox[6cm][l]{\bsc{Nom :}} & \makebox[6cm][l]{\bsc{Prénom :}} & \makebox[2.65cm][l]{\bsc{Groupe :}} \\[1mm] 
\hline 
\end{tabular}\\[1mm]
{\footnotesize \textsc{Durée : 90'\hfill Documents, calculettes, téléphones et ordinateurs interdits}}}

\def\notes{\begin{tabular}{|c|c|c|c|}  
\hline 
\makebox[0.5cm]{3} & \makebox[0.5cm]{2} & \makebox[0.5cm]{1} & \makebox[0.5cm]{0} \\  
\hline
\end{tabular}  
} 

\def\autoevaluation{$$\begin{tabular}{|c|c|c|}
\hline
\multicolumn{3}{|c|}{\textbf{Auto-évaluation}} \\
\hline
\textbf{M} & \textbf{V} & \textbf{R} \\
Méthode(s) & Vérification(s) & Résultat(s) \\
\notes & \notes & \notes \\[1mm]
\hline
\end{tabular}$$ $$ $$}

\def\reponse{\mbox{}\hfill \fbox{\huge Réponse page suivante}}
%-------------------------------------------------------------------------

%-------------------------------------------------------------------------
\tikzset{
xmin/.store in=\xmin, xmin/.default=-3, xmin=-3,
xmax/.store in=\xmax, xmax/.default=3,  xmax=3,
ymin/.store in=\ymin, ymin/.default=-3, ymin=-3,
ymax/.store in=\ymax, ymax/.default=3,  ymax=3,
}

\newcommand{\grille}{\draw[color=lightgray] (\xmin,\ymin) grid (\xmax,\ymax);}

\newcommand{\axes}{
	\draw[->] (\xmin,0) -- (\xmax,0);
	\draw[->] (0,\ymin) -- (0,\ymax);
}

\newcommand{\fenetre}{\clip (\xmin,\ymin) rectangle (\xmax,\ymax);}
%-------------------------------------------------------------------------

%-------------------------------------------------------------------------
\def\ga{\textsc{ga}}   
\def\bu{\textsc{bu}} 
\def\zo{\textsc{zo}} 
\def\meu{\textsc{meu}} 
%-------------------------------------------------------------------------

%-------------------------------------------------------------------------
\newenvironment{py}[1]{\begin{minipage}[t]{#1}\footnotesize}{\end{minipage}}
%-------------------------------------------------------------------------

%-------------------------------------------------------------------------
\input{sigle}
%-------------------------------------------------------------------------

\graphicspath{{../../fig/}}



%-------------------------------------------------------------------------

\usepackage{epsfig}

%-----------------------------------------------------------------------------
\begin{document}
%-----------------------------------------------------------------------------

%-----------------------------------------------------------------------------
\section{Fonctions numériques}
%-----------------------------------------------------------------------------

%-----------------------------------------------------------------------------
\subsection{Calcul de $\pi$}
\lstinputlisting[basicstyle=\footnotesize]{ds-algo-01-calculPi.py}

%-----------------------------------------------------------------------------
\subsection{Conversion décimal $\rightarrow$ base $b$}
\lstinputlisting[basicstyle=\footnotesize]{ds-algo-01-conversion.py}

%-----------------------------------------------------------------------------
\section{Fonctions graphiques}
%-----------------------------------------------------------------------------

%-----------------------------------------------------------------------------
\subsection{Courbes paramétrées}
\lstinputlisting[basicstyle=\footnotesize]{ds-algo-01-courbes.py}

%-----------------------------------------------------------------------------
\subsection{Courbes fractales}
\lstinputlisting[basicstyle=\footnotesize]{ds-algo-01-fractals.py}

Les courbes ci-dessous correspondent, respectivement de bas en haut, 
aux appels suivants~:
{\tt p(0,300)}, {\tt p(1,300)}, {\tt p(2,300)} et {\tt p(3,300)}.
\vspace*{1cm}

\centerline{\epsfig{figure=fractal.eps,width=8cm}}

%-----------------------------------------------------------------------------
%\entete
\section{Appels de fonctions}
%-----------------------------------------------------------------------------

%-----------------------------------------------------------------------------
\subsection{Portée des variables}
%\lstinputlisting[basicstyle=\footnotesize]{ds-algo-01-portees.py}

\begin{minipage}[t]{7cm}\footnotesize
\begin{verbatim}
>>> x = 5
>>> print(x)
5

>>> y = f(x)
>>> print(x)
f 10
5

>>> z = g(x)
>>> print(x)
f 10
g 20
5

>>> t = h(x)
>>> print(x)
f 10
f 20
g 40
h 80
5
\end{verbatim}
\end{minipage}
\hfill
\begin{minipage}[t]{7cm}\footnotesize
\begin{verbatim}
>>> x = 5
>>> print(x)
5

>>> x = f(x)
>>> print(x)
f 10
10

>>> x = g(x)
>>> print(x)
f 20
g 40
40

>>> x = h(x)
>>> print(x)
f 80
f 160
g 320
h 640
640
\end{verbatim}
\end{minipage}


%-----------------------------------------------------------------------------
\subsection{Récursivité}
\lstinputlisting[basicstyle=\footnotesize]{ds-algo-01-recherche.py}

%-----------------------------------------------------------------------------
%\entete
\section{Exécutions de fonctions}
%-----------------------------------------------------------------------------

%-----------------------------------------------------------------------------
\subsection{Exécution d'une fonction itérative}

\begin{minipage}[t]{10cm}
Il s'agit de l'algorithme du tri par sélection
et l'appel correspond au tri du tableau {\tt [3,6,4,5,2,1]}.
\end{minipage}
\hfill
\begin{py}{4cm}\tt
>>> f1([3,6,4,5,2,1])\\
0 [1, 6, 4, 5, 2, 3]\\
1 [1, 2, 4, 5, 6, 3]\\
2 [1, 2, 3, 5, 6, 4]\\
3 [1, 2, 3, 4, 6, 5]\\
4 [1, 2, 3, 4, 5, 6]\\
5 [1, 2, 3, 4, 5, 6]\\
\mbox{}[1, 2, 3, 4, 5, 6]\\
>>>
\mbox{}
\end{py}



%-----------------------------------------------------------------------------
\subsection{Exécution d'une fonction récursive}
\begin{minipage}[t]{10cm}

Il s'agit de l'algorithme des tours de Hanoï
et l'appel correspond au déplacement de 3 disques de la tour 4 à la tour 6
en utilisant la tour 5.
\end{minipage}
\hfill
\begin{py}{4cm}\tt
>>> f2(3,4,5,6)\\
4 6\\
4 5\\
6 5\\
4 6\\
5 4\\
5 6\\
4 6\\
>>>
\mbox{}
\end{py}


%-----------------------------------------------------------------------------
\label{fini}
\end{document}
%-----------------------------------------------------------------------------
