% corrige-ds-instruc-06.tex

%-------------------------------------------------------------------------
\documentclass[11pt,a4paper]{article}
%-------------------------------------------------------------------------

%-------------------------------------------------------------------------
%-------------------------------------------------------------------------
% ds-info-S1-preambule.tex
%-------------------------------------------------------------------------

%-------------------------------------------------------------------------
\usepackage{calc}
\usepackage[text={16cm,23cm},centering=true,showframe=false]{geometry}
\usepackage{fancybox,fancyvrb,fancyhdr,lastpage,lineno,import}
\usepackage{longtable,multirow}
\usepackage{xcolor,graphics,xmpmulti,pgf,pgfpages,tikz,wrapfig}
\usepackage{colortbl,color}
\usepackage{amsmath,amssymb,amsfonts}
\usepackage{hyperref,multimedia,rotating,framed,pstricks}
\usepackage{listings,index}
%
%---- pdflatex
%\usepackage[T1]{fontenc}
%\usepackage[utf8]{inputenc}
%---- xelatex
\usepackage{fontspec}
%
%\usepackage[french]{minitoc}
\usepackage[french]{babel}
\usepackage[french]{nomencl}
\usepackage[framed,hyperref,standard]{ntheorem}
\usepackage{eurosym,pifont}
%-------------------------------------------------------------------------

%-------------------------------------------------------------------------
\lstset
{
language=Python,
basicstyle=\ttfamily,
identifierstyle=\ttfamily,
keywordstyle=\color{blue}\ttfamily,
commentstyle=\color{gray}\ttfamily,
stringstyle=\color{green}\ttfamily,
showstringspaces=false,
extendedchars=true,
numbers=left, 
numberstyle=\color{blue}\tiny,
frame=lines,
linewidth=0.95\textwidth,
xleftmargin=5mm
} 
%-------------------------------------------------------------------------

%-------------------------------------------------------------------------
\pgfdeclareimage[width=3cm,interpolate=true]{logo-enib}{logo-enib}
%-------------------------------------------------------------------------

%-------------------------------------------------------------------------
\pagestyle{fancy}
\fancyhead{}
\fancyhead[L]{\hspace*{-3em}\begin{minipage}{3cm}\pgfuseimage{logo-enib}\end{minipage}}
\fancyhead[C]{Informatique S1}
\fancyhead[R]{\thepage/\pageref{LastPage}}
\fancyfoot{}
\fancyfoot[L]{}
\fancyfoot[C]{}
\fancyfoot[R]{}
\setlength{\headheight}{80pt}
\setlength{\footskip}{38pt}
\renewcommand{\headrulewidth}{0pt}
\renewcommand{\footrulewidth}{0pt}
%-------------------------------------------------------------------------

\voffset=-1cm

%-------------------------------------------------------------------------
\def\entete{\noindent\begin{tabular}{|l|l|l|} 
\hline 
 & & \\ 
\makebox[6cm][l]{\bsc{Nom :}} & \makebox[6cm][l]{\bsc{Prénom :}} & \makebox[2.65cm][l]{\bsc{Groupe :}} \\[1mm] 
\hline 
\end{tabular}\\[1mm]
{\footnotesize \textsc{Durée : 90'\hfill Documents, calculettes, téléphones et ordinateurs interdits}}}

\def\notes{\begin{tabular}{|c|c|c|c|}  
\hline 
\makebox[0.5cm]{3} & \makebox[0.5cm]{2} & \makebox[0.5cm]{1} & \makebox[0.5cm]{0} \\  
\hline
\end{tabular}  
} 

\def\autoevaluation{$$\begin{tabular}{|c|c|c|}
\hline
\multicolumn{3}{|c|}{\textbf{Auto-évaluation}} \\
\hline
\textbf{M} & \textbf{V} & \textbf{R} \\
Méthode(s) & Vérification(s) & Résultat(s) \\
\notes & \notes & \notes \\[1mm]
\hline
\end{tabular}$$ $$ $$}

\def\reponse{\mbox{}\hfill \fbox{\huge Réponse page suivante}}
%-------------------------------------------------------------------------

%-------------------------------------------------------------------------
\tikzset{
xmin/.store in=\xmin, xmin/.default=-3, xmin=-3,
xmax/.store in=\xmax, xmax/.default=3,  xmax=3,
ymin/.store in=\ymin, ymin/.default=-3, ymin=-3,
ymax/.store in=\ymax, ymax/.default=3,  ymax=3,
}

\newcommand{\grille}{\draw[color=lightgray] (\xmin,\ymin) grid (\xmax,\ymax);}

\newcommand{\axes}{
	\draw[->] (\xmin,0) -- (\xmax,0);
	\draw[->] (0,\ymin) -- (0,\ymax);
}

\newcommand{\fenetre}{\clip (\xmin,\ymin) rectangle (\xmax,\ymax);}
%-------------------------------------------------------------------------

%-------------------------------------------------------------------------
\def\ga{\textsc{ga}}   
\def\bu{\textsc{bu}} 
\def\zo{\textsc{zo}} 
\def\meu{\textsc{meu}} 
%-------------------------------------------------------------------------

%-------------------------------------------------------------------------
\newenvironment{py}[1]{\begin{minipage}[t]{#1}\footnotesize}{\end{minipage}}
%-------------------------------------------------------------------------

%-------------------------------------------------------------------------
\input{sigle}
%-------------------------------------------------------------------------

\graphicspath{{../../fig/}}



\usepackage{epsfig}
%-------------------------------------------------------------------------

%-----------------------------------------------------------------------------
\begin{document}
%-----------------------------------------------------------------------------

%-----------------------------------------------------------------------------
\section{Exécution d'une séquence d'instructions}
%-----------------------------------------------------------------------------
%\lstinputlisting[basicstyle=\footnotesize]{ds-instruc-06-racine.py}

\noindent\begin{minipage}{8cm}
Il s'agit du calcul de la racine carrée entière $y$ d'un nombre entier $a$ :
$y = 27 = \sqrt{729} = \sqrt{a}$.
\end{minipage}
\hfill
\begin{py}{7cm}
\begin{verbatim}
729 1 729 1 0
729 1024 729 1 0
729 256 473 768 384
729 64 153 448 224
729 16 153 208 104
729 4 53 108 54
729 1 0 55 27
729 1 0 55 27
\end{verbatim}
\end{py}

%-----------------------------------------------------------------------------
\section{Calcul de $\pi$}
%-----------------------------------------------------------------------------
\lstinputlisting[basicstyle=\footnotesize]{ds-instruc-06-calculPi.py}

%-----------------------------------------------------------------------------
\section{Zéro d'une fonction}
%-----------------------------------------------------------------------------
\lstinputlisting[basicstyle=\footnotesize]{ds-instruc-06-zero.py}

%-----------------------------------------------------------------------------
\newpage
\section{Le calcul Shadok}
%-----------------------------------------------------------------------------
Le système Shadok est un système de numération en base 4 :
{\sc ga} = 0, {\sc bu} = 1, {\sc zo} = 2 et {\sc meu} = 3.
\begin{enumerate}
\item conversions «~base Shadok~» $\rightarrow$ décimal :

	\begin{minipage}[t]{14cm}	
	\begin{tabular}{l@{ = }lp{6cm}}
	{\sc ga} {\sc ga} & $(00)_4$ & $= 0$\\[2mm]
	{\sc bu} {\sc bu} {\sc bu} & $(111)_4$ & $= 1\cdot 4^2 + 1\cdot 4^1 + 1\cdot 4^0$\newline
                                     $= 16 + 4 + 1$\newline $= 21$\\[2mm]
	{\sc zo} {\sc zo} {\sc zo} {\sc zo} & $(2222)_4$ & $= 2\cdot 4^3 + 2\cdot 4^2 + 2\cdot 4^1 + 2\cdot 4^0$\newline 
                                              $= 128 + 32 + 8 + 2$\newline $= 170$\\[2mm]
	{\sc meu} {\sc meu} {\sc meu} {\sc meu} {\sc meu} & $(33333)_4$ & $= 3 \cdot 4^4 + 3\cdot 4^3 + 3\cdot 4^2 + 3\cdot 4^1 + 3\cdot 4^0$\newline $= 768 + 192 + 48 + 12 + 3$\newline $= 1023$
	\end{tabular}
	\end{minipage}
\item calculs en «~base Shadok~» :

	\begin{minipage}[t]{14cm}
	\begin{tabular}{l@{ = }l}
	{\sc zo} {\sc zo} {\sc meu} $+$ {\sc bu} {\sc ga} {\sc meu} & $(223)_4 + (103)_4 = (332)_4 = 43 + 19 = 62$\\[2mm]
	{\sc meu} {\sc ga} {\sc meu} $-$ {\sc bu} {\sc meu} {\sc ga} & $(303)_4 - (130)_4 = (113)_4 = 51 - 28 = 23$\\[2mm]
	{\sc zo} {\sc meu} {\sc meu} $\times$ {\sc bu} {\sc ga} {\sc meu} & $(233)_4 \times (103)_4 = (31331)_4 = 47 \times 19 = 893$\\[2mm]
	{\sc zo} {\sc zo} {\sc zo} {\sc meu} $\div$ {\sc bu} {\sc ga} {\sc zo} & $(2223)_4 \div (102)_4 = (21)_4 = 171 \div 18 = 9$
	\end{tabular}
	\end{minipage}
\end{enumerate}


%-----------------------------------------------------------------------------
\label{fini}
\end{document}
%-----------------------------------------------------------------------------
