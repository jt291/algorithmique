%-------------------------------------------------------------------------
% ds-info-S1-affectation.tex
%-------------------------------------------------------------------------

%-------------------------------------------------------------------------
\documentclass[11pt,a4paper]{article}
%-------------------------------------------------------------------------

%-------------------------------------------------------------------------
%-------------------------------------------------------------------------
% ds-info-S1-preambule.tex
%-------------------------------------------------------------------------

%-------------------------------------------------------------------------
\usepackage{calc}
\usepackage[text={16cm,23cm},centering=true,showframe=false]{geometry}
\usepackage{fancybox,fancyvrb,fancyhdr,lastpage,lineno,import}
\usepackage{longtable,multirow}
\usepackage{xcolor,graphics,xmpmulti,pgf,pgfpages,tikz,wrapfig}
\usepackage{colortbl,color}
\usepackage{amsmath,amssymb,amsfonts}
\usepackage{hyperref,multimedia,rotating,framed,pstricks}
\usepackage{listings,index}
%
%---- pdflatex
%\usepackage[T1]{fontenc}
%\usepackage[utf8]{inputenc}
%---- xelatex
\usepackage{fontspec}
%
\usepackage[french]{minitoc}
\usepackage[french]{babel}
\usepackage[french]{nomencl}
\usepackage[framed,hyperref,standard]{ntheorem}
\usepackage{eurosym,pifont}
%-------------------------------------------------------------------------

%-------------------------------------------------------------------------
\lstset
{
language=Python,
basicstyle=\ttfamily,
identifierstyle=\ttfamily,
keywordstyle=\color{blue}\ttfamily,
commentstyle=\color{gray}\ttfamily,
stringstyle=\color{green}\ttfamily,
showstringspaces=false,
extendedchars=true,
numbers=left, 
numberstyle=\color{blue}\tiny,
frame=lines,
linewidth=0.95\textwidth,
xleftmargin=5mm
} 
%-------------------------------------------------------------------------

%-------------------------------------------------------------------------
\pgfdeclareimage[width=3cm,interpolate=true]{logo-enib}{logo-enib}
%-------------------------------------------------------------------------

%-------------------------------------------------------------------------
\pagestyle{fancy}
\fancyhead{}
\fancyhead[L]{\hspace*{-3em}\begin{minipage}{3cm}\pgfuseimage{logo-enib}\end{minipage}}
\fancyhead[C]{Informatique S1}
\fancyhead[R]{\thepage/\pageref{LastPage}}
\fancyfoot{}
\fancyfoot[L]{}
\fancyfoot[C]{}
\fancyfoot[R]{}
\setlength{\headheight}{80pt}
\setlength{\footskip}{38pt}
\renewcommand{\headrulewidth}{0pt}
\renewcommand{\footrulewidth}{0pt}
%-------------------------------------------------------------------------


%-------------------------------------------------------------------------
\def\entete{\noindent\begin{tabular}{|l|l|l|l|} 
\hline 
 & & & \\ 
\makebox[4.55cm][l]{\bsc{Nom :}} & \makebox[4.5cm][l]{\bsc{Prénom :}} & \makebox[2.5cm][l]{\bsc{Groupe :}} & \makebox[2.75cm][l]{\bsc{Question :}} \\[1mm] 
\hline 
\end{tabular}\\[1mm]
{\footnotesize \textsc{Durée : 15'\hfill Documents, calculettes, téléphones et ordinateurs interdits}}}

\def\notes{\begin{tabular}{|c|c|c|c|}  
\hline 
\makebox[0.5cm]{3} & \makebox[0.5cm]{2} & \makebox[0.5cm]{1} & \makebox[0.5cm]{0} \\  
\hline
\end{tabular}  
} 

\def\autoevaluation{$$\begin{tabular}{|c|c|c|}
\hline
\multicolumn{3}{|c|}{\textbf{Auto-évaluation}} \\
\hline
\textbf{M} & \textbf{V} & \textbf{R} \\
Méthode(s) & Vérification(s) & Résultat(s) \\
\notes & \notes & \notes \\[1mm]
\hline
\end{tabular}$$ $$ $$}

\def\reponse{\mbox{}\hfill \fbox{\huge Réponse page suivante}}
%-------------------------------------------------------------------------

%-------------------------------------------------------------------------
\tikzset{
xmin/.store in=\xmin, xmin/.default=-3, xmin=-3,
xmax/.store in=\xmax, xmax/.default=3,  xmax=3,
ymin/.store in=\ymin, ymin/.default=-3, ymin=-3,
ymax/.store in=\ymax, ymax/.default=3,  ymax=3,
}

\newcommand{\grille}{\draw[color=lightgray] (\xmin,\ymin) grid (\xmax,\ymax);}

\newcommand{\axes}{
	\draw[->] (\xmin,0) -- (\xmax,0);
	\draw[->] (0,\ymin) -- (0,\ymax);
}

\newcommand{\fenetre}{\clip (\xmin,\ymin) rectangle (\xmax,\ymax);}
%-------------------------------------------------------------------------

%-------------------------------------------------------------------------
\def\ga{\textsc{ga}}   
\def\bu{\textsc{bu}} 
\def\zo{\textsc{zo}} 
\def\meu{\textsc{meu}} 
%-------------------------------------------------------------------------


%-------------------------------------------------------------------------
\input{sigle}
%-------------------------------------------------------------------------

\graphicspath{{../../fig/}}



%-------------------------------------------------------------------------

%-------------------------------------------------------------------------
\begin{document}
%-------------------------------------------------------------------------
%\noindent\framebox[\textwidth]{}
\entete

\autoevaluation


$$\mbox{\textbf{\large Facteurs de conversion}}$$


%-------------------------------------------------------------------------
\paragraph{Contexte :} Le système international d'unités (\si) en physique est 
composé de 7 unités de base et de 2 unités supplémentaires définies dans le
tableau ci-dessous :
{\footnotesize
$$\begin{tabular}{|l|l|c|}
\hline
\textbf{Grandeur} & \textbf{Unité} & \textbf{Symbole}\\ 
\hline 
longueur						& mètre			& m \\ 
masse 							& kilogramme	& kg \\ 
temps							& seconde		& s \\ 
intensité de courant électrique & ampère		& A \\ 
température thermodynamique 	& kelvin		& K \\ 
quantité de matière 			& mole			& mol \\ 
intensité lumineuse 			& candela		& cd \\ 
\hline 
angle plan 						& radian		& rad \\ 
angle solide					& stéradian		& sr \\ 
\hline 
\end{tabular}$$}

\noindent Pour des raisons historiques, culturelles ou pragmatiques, un certain nombre 
d'unités hors-système, telles que l'heure (h), le grade (gr),
le pound (lb), le n\oe ud (kn), l'année-lumière (al) ou encore l'électronvolt (eV), 
peuvent être utilisées. Il est par contre nécessaire de connaître leur facteur 
de conversion en unités \si. Le tableau ci-dessous donne les facteurs de conversion
des unités les plus connues.
{\footnotesize
$$\begin{tabular}{|l|l@{ = }r@{ $\times$ }ll|l|}  
\hline
\bf Grandeur    & \multicolumn{4}{c|}{\textbf{Facteur de conversion}} & \textbf{Domaine} \\ 
\hline
année-lumière 	& 1 al 		& $9.46053$	& $10^{15}$ 	& $\textrm{m}$ 		& astronomie\\ 
atmosphère		& 1 atm 	& $1.01325$	& $10^{5}$  	& $\textrm{Pa}$ 	& météorologie \\
baril			& 1 b		& $0.15891$	& $10^{0}$		& $\textrm{m}^3$	& pétrole \\
calorie			& 1 cal		& $4.184$	& $10^{0}$		& $\textrm{J}$ 		& thermique \\
cheval-vapeur	& 1 ch		& $735.499$	& $10^{0}$		& $\textrm{W}$ 		& mécanique \\
curie			& 1 Ci		& $3.7$		& $10^{10}$ 	& $\textrm{Bq}$ 	& radioactivité \\
degré			& 1 $^\circ$& $1.745329$& $10^{-2}$   	& $\textrm{rad}$ 	& géométrie \\ 
électronvolt	& 1 eV		& $1.602189$& $10^{-19}$    & $\textrm{J}$ 		& physique nucléaire \\
faraday 		& 1 F		& $9.64870$	& $10^{4}$		& $\textrm{C}$ 		& électricité \\ 
foot			& 1 ft		& $30.48$	& $10^{-2}$		& $\textrm{m}$ 		& géométrie \\
franklin		& 1 Fr		& $3.33564$	& $10^{-10}$	& $\textrm{C}$ 		& électricité \\
frigorie		& 1 fg		& $4.186$	& $10^{3}$		& $\textrm{J}$ 		& thermique \\ 
gallon			& 1 gal		& $3.78541$	& $10^{-3}$		& $\textrm{m}^3$	& volume \\
grade			& 1 gr		& $1.570796$& $10^{-2}$		& $\textrm{rad}$ 	& géométrie \\
heure			& 1 h		& $3.6$		& $10^{3}$		& $\textrm{s}$ 		& temps \\
inch			& 1 in		& $2.54$	& $10^{-2}$		& $\textrm{m}$ 		& géométrie \\
lambert			& 1 L		& $3.183$	& $10^{3}$		& $\textrm{cd}\cdot\textrm{m}^{-2}$ 	& photométrie \\
mile			& 1 mile	& $1.609344$& $10^{3}$		& $\textrm{m}$ 		& géométrie \\ 
millimètre de mercure & 1 mmHg & $133.3224$ & $10^{0}$	& $\textrm{Pa}$ 	& météorologie \\
n\oe ud			& 1 nd		& $0.514444$& $10^{0}$		& $\textrm{m}\cdot\textrm{s}^{-1}$ 		& marine \\
oersted			& 1 Oe		& $79.57747$& $10^{0}$		& $\textrm{A}\cdot\textrm{m}^{-1}$ 		& magnétisme \\
parsec			& 1 pc		& $3.0857$	& $10^{16}$		& $\textrm{m}$ 		& astronomie\\ 
pica			& 1 pica	& $4.2175$	& $10^{-3}$		& $\textrm{m}$ 		& typographie\\ 
torr			& 1 Torr	& $133.3224$& $10^{0}$		& $\textrm{Pa}$		& météorologie \\
\hline
\end{tabular}$$ 
}

\paragraph{Objectif :} utiliser l'affectation pour calculer un facteur de conversion entre 
deux unités physiques comptatibles.

%-------------------------------------------------------------------------
\paragraph{Question :}
Ecrire une affectation qui calcule les facteurs de conversion suivants
(chaque étudiant traite une seule question : celle correspondant à son numéro d'ordre dans la liste de \textsc{Labo}).

\noindent\begin{minipage}[t]{7cm}
\begin{enumerate}
\item année en minute
\item année-lumière en mile
\item baril en gallon
\item année-lumière en parsec
\item litre en gallon
\item parsec en inch
\item inch en pica
\item mile en foot
\item frigorie en calorie
\item centimètre en pica
\item électronvolt en frigorie
\item franklin en faraday
\end{enumerate}
\end{minipage}
\hfill
\begin{minipage}[t]{7cm}
\begin{enumerate}\setcounter{enumi}{12}
\item baril en litre
\item parsec en foot
\item année en seconde
\item électronvolt en calorie
\item atmosphère en torr
\item parsec en année-lumière
\item foot en inch
\item n\oe ud en kilomètre par heure
\item mile en inch
\item année-lumière en pica
%\item torr en millimètre de mercure
\item inch en foot
\item litre en baril
\end{enumerate}
\end{minipage}

%-------------------------------------------------------------------------
\paragraph{Réponse :}\mbox{}

\noindent\framebox[0.9955\textwidth]{$\rule{0cm}{0.5\textheight}$}

%-------------------------------------------------------------------------
\end{document}
%-------------------------------------------------------------------------

