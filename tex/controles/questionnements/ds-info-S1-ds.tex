%-------------------------------------------------------------------------
% ds-info-S1-ds.tex
%-------------------------------------------------------------------------

%-------------------------------------------------------------------------
\documentclass[11pt,a4paper]{article}
%-------------------------------------------------------------------------

%-------------------------------------------------------------------------
%-------------------------------------------------------------------------
% ds-info-S1-preambule.tex
%-------------------------------------------------------------------------

%-------------------------------------------------------------------------
\usepackage{calc}
\usepackage[text={16cm,23cm},centering=true,showframe=false]{geometry}
\usepackage{fancybox,fancyvrb,fancyhdr,lastpage,lineno,import}
\usepackage{longtable,multirow}
\usepackage{xcolor,graphics,xmpmulti,pgf,pgfpages,tikz,wrapfig}
\usepackage{colortbl,color}
\usepackage{amsmath,amssymb,amsfonts}
\usepackage{hyperref,multimedia,rotating,framed,pstricks}
\usepackage{listings,index}
%
%---- pdflatex
%\usepackage[T1]{fontenc}
%\usepackage[utf8]{inputenc}
%---- xelatex
\usepackage{fontspec}
%
\usepackage[french]{minitoc}
\usepackage[french]{babel}
\usepackage[french]{nomencl}
\usepackage[framed,hyperref,standard]{ntheorem}
\usepackage{eurosym,pifont}
%-------------------------------------------------------------------------

%-------------------------------------------------------------------------
\lstset
{
language=Python,
basicstyle=\ttfamily,
identifierstyle=\ttfamily,
keywordstyle=\color{blue}\ttfamily,
commentstyle=\color{gray}\ttfamily,
stringstyle=\color{green}\ttfamily,
showstringspaces=false,
extendedchars=true,
numbers=left, 
numberstyle=\color{blue}\tiny,
frame=lines,
linewidth=0.95\textwidth,
xleftmargin=5mm
} 
%-------------------------------------------------------------------------

%-------------------------------------------------------------------------
\pgfdeclareimage[width=3cm,interpolate=true]{logo-enib}{logo-enib}
%-------------------------------------------------------------------------

%-------------------------------------------------------------------------
\pagestyle{fancy}
\fancyhead{}
\fancyhead[L]{\hspace*{-3em}\begin{minipage}{3cm}\pgfuseimage{logo-enib}\end{minipage}}
\fancyhead[C]{Informatique S1}
\fancyhead[R]{\thepage/\pageref{LastPage}}
\fancyfoot{}
\fancyfoot[L]{}
\fancyfoot[C]{}
\fancyfoot[R]{}
\setlength{\headheight}{80pt}
\setlength{\footskip}{38pt}
\renewcommand{\headrulewidth}{0pt}
\renewcommand{\footrulewidth}{0pt}
%-------------------------------------------------------------------------


%-------------------------------------------------------------------------
\def\entete{\noindent\begin{tabular}{|l|l|l|l|} 
\hline 
 & & & \\ 
\makebox[4.55cm][l]{\bsc{Nom :}} & \makebox[4.5cm][l]{\bsc{Prénom :}} & \makebox[2.5cm][l]{\bsc{Groupe :}} & \makebox[2.75cm][l]{\bsc{Question :}} \\[1mm] 
\hline 
\end{tabular}\\[1mm]
{\footnotesize \textsc{Durée : 15'\hfill Documents, calculettes, téléphones et ordinateurs interdits}}}

\def\notes{\begin{tabular}{|c|c|c|c|}  
\hline 
\makebox[0.5cm]{3} & \makebox[0.5cm]{2} & \makebox[0.5cm]{1} & \makebox[0.5cm]{0} \\  
\hline
\end{tabular}  
} 

\def\autoevaluation{$$\begin{tabular}{|c|c|c|}
\hline
\multicolumn{3}{|c|}{\textbf{Auto-évaluation}} \\
\hline
\textbf{M} & \textbf{V} & \textbf{R} \\
Méthode(s) & Vérification(s) & Résultat(s) \\
\notes & \notes & \notes \\[1mm]
\hline
\end{tabular}$$ $$ $$}

\def\reponse{\mbox{}\hfill \fbox{\huge Réponse page suivante}}
%-------------------------------------------------------------------------

%-------------------------------------------------------------------------
\tikzset{
xmin/.store in=\xmin, xmin/.default=-3, xmin=-3,
xmax/.store in=\xmax, xmax/.default=3,  xmax=3,
ymin/.store in=\ymin, ymin/.default=-3, ymin=-3,
ymax/.store in=\ymax, ymax/.default=3,  ymax=3,
}

\newcommand{\grille}{\draw[color=lightgray] (\xmin,\ymin) grid (\xmax,\ymax);}

\newcommand{\axes}{
	\draw[->] (\xmin,0) -- (\xmax,0);
	\draw[->] (0,\ymin) -- (0,\ymax);
}

\newcommand{\fenetre}{\clip (\xmin,\ymin) rectangle (\xmax,\ymax);}
%-------------------------------------------------------------------------

%-------------------------------------------------------------------------
\def\ga{\textsc{ga}}   
\def\bu{\textsc{bu}} 
\def\zo{\textsc{zo}} 
\def\meu{\textsc{meu}} 
%-------------------------------------------------------------------------


%-------------------------------------------------------------------------
\input{sigle}
%-------------------------------------------------------------------------

\graphicspath{{../../fig/}}



%-------------------------------------------------------------------------

%-------------------------------------------------------------------------
\begin{document}
%-------------------------------------------------------------------------
%$$\mbox{\Large \textbf{DS : algorithmique}}$$

%-------------------------------------------------------------------------
\section{Boucle simple}
\entete{25'}
\vspace*{5mm}
%-------------------------------------------------------------------------
%\autoevaluation

\paragraph{Enoncé :} Soit $f(x)$ une fonction continue de $R \rightarrow R$ à intégrer sur $[a,b]$ 
(on supposera que $f$ à toutes les bonnes propriétés mathématiques pour être
intégrable sur l'intervalle considéré). On cherche à calculer son intégrale
$\displaystyle I = \int_a^b f(x)dx$ qui représente classiquement l'aire
comprise entre la courbe représentative de $f$ et les droites d'équations 
$x=a$, $x=b$ et $y=0$. Les méthodes classiques d'intégration numérique (méthode des rectangles, 
méthode des trapèzes et méthode de Simpson) consistent 
essentiellement à trouver une bonne approximation de cette aire.


Dans la méthode des rectangles, on subdivise l'intervalle d'intégration de
	longueur $b-a$ en $n$ parties égales de longueur 
	$\displaystyle\delta x = \frac{b-a}{n}$. Soient $x_1$, $x_2$, \ldots,$x_i$, \ldots,
	$x_n$ les points milieux de ces $n$ intervalles. Les $n$ rectangles
	formés avec les ordonnées correspondantes ont pour surface $f(x_1)\delta
	x$, $f(x_2)\delta x$, \ldots, $f(x_i)\delta x$, \ldots, $f(x_n)\delta x$. L'aire sous la courbe 
	est alors assimilée à la somme des aires de ces rectangles, soit :
	$$\displaystyle I = \int_a^b f(x)dx \approx
	\left(f(x_1)+f(x_2)+\cdots+f(x_i)+\cdots+f(x_n)\right)\delta x$$ 
	C'est la formule dite
	des rectangles qui repose sur une approximation par une fonction {en
	escalier}.
$$\begin{tikzpicture}[xmin=-1,xmax=4.2,ymin=-0.25,ymax=1.25,xscale=2,yscale=3]
%\grille 
\axes 
\fenetre
\draw[thick,smooth,color=blue] plot (\x,{sin(\x r)});
\draw[dashed] (0.5,-0.5) -- (0.5,1.25);
\draw[dashed] (2.5,-0.5) -- (2.5,1.25);
\draw (0.5,0) node {$\bullet$};
\draw (0.5,-0.1) node[left] {$a$};
\draw (2.5,0) node {$\bullet$};
\draw (2.5,-0.1) node[right] {$b$};
\draw (4,-0.1) node {$x$};
\draw (0,1.15) node[left] {$f(x)$};
\foreach \x in {0.5,0.75,1,...,2.25} \draw (\x,0) rectangle (\x+0.25,{sin((\x+0.125) r)});
\draw (1.125,0) node[below] {$\delta x$};
\draw[dotted] (1,-0.25) -- (1,0);
\draw[dotted] (1.25,-0.25) -- (1.25,0);
\draw[->] (0.875,-0.125) -- (1,-0.125);
\draw[->] (1.375,-0.125) -- (1.25,-0.125);
\foreach \x in {0.625,0.875,...,2.375} \draw (\x,-0.025) -- (\x,0.025);
\draw (1.875,0) node[below] {$x_i$};
\end{tikzpicture}$$

	
\paragraph{Question :}
Définir la fonction \texttt{integration} qui calcule l'intégrale $I$ d'une fonction 
	\texttt{f} sur \texttt{[a,b]} à l'ordre \texttt{n} par la méthode des rectangles.
\vspace*{2mm}

\noindent
\begin{minipage}[t]{7cm}\footnotesize
\begin{Verbatim}
>>> integration(cos,0.,pi,10000)
4.307866381890461e-16
>>> integration(cos,-pi/2,pi/2,10000)
2.000000008224676
>>> integration(sin,0.,pi,10000)
2.000000008224676
>>> integration(exp,0.,1.,10000)
1.7182818277430947
>>> integration(log,1.,2.,100)
0.38629644443195715
\end{Verbatim}
\end{minipage}
\hfill
\begin{minipage}[t]{7cm}\footnotesize
\begin{Verbatim}
>>> integration(lambda x: 3,1.,2.,10)
3.0
>>> integration(lambda x: 3,2.,1.,10)
-3.0
>>> integration(lambda x: x,-1.,1.,10)
8.881784197001253e-17
>>> integration(lambda x: x*x,-1.,1.,100)
0.6666000000000001
>>> integration(lambda x: x**3,0.,1.,100)
0.24998750000000006
\end{Verbatim}
\end{minipage}

\newpage
\paragraph{Réponse :}\mbox{}\\
\noindent\framebox[0.99\textwidth]{$\rule{0cm}{22cm}$}

%-------------------------------------------------------------------------
\section{Appels récursifs}
\entete{25'}
\vspace*{5mm}
%-------------------------------------------------------------------------
%\autoevaluation

\paragraph{Enoncé :} On considère les fonctions \texttt{f} et \texttt{g} ci-dessous qui utilisent
les primitives classiques de la tortue \textsc{Logo} (module \texttt{turtle}) située en $(x,y)$
avec une orientation $\theta$ par rapport à l'horizontale.


{\footnotesize
$$\begin{tabular}{l@{ : }p{4cm}c}
\texttt{forward(distance)} 	& Move the turtle forward by the specified \texttt{distance}
								(integer or float), in the direction the turtle is headed. 
							& 
	\begin{minipage}{3.5cm}
	\begin{tikzpicture}
	\draw[color=lightgray] (0,0) -- (2.5,0);
	\draw[color=lightgray] (0,0) -- (2,2);
	\draw (0,0) node {$\bullet$};
	\draw[color=lightgray] (1.5,1.5) node {$\bullet$};
	\draw (0,0) node[left] {$(x,y)$};
	\draw[thick,->] (0,0) -- (0.75,0.75);
	\draw[->] (0.5,0) arc (0:45:0.5);
	\draw (0.4,0.25) node[right] {$\theta$};
	\draw[color=lightgray] (0,0) -- (-0.25,0.25);
	\draw[color=lightgray] (1.5,1.5) -- (1.25,1.75);
	\draw[color=lightgray,<->] (-0.15,0.15) -- (1.35,1.65);
	\draw (0.5,1) node[rotate=45] {\scriptsize distance};
	\end{tikzpicture}
	\end{minipage} \\
\texttt{right(angle)}		& Turn turtle right by \texttt{angle} (integer or float) units.
								Units are by default degrees. 
							& 
	\begin{minipage}{3.5cm}
	\begin{tikzpicture}
	\draw[color=lightgray] (0,0) -- (2.5,0);
	\draw[color=lightgray] (0,0) -- (2,2);
	\draw (0,0) node {$\bullet$};
	\draw (0,0) node[left] {$(x,y)$};
	\draw[thick,->] (0,0) -- (0.75,0.75);
	\draw[->] (0.5,0) arc (0:45:0.5);
	\draw (0.4,0.25) node[right] {$\theta$};
	\draw[color=lightgray] (0,0) -- ({2*cos(15)},{-2*sin(15)});
	\draw[color=lightgray,<-] ({1.5*cos(15)},{-1.5*sin(15)}) arc (-15:45:1.5);
	\draw (1.85,0.5) node[left] {\scriptsize angle};
	\end{tikzpicture}
	\end{minipage} \\
\texttt{circle(radius)}		& Draw a circle with given \texttt{radius} (integer or float). 
								The center $C$ is radius units left of the turtle. 
							& 
	\begin{minipage}{6cm}
	\begin{tikzpicture}
	\draw[color=lightgray] (0,0) -- (2.5,0);
	\draw[color=lightgray] (0,0) -- (2,2);
	\draw (0,0) node {$\bullet$};
	\draw (0,0) node[left] {$(x,y)$};
	\draw[thick,->] (0,0) -- (0.75,0.75);
	\draw[->] (0.5,0) arc (0:45:0.5);
	\draw (0.4,0.25) node[right] {$\theta$};
	\draw[color=lightgray] (0,0) -- (-1.5,1.5);
	%\draw[color=lightgray] (0,0) -- (-1.5,1.5);
	\draw[color=lightgray] (-0.25,0.25) -- (0,0.5) -- (0.25,0.25);
	\draw[color=lightgray] (-1.25,1.25) node {$\bullet$};
	\draw (-1.25,1.25) node[above] {$C$};
	\draw (-0.55,0.75) node[rotate=-45] {\scriptsize radius};
	\draw[color=lightgray,->] (0,0) arc (-45:360:1.7678);
	\end{tikzpicture}
	\end{minipage} 
\end{tabular}$$
}
    
\noindent
\begin{minipage}[t]{7cm}\footnotesize
\begin{lstlisting}
from turtle import *

def f(n,m,d) :
    assert type(n) is int
    assert n >= 0
    assert type(m) is int
    assert m >= 0
    assert type(d) is float
    assert d >= 0.0
    
    for i in range(n):
        g(m,d)
        right(360/n)
    return
\end{lstlisting}
\end{minipage}
\hfill
\begin{minipage}[t]{7cm}\footnotesize
\begin{lstlisting}
def g(m,d) :
    assert type(m) is int
    assert m >= 0
    assert type(d) is float
    assert d >= 0.0
    
    if m > 0 :
        g(m-1,d/2)
        circle(d/2)
        g(m-1,d/2)
    else :
        forward(d)
    return
\end{lstlisting}
\end{minipage}
\vspace*{2mm}

\paragraph{Question :}
Que dessinent les appels suivants ?\\
\begin{minipage}[t]{5cm}
\begin{enumerate}
\item \texttt{>{>}> f(1,0,100)}
\item \texttt{>{>}> f(1,1,100)}
\end{enumerate}
\end{minipage}
\hfill
\begin{minipage}[t]{5cm}
\begin{enumerate}\setcounter{enumi}{2}
\item \texttt{>{>}> f(1,2,100)}
\item \texttt{>{>}> f(2,2,100)}
\end{enumerate}
\end{minipage}
\hfill
\begin{minipage}[t]{5cm}
\begin{enumerate}\setcounter{enumi}{4}
\item \texttt{>{>}> f(3,2,100)}
\item \texttt{>{>}> f(4,3,100)}
\end{enumerate}
\end{minipage}

\newpage
\paragraph{Réponses :}\mbox{}\\
\noindent
\framebox[7.5cm]{\begin{minipage}[t]{7cm}\footnotesize
\begin{enumerate}
\item \texttt{>{>}> f(1,0,100)}
\end{enumerate}
$\rule{0cm}{5.5cm}$
\end{minipage}}
\hfill
\framebox[7.5cm]{\begin{minipage}[t]{7cm}\footnotesize
\begin{enumerate}\setcounter{enumi}{1}
\item \texttt{>{>}> f(1,1,100)}
\end{enumerate}
$\rule{0cm}{5.5cm}$
\end{minipage}}
\vspace*{2mm}

\noindent
\framebox[7.5cm]{\begin{minipage}[t]{7cm}\footnotesize
\begin{enumerate}\setcounter{enumi}{2}
\item \texttt{>{>}> f(1,2,100)}
\end{enumerate}
$\rule{0cm}{7cm}$
\end{minipage}}
\hfill
\framebox[7.5cm]{\begin{minipage}[t]{7cm}\footnotesize
\begin{enumerate}\setcounter{enumi}{3}
\item \texttt{>{>}> f(2,2,100)}
\end{enumerate}
$\rule{0cm}{7cm}$
\end{minipage}}
\vspace*{2mm}

\noindent
\framebox[7.5cm]{\begin{minipage}[t]{7cm}\footnotesize
\begin{enumerate}\setcounter{enumi}{4}
\item \texttt{>{>}> f(3,2,100)}
\end{enumerate}
$\rule{0cm}{7cm}$
\end{minipage}}
\hfill
\framebox[7.5cm]{\begin{minipage}[t]{7cm}\footnotesize
\begin{enumerate}\setcounter{enumi}{5}
\item \texttt{>{>}> f(4,3,100)}
\end{enumerate}
$\rule{0cm}{7cm}$
\end{minipage}}
\vspace*{2mm}


\newpage
%-------------------------------------------------------------------------
\section{Liste de listes}
\entete{25'}
\vspace*{5mm}
%-------------------------------------------------------------------------

\paragraph{Enoncé :}
Le taquin est un jeu en forme de damier $(n\times n)$.
Il est composé de $(n^2 - 1)$ petits carreaux numérotés de 1 à $(n^2 - 1)$ qui glissent dans un cadre prévu pour $n^2$ carreaux. Il consiste à remettre dans l'ordre les $(n^2 - 1)$ carreaux à partir d'une configuration initiale quelconque.
L'état du taquin $(n\times n)$ sera représenté par une liste de $n$ listes de $n$ éléments
chacune, chaque élément portant un numéro de $1$ à $(n-1)$, le carreau vide étant numéroté $0$.
La figure ci-dessous donne un exemple de taquin $(3\times 3)$.

\noindent
\centerline{\begin{minipage}{5.575cm}
\setlength{\unitlength}{0.6cm}
\begin{picture}(9,5)
\put(0,4){\makebox(3,1){initial}}
\put(0,1){\framebox(3,3){}} 
\put(0.1,1.1){\framebox(0.8,0.8){7}}
\put(1.1,1.1){\framebox(0.8,0.8){5}}
\put(2.1,2.1){\framebox(0.8,0.8){3}}
\put(2.1,1.1){\framebox(0.8,0.8){4}}
\put(2.1,3.1){\framebox(0.8,0.8){8}}
\put(1.1,3.1){\framebox(0.8,0.8){1}}
\put(0.1,3.1){\framebox(0.8,0.8){2}}
\put(0.1,2.1){\framebox(0.8,0.8){6}}
\put(6,4){\makebox(3,1){final}}
\put(6,1){\framebox(3,3){}} 
\put(6.1,1.1){\framebox(0.8,0.8){7}}
\put(7.1,1.1){\framebox(0.8,0.8){8}}
\put(6.1,2.1){\framebox(0.8,0.8){4}}
\put(7.1,2.1){\framebox(0.8,0.8){5}}
\put(8.1,2.1){\framebox(0.8,0.8){6}}
\put(6.1,3.1){\framebox(0.8,0.8){1}}
\put(7.1,3.1){\framebox(0.8,0.8){2}}
\put(8.1,3.1){\framebox(0.8,0.8){3}}
\put(3.5,2.5){\color{orange}\vector(1,0){2}}
\put(4.5,2.5){\makebox(0,1){\color{orange}\bf ?}}
\put(4.5,0.15){\makebox(0,0){taquin $(3\times 3)$}}
\end{picture}
\end{minipage}
\hfill
\begin{minipage}{8.5cm}
Le carreau vide est numéroté \texttt{0} :
$$\begin{tabular}{lll}
état initial 	&:& \texttt{[[2,1,8],[6,0,3],[7,5,4]]}\\ 
\color{orange}$\downarrow$ ?   & & \\
état final 		&:& \texttt{[[1,2,3],[4,5,6],[7,8,0]]}
\end{tabular}$$
\end{minipage}}

\paragraph{Questions :}\mbox{}
\begin{enumerate}
\item Définir la fonction \texttt{listeCarree} qui teste si une liste \texttt{t} est une
	liste de $n$ éléments dont chaque élément est lui-même une liste de $n$ éléments.

\noindent
\begin{minipage}[t]{7cm}\footnotesize
\begin{Verbatim}
>>> listeCarree([])
True
>>> listeCarree([[1]])
True
>>> listeCarree([[1],[2]])
False
>>> listeCarree([[1,7],[2,-3]])
True
\end{Verbatim}
\end{minipage}
\hfill
\begin{minipage}[t]{7cm}\footnotesize
\begin{Verbatim}
>>> listeCarree([[2,1,8],[6,0,3]])
False
>>> listeCarree([[2,1,8],[6,0,3],[7,5,4]])
True
>>> listeCarree([[2,1,8],[6,0,3],[7]])
False
\end{Verbatim}
\end{minipage}
\vspace*{2mm}

\item Définir la fonction \texttt{appartient} qui teste si un élément \texttt{e}
	appartient (\texttt{True}) ou non (\texttt{False}) à une liste carrée \texttt{t}.

\noindent
\begin{minipage}[t]{7cm}\footnotesize
\begin{Verbatim}
>>> t = [[2,1,8],[6,0,3],[7,5,4]]
>>> 5 in t
False
>>> 9 in t
False
\end{Verbatim}
\end{minipage}
\hfill
\begin{minipage}[t]{7cm}\footnotesize
\begin{Verbatim}
>>> t = [[2,1,8],[6,0,3],[7,5,4]]
>>> appartient(5,t)
True
>>> appartient(9,t)
False
\end{Verbatim}
\end{minipage}
\vspace*{2mm}

	
\item Définir la fonction \texttt{Taquin} qui teste si une liste \texttt{t} représente
	(\texttt{True}) ou non (\texttt{False}) un jeu de taquin.
	
\noindent
\begin{minipage}[t]{7cm}\footnotesize
\begin{Verbatim}
>>> Taquin([[0,1,2]])
False
>>> Taquin([[3,1],[0,2]])
True
\end{Verbatim}
\end{minipage}
\hfill
\begin{minipage}[t]{7cm}\footnotesize
\begin{Verbatim}
>>> Taquin([[2,1,8],[6,0,3],[7,5,4]])
True
>>> Taquin([[2,1,8],[6,0,3],[7,5,0]])
False
\end{Verbatim}
\end{minipage}
\vspace*{2mm}

\item Définir la fonction \texttt{jouerTaquin} qui, à partir d'une configuration 
	donnée \texttt{jeu} d'un damier $(n\times n)$, retourne la liste des 
	configurations possibles après un déplacement élémentaire du carreau vide.
	
\noindent\begin{minipage}[t]{7cm}\footnotesize
\begin{Verbatim}
>>> jeu  = [[3,1],[0,2]]
>>> jouerTaquin(jeu)
[[[0, 1], [3, 2]] , 
 [[3, 1], [2, 0]]]
\end{Verbatim}
\end{minipage}
 \hfill
\begin{minipage}[t]{7cm}\footnotesize
\begin{Verbatim}
>>> jeu  = [[2,1,8],[6,0,3],[7,5,4]]
>>> jouerTaquin(jeu)
[[[2, 0, 8], [6, 1, 3], [7, 5, 4]], 
 [[2, 1, 8], [6, 5, 3], [7, 0, 4]], 
 [[2, 1, 8], [0, 6, 3], [7, 5, 4]], 
 [[2, 1, 8], [6, 3, 0], [7, 5, 4]]]
\end{Verbatim}
\end{minipage}

\end{enumerate}

\paragraph{Réponses :}\mbox{}\\
\noindent\framebox[0.99\textwidth]{$\rule{0cm}{19cm}$}

\noindent\framebox[0.99\textwidth]{$\rule{0cm}{22cm}$}

\noindent\framebox[0.99\textwidth]{$\rule{0cm}{22cm}$}


%-------------------------------------------------------------------------
\end{document}
%-------------------------------------------------------------------------
