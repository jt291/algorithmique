%-------------------------------------------------------------------------
% ds-info-S1-instructions.tex
%-------------------------------------------------------------------------

%-------------------------------------------------------------------------
\documentclass[11pt,a4paper,colorlinks,breaklinks]{article}
%-------------------------------------------------------------------------

%-------------------------------------------------------------------------
%-------------------------------------------------------------------------
% ds-info-S1-preambule.tex
%-------------------------------------------------------------------------

%-------------------------------------------------------------------------
\usepackage{calc}
\usepackage[text={16cm,23cm},centering=true,showframe=false]{geometry}
\usepackage{fancybox,fancyvrb,fancyhdr,lastpage,lineno,import}
\usepackage{longtable,multirow}
\usepackage{xcolor,graphics,xmpmulti,pgf,pgfpages,tikz,wrapfig}
\usepackage{colortbl,color}
\usepackage{amsmath,amssymb,amsfonts}
\usepackage{hyperref,multimedia,rotating,framed,pstricks}
\usepackage{listings,index}
%
%---- pdflatex
%\usepackage[T1]{fontenc}
%\usepackage[utf8]{inputenc}
%---- xelatex
\usepackage{fontspec}
%
\usepackage[french]{minitoc}
\usepackage[french]{babel}
\usepackage[french]{nomencl}
\usepackage[framed,hyperref,standard]{ntheorem}
\usepackage{eurosym,pifont}
%-------------------------------------------------------------------------

%-------------------------------------------------------------------------
\lstset
{
language=Python,
basicstyle=\ttfamily,
identifierstyle=\ttfamily,
keywordstyle=\color{blue}\ttfamily,
commentstyle=\color{gray}\ttfamily,
stringstyle=\color{green}\ttfamily,
showstringspaces=false,
extendedchars=true,
numbers=left, 
numberstyle=\color{blue}\tiny,
frame=lines,
linewidth=0.95\textwidth,
xleftmargin=5mm
} 
%-------------------------------------------------------------------------

%-------------------------------------------------------------------------
\pgfdeclareimage[width=3cm,interpolate=true]{logo-enib}{logo-enib}
%-------------------------------------------------------------------------

%-------------------------------------------------------------------------
\pagestyle{fancy}
\fancyhead{}
\fancyhead[L]{\hspace*{-3em}\begin{minipage}{3cm}\pgfuseimage{logo-enib}\end{minipage}}
\fancyhead[C]{Informatique S1}
\fancyhead[R]{\thepage/\pageref{LastPage}}
\fancyfoot{}
\fancyfoot[L]{}
\fancyfoot[C]{}
\fancyfoot[R]{}
\setlength{\headheight}{80pt}
\setlength{\footskip}{38pt}
\renewcommand{\headrulewidth}{0pt}
\renewcommand{\footrulewidth}{0pt}
%-------------------------------------------------------------------------


%-------------------------------------------------------------------------
\def\entete{\noindent\begin{tabular}{|l|l|l|l|} 
\hline 
 & & & \\ 
\makebox[4.55cm][l]{\bsc{Nom :}} & \makebox[4.5cm][l]{\bsc{Prénom :}} & \makebox[2.5cm][l]{\bsc{Groupe :}} & \makebox[2.75cm][l]{\bsc{Question :}} \\[1mm] 
\hline 
\end{tabular}\\[1mm]
{\footnotesize \textsc{Durée : 15'\hfill Documents, calculettes, téléphones et ordinateurs interdits}}}

\def\notes{\begin{tabular}{|c|c|c|c|}  
\hline 
\makebox[0.5cm]{3} & \makebox[0.5cm]{2} & \makebox[0.5cm]{1} & \makebox[0.5cm]{0} \\  
\hline
\end{tabular}  
} 

\def\autoevaluation{$$\begin{tabular}{|c|c|c|}
\hline
\multicolumn{3}{|c|}{\textbf{Auto-évaluation}} \\
\hline
\textbf{M} & \textbf{V} & \textbf{R} \\
Méthode(s) & Vérification(s) & Résultat(s) \\
\notes & \notes & \notes \\[1mm]
\hline
\end{tabular}$$ $$ $$}

\def\reponse{\mbox{}\hfill \fbox{\huge Réponse page suivante}}
%-------------------------------------------------------------------------

%-------------------------------------------------------------------------
\tikzset{
xmin/.store in=\xmin, xmin/.default=-3, xmin=-3,
xmax/.store in=\xmax, xmax/.default=3,  xmax=3,
ymin/.store in=\ymin, ymin/.default=-3, ymin=-3,
ymax/.store in=\ymax, ymax/.default=3,  ymax=3,
}

\newcommand{\grille}{\draw[color=lightgray] (\xmin,\ymin) grid (\xmax,\ymax);}

\newcommand{\axes}{
	\draw[->] (\xmin,0) -- (\xmax,0);
	\draw[->] (0,\ymin) -- (0,\ymax);
}

\newcommand{\fenetre}{\clip (\xmin,\ymin) rectangle (\xmax,\ymax);}
%-------------------------------------------------------------------------

%-------------------------------------------------------------------------
\def\ga{\textsc{ga}}   
\def\bu{\textsc{bu}} 
\def\zo{\textsc{zo}} 
\def\meu{\textsc{meu}} 
%-------------------------------------------------------------------------


%-------------------------------------------------------------------------
\input{sigle}
%-------------------------------------------------------------------------

\graphicspath{{../../fig/}}



%-------------------------------------------------------------------------

%-------------------------------------------------------------------------
\begin{document}
%-------------------------------------------------------------------------

%%-------------------------------------------------------------------------
%\newpage
%\entete{10}
%\section{Affectation}
%%-------------------------------------------------------------------------
%
%\paragraph{Enoncé :} 
%Un libraire propose une réduction de 3.5\% sur le prix hors taxes (HT) 
%d'un livre à 12.35 \euro{} HT.
%Sachant que la taxe sur la valeur ajoutée (TVA) sur les livres est de 5.5\%, 
%proposer une instruction de type «~affectation~» qui permettra de calculer 
%le prix final toutes taxes comprises (TTC) pour le client.
%	
%\paragraph{Réponse :}\mbox{}\\
%\noindent\framebox[0.99\textwidth][l]{\textbf{Méthode :} $\rule[-16cm]{0cm}{16cm}$}
%
%\noindent\framebox[0.99\textwidth][l]{\textbf{Résultat :} $\rule[-6cm]{0cm}{6cm}$}
%\vspace*{5mm}
%
%\noindent\framebox[0.99\textwidth][l]{\textbf{Vérification :} $\rule[-15cm]{0cm}{15cm}$}

%-------------------------------------------------------------------------
\newpage
\entete{15}
\section{Tests}
%-------------------------------------------------------------------------

\paragraph{Enoncé :} 
Une grande surface propose un service de photocopies aux conditions suivantes :
les 100 premières pages sont facturées 0.1 \euro{} la page,
les 500 suivantes sont facturées 0.08 \euro{} la page
et au-delà, la page est facturée à 0.05 \euro{}.
Proposer une instruction de type «~alternative multiple~» 
qui permettra de calculer le prix total des photocopies pour un document 
de $n$ pages.

\paragraph{}\mbox{}\\
\noindent\framebox[0.99\textwidth][l]{\textbf{Méthode :} $\rule[-16cm]{0cm}{16cm}$}

\noindent\framebox[0.99\textwidth][l]{\textbf{Résultat :} $\rule[-6cm]{0cm}{6cm}$}
\vspace*{5mm}

\noindent\framebox[0.99\textwidth][l]{\textbf{Vérification :} $\rule[-15cm]{0cm}{15cm}$}

%-------------------------------------------------------------------------
\newpage
\entete{25}
\section{Boucles}
%-------------------------------------------------------------------------

\paragraph{Enoncé :}
Proposer une instruction de type «~boucle~» qui permettra de calculer l'intégrale $I = \int cos(x) dx$ 
sur l'intervalle $[-\frac{\pi}{4},\frac{\pi}{4}]$ par la méthode des $n$ rectangles.
$$I = \int_{-\frac{\pi}{4}}^{+\frac{\pi}{4}} cos(x) dx \approx
\frac{\pi}{2n}\cdot\sum_{k=0}^{n-1} \cos(-\frac{\pi}{4}+\frac{k\pi}{2n})$$
\noindent\begin{tikzpicture}[scale=1]\footnotesize\shorthandoff{:}
\draw (0,1.1) node[left] {$1$};
\draw (0,-1.1) node[left] {$-1$};
\draw[color=lightgray] (-3.25,1) -- (3.25,1) ;
\draw[color=lightgray] (-3.25,-1) -- (3.25,-1) ;
\draw ({-pi},-0.05) -- ({-pi},0.05);
\draw ({-pi},0) node[below] {$-\pi$};
\draw ({-3*pi/4},-0.05) -- ({-3*pi/4},0.05);
\draw ({-3*pi/4},0) node[below] {$-\frac{3\pi}{4}$};
\draw ({-pi/2},-0.05) -- ({-pi/2},0.05);
\draw ({-pi/2},0) node[below] {$-\frac{\pi}{2}$};
\draw ({-pi/4},-0.05) -- ({-pi/4},0.05);
\draw ({-pi/4},0) node[below] {$-\frac{\pi}{4}$};
\draw ({pi/4},-0.05) -- ({pi/4},0.05);
\draw ({pi/4},0) node[below] {$\frac{\pi}{4}$};
\draw ({pi/2},-0.05) -- ({pi/2},0.05);
\draw ({pi/2},0) node[below] {$\frac{\pi}{2}$};
\draw ({3*pi/4},-0.05) -- ({3*pi/4},0.05);
\draw ({3*pi/4},0) node[below] {$\frac{3\pi}{4}$};
\draw ({pi},-0.05) -- ({pi},0.05);
\draw ({pi},0) node[below] {$\pi$};
\draw[dashed,color=lightgray] ({-pi/4},-1) -- ({-pi/4},1);
\draw[dashed,color=lightgray] ({pi/4},-1) -- ({pi/4},1);
\fill[color=gray!20] ({-pi/4},0) -- ({-pi/4},{cos((-pi/4) r)}) -- plot [domain=(-pi/4):(pi/4)] (\x,{cos(\x r)}) -- ({cos((pi/4) r)},0) -- cycle;
\draw[color=blue,domain=-3.5:3.5] plot (\x,{cos(\x r)});
\draw[dashed,color=lightgray] ({-pi/4},-1) -- ({-pi/4},1);
\draw[dashed,color=lightgray] ({pi/4},-1) -- ({pi/4},1);
\draw[->] (0,-1.25) -- (0,1.25) ;
\draw (0,1.25) node[above] {$f(x)$};
\draw[->] (-3.5,0) -- (3.5,0) ;
\draw (3.5,0) node[right] {$x$};
\draw[fill] (0,0) circle (0.05);
\end{tikzpicture}
\hfill
\begin{tikzpicture}[xscale=1.8,yscale=1.4]\footnotesize\shorthandoff{:}
\fill[color=gray!20] ({-pi/4},0) -- ({-pi/4},{cos((-pi/4) r)}) -- plot [domain=(-pi/4):(pi/4)] (\x,{cos(\x r)}) -- ({cos((pi/4) r)},0) -- cycle;
\foreach \x in {0,1,...,4} \draw ({-pi/4 + \x*pi/20},0) -- ({-pi/4 + \x*pi/20},{cos((-pi/4 + (\x+0.5)*pi/20) r)});
\foreach \x in {0,1,...,4} \draw ({-pi/4 + \x*pi/20},{cos((-pi/4 + (\x+0.5)*pi/20) r)}) -- ({-pi/4 + (\x+1)*pi/20},{cos((-pi/4 + (\x+0.5)*pi/20) r)});
\foreach \x in {0,1,...,4} \draw ({pi/4 - \x*pi/20},0) -- ({pi/4 - \x*pi/20},{cos((pi/4 - (\x+0.5)*pi/20) r)});
\foreach \x in {0,1,...,4} \draw ({pi/4 - \x*pi/20},{cos((pi/4 - (\x+0.5)*pi/20) r)}) -- ({pi/4 - (\x+1)*pi/20},{cos((pi/4 - (\x+0.5)*pi/20) r)});
\draw[color=blue,domain=-1.25:1.25] plot (\x,{cos(\x r)});
\draw[->] (0,-0.25) -- (0,1.25) ;
\draw[->] (-1.75,0) -- (1.75,0) ;
\draw (1.75,0) node[right] {$x$};
\draw (0,1.25) node[above] {$f(x)$};
\draw ({-pi/4},0) node[below] {$-\frac{\pi}{4}$};
\draw ({pi/4},0) node[below] {$\frac{\pi}{4}$};
\draw ({-pi/4 + 2.5*pi/20},0) node[below] {$x_i$};
\draw ({-pi/4 + 2.5*pi/20},-0.05) -- ({-pi/4 + 2.5*pi/20},0.05);
\draw ({2.5*pi/20},-0.5) node[below] {$\Delta x$};
\draw[dashed] ({2*pi/20},0) -- ({2*pi/20},-0.5);
\draw[dashed] ({3*pi/20},0) -- ({3*pi/20},-0.5);
\draw[->] ({1.5*pi/20},-0.4) -- ({2*pi/20},-0.4);
\draw[->] ({3.5*pi/20},-0.4) -- ({3*pi/20},-0.4);
\draw ({3*pi/20},-0.4) -- ({2*pi/20},-0.4);
\draw[fill] (0,0) circle(0.03);
\draw[fill] ({-pi/4},0) circle(0.03);
\draw[fill] ({pi/4},0) circle(0.03);
\draw ({2.5*pi/20},-1) node {$\displaystyle \Delta x = \frac{\left(\frac{\pi}{4} - \frac{-\pi}{4}\right)}{n} = \frac{\pi}{2n}$};
\end{tikzpicture}

\paragraph{}\mbox{}\\
\noindent\framebox[0.99\textwidth][l]{\textbf{Méthode :} $\rule[-11cm]{0cm}{11cm}$}

\noindent\framebox[0.99\textwidth][l]{\textbf{Résultat :} $\rule[-6cm]{0cm}{6cm}$}
\vspace*{5mm}

\noindent\framebox[0.99\textwidth][l]{\textbf{Vérification :} $\rule[-15cm]{0cm}{15cm}$}


%-------------------------------------------------------------------------
\newpage
\entete{25}
\section{Boucles et tests}
%-------------------------------------------------------------------------

\paragraph{Enoncé :}
Ecrire un algorithme qui permettra de déterminer le zéro de $\cos(x)$ dans $[1,2]$ selon une méthode
par dichotomie.

\begin{minipage}{6cm}
\begin{tikzpicture}[xscale=1.4,yscale=1.4]\footnotesize\shorthandoff{:}
\draw[color=blue,domain=0.25:2.75] plot (\x,{cos(\x r)});
\draw[dashed,color=lightgray] (1,-1) -- (1,1);
\draw[dashed,color=lightgray] (2,-1) -- (2,1);
\draw[dashed,color=lightgray] (0,-1) -- (3.25,-1);
\draw[dashed,color=lightgray] (0,1) -- (3.25,1);
\draw (0,1) node[left] {1};
\draw (0,-1) node[left] {-1};
\draw[fill] (0,0) circle(0.03);
\draw[fill] (1,0) circle(0.03);
\draw[fill] (2,0) circle(0.03);
\draw[fill] (3,0) circle(0.03);
\draw (0,-0.15) node[left] {0};
\draw (1,-0.15) node[left] {1};
\draw (2,-0.15) node[left] {2};
\draw (3,-0.15) node[left] {3};
\draw[->] (0,-1.25) -- (0,1.25) ;
\draw[->] (-0.25,0) -- (3.25,0) ;
\draw (3.25,0) node[right] {$x$};
\draw (0,1.25) node[above] {$f(x)$};
%\draw[color=blue] (3.14/2,0) circle(0.05);
\draw[->,color=gray] (1.95,0.5) -- (3.14/2,0);
\draw[color=gray] (1.95,0.5) node[right] {zéro de $\cos(x)$};
\end{tikzpicture}
\end{minipage}
\hfill
\begin{minipage}{8cm}
Indications : Soient $[x_1,x_2]$ l'intervalle de recherche et $x_m = (x_1+x_2)/2$ le point milieu 
de cet intervalle. Si $f(x_1)\cdot f(x_m) < 0$, le zéro recherché est dans $[x_1,x_m]$, sinon le
zéro est dans $[x_m,x_2]$. On réitère le procédé sur le nouvel intervalle de recherche jusqu'à ce que
la longueur de l'intervalle soit suffisamment petite. Le milieu de ce dernier intervalle sera le
zéro recherché.
\end{minipage}

\paragraph{}\mbox{}\\
\noindent\framebox[0.99\textwidth][l]{\textbf{Méthode :} $\rule[-12cm]{0cm}{12cm}$}

\noindent\framebox[0.99\textwidth][l]{\textbf{Résultat :} $\rule[-6cm]{0cm}{6cm}$}
\vspace*{5mm}

\noindent\framebox[0.99\textwidth][l]{\textbf{Vérification :} $\rule[-15cm]{0cm}{15cm}$}


%-------------------------------------------------------------------------
\newpage
\entete{15}
\section{Exécution d'une séquence d'instructions}
%-------------------------------------------------------------------------

\paragraph{Enoncé :}
On considère la séquence d'instructions suivantes :

\begin{lstlisting}
n = 0
while n <= k:
    j = 0
    while j <= n:
        num, den = 1, 1
        i = 1
        while i <= j:
            num = num * (n - i + 1)
            den = den * i
            i = i + 1
        c = num//den     # division entière
        print(c,end=' ')
        j = j + 1
    print()
    n = n + 1
\end{lstlisting}
\vspace*{3mm}

\noindent Qu'affiche cette séquence pour \texttt{k = 6} ?

\paragraph{Réponse :} On n'attend pas ici une réponse concernant la méthode (M).
En guise de vérification (V), on pourra proposer un nom à cet algorithme.

\noindent Méthode : Se mettre à la place de la
machine pour exécuter scrupuleusement les instructions en inscrivant dans
le tableau ci-dessous chaque affichage de la fonction \texttt{print}.

$$\begin{tabular}{|l|}
\multicolumn{1}{l}{\textbf{Résultat :}}\\
\hline
\makebox[10cm]{}\\[3mm]
\hline
\mbox{}\\[3mm]
\hline
\mbox{}\\[3mm]
\hline
\mbox{}\\[3mm]
\hline
\mbox{}\\[3mm]
\hline
\mbox{}\\[3mm]
\hline
\mbox{}\\[3mm]
\hline
\mbox{}\\[3mm]
\hline
\mbox{}\\[3mm]
\hline
\end{tabular}$$

$$\begin{tabular}{|l|}
\multicolumn{1}{l}{\textbf{Vérification :}}\\
\hline
\makebox[10cm][l]{\bf Nom de l'algorithme :}\\[3mm]
\mbox{}\\[12mm]
\hline
\end{tabular}$$

%-------------------------------------------------------------------------
\end{document}
%-------------------------------------------------------------------------
