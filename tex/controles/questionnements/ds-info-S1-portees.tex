%-------------------------------------------------------------------------
% ds-info-S1-portees.tex
%-------------------------------------------------------------------------

%-------------------------------------------------------------------------
\documentclass[11pt,a4paper]{article}
%-------------------------------------------------------------------------

%-------------------------------------------------------------------------
\input{ds-info-S1-preambule.tex}
%-------------------------------------------------------------------------

%-------------------------------------------------------------------------
\begin{document}
%-------------------------------------------------------------------------
\entete

%\autoevaluation


$$\mbox{\textbf{\large Appels de fonctions}}$$

%-------------------------------------------------------------------------
\paragraph{Contexte :}
%-------------------------------------------------------------------------
On considère les 4 fonctions \texttt{f}, \texttt{g}, \texttt{h} et \texttt{i} suivantes :\vspace*{2mm}

\noindent\begin{minipage}{3.5cm}\footnotesize
\begin{Verbatim}
def f(x) :
    x = i(x)
    print('f :', x)
    return x
\end{Verbatim}
\end{minipage}
\hfill
\begin{minipage}{3.5cm}\footnotesize
\begin{Verbatim}
def g(x) :
    x = x/2
    print('g :', x)
    return x
\end{Verbatim}
\end{minipage}
\hfill
\begin{minipage}{3.5cm}\footnotesize
\begin{Verbatim}
def h(x) :
    x = g(x)
    print('h :', x)
    return x
\end{Verbatim}
\end{minipage}
\hfill
\begin{minipage}{3.5cm}\footnotesize
\begin{Verbatim}
def i(x) :
    x = 2*x
    print('i :', x)
    return x
\end{Verbatim}
\end{minipage}

%-------------------------------------------------------------------------
\paragraph{Questions :}
%-------------------------------------------------------------------------
Qu'affichent les appels suivants pour une valeur de \texttt{x} donnée ? 
Indiquer la nouvelle valeur de \texttt{x} après chaque appel.
\vspace*{3mm}

\noindent\begin{minipage}{7cm}
\begin{enumerate}
\item \texttt{h(f(i(g(x))))} avec \texttt{x = -3/2}
\item \texttt{f(h(i(g(x))))} avec \texttt{x = -2} 
\item \texttt{g(h(i(f(x))))} avec \texttt{x = -1/2}
\item \texttt{f(g(i(h(x))))} avec \texttt{x = 1/2}
\item \texttt{f(h(g(i(x))))} avec \texttt{x = 1/2}
\item \texttt{i(f(g(h(x))))} avec \texttt{x = -3/2}
\item \texttt{i(h(f(g(x))))} avec \texttt{x = 3/2}
\item \texttt{i(f(h(g(x))))} avec \texttt{x = 1}
\item \texttt{g(f(i(h(x))))} avec \texttt{x = 1}
\item \texttt{h(i(f(g(x))))} avec \texttt{x = -1}
\item \texttt{h(g(i(f(x))))} avec \texttt{x = 2}
\item \texttt{i(g(h(f(x))))} avec \texttt{x = -1/2}
\end{enumerate}
\end{minipage}
\hfill
\begin{minipage}{7cm}
\begin{enumerate}\setcounter{enumi}{12}
\item \texttt{h(f(g(i(x))))} avec \texttt{x = 1}
\item \texttt{g(f(h(i(x))))} avec \texttt{x = -3/2}
\item \texttt{g(i(h(f(x))))} avec \texttt{x = 2}
\item \texttt{g(i(f(h(x))))} avec \texttt{x = 3/2}
\item \texttt{h(g(f(i(x))))} avec \texttt{x = 3/2}
\item \texttt{h(i(g(f(x))))} avec \texttt{x = -1/2}
\item \texttt{f(i(h(g(x))))} avec \texttt{x = 1/2}
\item \texttt{g(h(f(i(x))))} avec \texttt{x = -1}
\item \texttt{i(g(f(h(x))))} avec \texttt{x = -1}
\item \texttt{i(h(g(f(x))))} avec \texttt{x = 2}
\item \texttt{f(g(h(i(x))))} avec \texttt{x = -2} 
\item \texttt{f(i(g(h(x))))} avec \texttt{x = -2} 
\end{enumerate}
\end{minipage}

%-------------------------------------------------------------------------
%\newpage
\paragraph{Réponse :} \mbox{}
%-------------------------------------------------------------------------

\noindent\framebox[\textwidth]{$\rule{0cm}{7cm}$}
%-------------------------------------------------------------------------
\end{document}
%-------------------------------------------------------------------------
