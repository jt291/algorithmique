%-------------------------------------------------------------------------
% ds-info-S1-portees.tex
%-------------------------------------------------------------------------

%-------------------------------------------------------------------------
\documentclass[11pt,a4paper]{article}
%-------------------------------------------------------------------------

%-------------------------------------------------------------------------
%-------------------------------------------------------------------------
% ds-info-S1-preambule.tex
%-------------------------------------------------------------------------

%-------------------------------------------------------------------------
\usepackage{calc}
\usepackage[text={16cm,23cm},centering=true,showframe=false]{geometry}
\usepackage{fancybox,fancyvrb,fancyhdr,lastpage,lineno,import}
\usepackage{longtable,multirow}
\usepackage{xcolor,graphics,xmpmulti,pgf,pgfpages,tikz,wrapfig}
\usepackage{colortbl,color}
\usepackage{amsmath,amssymb,amsfonts}
\usepackage{hyperref,multimedia,rotating,framed,pstricks}
\usepackage{listings,index}
%
%---- pdflatex
%\usepackage[T1]{fontenc}
%\usepackage[utf8]{inputenc}
%---- xelatex
\usepackage{fontspec}
%
\usepackage[french]{minitoc}
\usepackage[french]{babel}
\usepackage[french]{nomencl}
\usepackage[framed,hyperref,standard]{ntheorem}
\usepackage{eurosym,pifont}
%-------------------------------------------------------------------------

%-------------------------------------------------------------------------
\lstset
{
language=Python,
basicstyle=\ttfamily,
identifierstyle=\ttfamily,
keywordstyle=\color{blue}\ttfamily,
commentstyle=\color{gray}\ttfamily,
stringstyle=\color{green}\ttfamily,
showstringspaces=false,
extendedchars=true,
numbers=left, 
numberstyle=\color{blue}\tiny,
frame=lines,
linewidth=0.95\textwidth,
xleftmargin=5mm
} 
%-------------------------------------------------------------------------

%-------------------------------------------------------------------------
\pgfdeclareimage[width=3cm,interpolate=true]{logo-enib}{logo-enib}
%-------------------------------------------------------------------------

%-------------------------------------------------------------------------
\pagestyle{fancy}
\fancyhead{}
\fancyhead[L]{\hspace*{-3em}\begin{minipage}{3cm}\pgfuseimage{logo-enib}\end{minipage}}
\fancyhead[C]{Informatique S1}
\fancyhead[R]{\thepage/\pageref{LastPage}}
\fancyfoot{}
\fancyfoot[L]{}
\fancyfoot[C]{}
\fancyfoot[R]{}
\setlength{\headheight}{80pt}
\setlength{\footskip}{38pt}
\renewcommand{\headrulewidth}{0pt}
\renewcommand{\footrulewidth}{0pt}
%-------------------------------------------------------------------------


%-------------------------------------------------------------------------
\def\entete{\noindent\begin{tabular}{|l|l|l|l|} 
\hline 
 & & & \\ 
\makebox[4.55cm][l]{\bsc{Nom :}} & \makebox[4.5cm][l]{\bsc{Prénom :}} & \makebox[2.5cm][l]{\bsc{Groupe :}} & \makebox[2.75cm][l]{\bsc{Question :}} \\[1mm] 
\hline 
\end{tabular}\\[1mm]
{\footnotesize \textsc{Durée : 15'\hfill Documents, calculettes, téléphones et ordinateurs interdits}}}

\def\notes{\begin{tabular}{|c|c|c|c|}  
\hline 
\makebox[0.5cm]{3} & \makebox[0.5cm]{2} & \makebox[0.5cm]{1} & \makebox[0.5cm]{0} \\  
\hline
\end{tabular}  
} 

\def\autoevaluation{$$\begin{tabular}{|c|c|c|}
\hline
\multicolumn{3}{|c|}{\textbf{Auto-évaluation}} \\
\hline
\textbf{M} & \textbf{V} & \textbf{R} \\
Méthode(s) & Vérification(s) & Résultat(s) \\
\notes & \notes & \notes \\[1mm]
\hline
\end{tabular}$$ $$ $$}

\def\reponse{\mbox{}\hfill \fbox{\huge Réponse page suivante}}
%-------------------------------------------------------------------------

%-------------------------------------------------------------------------
\tikzset{
xmin/.store in=\xmin, xmin/.default=-3, xmin=-3,
xmax/.store in=\xmax, xmax/.default=3,  xmax=3,
ymin/.store in=\ymin, ymin/.default=-3, ymin=-3,
ymax/.store in=\ymax, ymax/.default=3,  ymax=3,
}

\newcommand{\grille}{\draw[color=lightgray] (\xmin,\ymin) grid (\xmax,\ymax);}

\newcommand{\axes}{
	\draw[->] (\xmin,0) -- (\xmax,0);
	\draw[->] (0,\ymin) -- (0,\ymax);
}

\newcommand{\fenetre}{\clip (\xmin,\ymin) rectangle (\xmax,\ymax);}
%-------------------------------------------------------------------------

%-------------------------------------------------------------------------
\def\ga{\textsc{ga}}   
\def\bu{\textsc{bu}} 
\def\zo{\textsc{zo}} 
\def\meu{\textsc{meu}} 
%-------------------------------------------------------------------------


%-------------------------------------------------------------------------
\input{sigle}
%-------------------------------------------------------------------------

\graphicspath{{../../fig/}}



%-------------------------------------------------------------------------

%-------------------------------------------------------------------------
\begin{document}
%-------------------------------------------------------------------------
\entete

%\autoevaluation


$$\mbox{\textbf{\large Appels de fonctions}}$$

%-------------------------------------------------------------------------
\paragraph{Contexte :}
%-------------------------------------------------------------------------
On considère les 4 fonctions \texttt{f}, \texttt{g}, \texttt{h} et \texttt{i} suivantes :\vspace*{2mm}

\noindent\begin{minipage}{3.5cm}\footnotesize
\begin{Verbatim}
def f(x) :
    x = i(x)
    print('f :', x)
    return x
\end{Verbatim}
\end{minipage}
\hfill
\begin{minipage}{3.5cm}\footnotesize
\begin{Verbatim}
def g(x) :
    x = x/2
    print('g :', x)
    return x
\end{Verbatim}
\end{minipage}
\hfill
\begin{minipage}{3.5cm}\footnotesize
\begin{Verbatim}
def h(x) :
    x = g(x)
    print('h :', x)
    return x
\end{Verbatim}
\end{minipage}
\hfill
\begin{minipage}{3.5cm}\footnotesize
\begin{Verbatim}
def i(x) :
    x = 2*x
    print('i :', x)
    return x
\end{Verbatim}
\end{minipage}

%-------------------------------------------------------------------------
\paragraph{Questions :}
%-------------------------------------------------------------------------
Qu'affichent les appels suivants pour une valeur de \texttt{x} donnée ? 
Indiquer la nouvelle valeur de \texttt{x} après chaque appel.
\vspace*{3mm}

\noindent\begin{minipage}{7cm}
\begin{enumerate}
\item \texttt{h(f(i(g(x))))} avec \texttt{x = -3/2}
\item \texttt{f(h(i(g(x))))} avec \texttt{x = -2} 
\item \texttt{g(h(i(f(x))))} avec \texttt{x = -1/2}
\item \texttt{f(g(i(h(x))))} avec \texttt{x = 1/2}
\item \texttt{f(h(g(i(x))))} avec \texttt{x = 1/2}
\item \texttt{i(f(g(h(x))))} avec \texttt{x = -3/2}
\item \texttt{i(h(f(g(x))))} avec \texttt{x = 3/2}
\item \texttt{i(f(h(g(x))))} avec \texttt{x = 1}
\item \texttt{g(f(i(h(x))))} avec \texttt{x = 1}
\item \texttt{h(i(f(g(x))))} avec \texttt{x = -1}
\item \texttt{h(g(i(f(x))))} avec \texttt{x = 2}
\item \texttt{i(g(h(f(x))))} avec \texttt{x = -1/2}
\end{enumerate}
\end{minipage}
\hfill
\begin{minipage}{7cm}
\begin{enumerate}\setcounter{enumi}{12}
\item \texttt{h(f(g(i(x))))} avec \texttt{x = 1}
\item \texttt{g(f(h(i(x))))} avec \texttt{x = -3/2}
\item \texttt{g(i(h(f(x))))} avec \texttt{x = 2}
\item \texttt{g(i(f(h(x))))} avec \texttt{x = 3/2}
\item \texttt{h(g(f(i(x))))} avec \texttt{x = 3/2}
\item \texttt{h(i(g(f(x))))} avec \texttt{x = -1/2}
\item \texttt{f(i(h(g(x))))} avec \texttt{x = 1/2}
\item \texttt{g(h(f(i(x))))} avec \texttt{x = -1}
\item \texttt{i(g(f(h(x))))} avec \texttt{x = -1}
\item \texttt{i(h(g(f(x))))} avec \texttt{x = 2}
\item \texttt{f(g(h(i(x))))} avec \texttt{x = -2} 
\item \texttt{f(i(g(h(x))))} avec \texttt{x = -2} 
\end{enumerate}
\end{minipage}

%-------------------------------------------------------------------------
%\newpage
\paragraph{Réponse :} \mbox{}
%-------------------------------------------------------------------------

\noindent\framebox[\textwidth]{$\rule{0cm}{7cm}$}
%-------------------------------------------------------------------------
\end{document}
%-------------------------------------------------------------------------
