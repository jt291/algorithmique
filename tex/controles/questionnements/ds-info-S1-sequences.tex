%-------------------------------------------------------------------------
% ds-info-S1-sequences.tex
%-------------------------------------------------------------------------

%-------------------------------------------------------------------------
\documentclass[11pt,a4paper]{article}
%-------------------------------------------------------------------------

%-------------------------------------------------------------------------
\input{ds-info-S1-preambule.tex}
%-------------------------------------------------------------------------

%-------------------------------------------------------------------------
\begin{document}
%-------------------------------------------------------------------------
%\noindent\framebox[\textwidth]{}
\entete

\autoevaluation


$$\mbox{\textbf{\large Tri d'une séquence}}$$


%-------------------------------------------------------------------------
%\newpage
\paragraph{Questions :}
%-------------------------------------------------------------------------
Un annuaire est représenté ici par une liste de quadruplets 
\texttt{(nom, age, ville, téléphone)}.\\
Exemple d'annuaire :
\begin{minipage}[t]{7cm}
\begin{Verbatim}
item1 = ('dupont', 23, 'brest', '06789656')
item2 = ('abgral', 61, 'lille', '06231298')
item3 = ('dupont', 23, 'brest', '02989656')
item4 = ('abgral', 67, 'brest', '06556438')
item5 = ('martin', 38, 'paris', '01674523')
item6 = ('abgral', 67, 'lille', '06231298')

annuaire = [item1, item2, item3, item4, item5, item6]
\end{Verbatim}
\end{minipage}
\vspace*{3mm}

Trier l'annuaire, par ordre croissant ou décroissant, selon des critères donnés 
par une liste des clés successives. 
L'ordre des critères est précisé par une liste des rangs successifs des champs du quadruplet
\texttt{(nom, age, ville, téléphone)}. Par exemple, la liste \texttt{[3,0,2,1]} indique
qu'il faut d'abord (clé primaire) trier selon les numéros de téléphone (champ \no 3 dans le quadruplet), puis (clé secondaire) selon les noms (champ \no 0 dans le quadruplet), 
puis selon la ville (champ \no 2 dans le quadruplet) et enfin selon les âges 
(champ \no 1 dans le quadruplet).

\noindent
\begin{minipage}[t]{7cm}
\begin{enumerate}
\item \texttt{cles = [3,0,1,2]}, croissant
\item \texttt{cles = [3,0,2,1]}, croissant
\item \texttt{cles = [3,1,0,2]}, croissant
\item \texttt{cles = [3,1,2,0]}, croissant
\item \texttt{cles = [3,2,0,1]}, croissant
\item \texttt{cles = [3,2,1,0]}, croissant
\item \texttt{cles = [1,0,2,3]}, décroissant
\item \texttt{cles = [1,0,3,2]}, décroissant
\item \texttt{cles = [1,2,0,3]}, décroissant
\item \texttt{cles = [1,2,3,0]}, décroissant
\item \texttt{cles = [1,3,0,2]}, décroissant
\item \texttt{cles = [1,3,2,0]}, décroissant
\end{enumerate}
\end{minipage}
\hfill
\begin{minipage}[t]{7cm}
\begin{enumerate}\setcounter{enumi}{12}
\item \texttt{cles = [2,0,1,3]}, croissant
\item \texttt{cles = [2,0,3,1]}, croissant
\item \texttt{cles = [2,1,0,3]}, croissant
\item \texttt{cles = [2,1,3,0]}, croissant
\item \texttt{cles = [2,3,0,1]}, croissant
\item \texttt{cles = [2,3,1,0]}, croissant
\item \texttt{cles = [0,1,2,3]}, décroissant
\item \texttt{cles = [0,1,3,2]}, décroissant
\item \texttt{cles = [0,2,1,3]}, décroissant
\item \texttt{cles = [0,2,3,1]}, décroissant
\item \texttt{cles = [0,3,1,2]}, décroissant
\item \texttt{cles = [0,3,2,1]}, décroissant
\end{enumerate}
\end{minipage}
\paragraph{Réponse :}\mbox{}\\
\framebox[\textwidth]{$\rule{0cm}{22cm}$}

%-------------------------------------------------------------------------
\end{document}
%-------------------------------------------------------------------------

