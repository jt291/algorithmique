% mrv-instructions.tex

%-------------------------------------------------------------------------
\section{Rappels de cours}\label{instructions:cours}
%-------------------------------------------------------------------------

%-------------------------------------------------------------------------
\section{Vie courante}\label{instructions:vie-courante}
%-------------------------------------------------------------------------
\begin{description}
\item[Principal : ] mettre en \oe uvre les instructions de base : affectation, tests, boucles.
\item[Secondaire :] .
\end{description}

\subsection{Objectif}\label{instructions:vie-courante:objectif}

\subsection{Syntaxe \python}\label{instructions:vie-courante:python}

\subsection{Enoncé}\label{instructions:vie-courante:enonce}

\subsection{Méthode}\label{instructions:vie-courante:methode}

\subsection{Résultat}\label{instructions:vie-courante:resultat}

\subsection{Vérification}\label{instructions:vie-courante:verification}

\subsection{Généricité}\label{instructions:vie-courante:genericite}

\subsection{Entraînement}\label{instructions:vie-courante:entrainement}

%-------------------------------------------------------------------------
\section{Jeux : drapeau tricolore}\label{instructions:jeux}
%-------------------------------------------------------------------------

\subsection{Objectif}\label{instructions:jeux:objectif}
\begin{description}
\item[Principal : ] mettre en \oe uvre les instructions de base : affectation, tests, boucles.
\item[Secondaire :] .
\end{description}


\subsection{Syntaxe \python}\label{instructions:jeux:python}

\subsection{Enoncé}\label{instructions:jeux:enonce}

\subsection{Méthode}\label{instructions:jeux:methode}

\subsection{Résultat}\label{instructions:jeux:resultat}

\subsection{Vérification}\label{instructions:jeux:verification}

\subsection{Généricité}\label{instructions:jeux:genericite}

\subsection{Entraînement}\label{instructions:jeux:entrainement}


%-------------------------------------------------------------------------
\section{Textes : recherche d'un motif}\label{instructions:textes}
%-------------------------------------------------------------------------

\subsection{Objectif}\label{instructions:textes:objectif}
\begin{description}
\item[Principal : ] mettre en \oe uvre les instructions de base : affectation, tests, boucles.
\item[Secondaire :] recherche une chaîne de caractères au sein d'une autre chaîne.
\end{description}


\subsection{Syntaxe \python}\label{instructions:textes:python}

\subsection{Enoncé}\label{instructions:textes:enonce}

\subsection{Méthode}\label{instructions:textes:methode}

\subsection{Résultat}\label{instructions:textes:resultat}

\subsection{Vérification}\label{instructions:textes:verification}

\subsection{Généricité}\label{instructions:textes:genericite}

\subsection{Entraînement}\label{instructions:textes:entrainement}

%-------------------------------------------------------------------------
\section{Nombres : crible d'Eratostène}\label{instructions:nombres}
%-------------------------------------------------------------------------

\subsection{Objectif}\label{instructions:nombres:objectif}
\begin{description}
\item[Principal : ] mettre en \oe uvre les instructions de base : affectation, tests, boucles.
\item[Secondaire :] déterminer les $n$ premiers nombres premiers.
\end{description}


\subsection{Syntaxe \python}\label{instructions:nombres:python}

\subsection{Enoncé}\label{instructions:nombres:enonce}

\subsection{Méthode}\label{instructions:nombres:methode}

\subsection{Résultat}\label{instructions:nombres:resultat}

\subsection{Vérification}\label{instructions:nombres:verification}

\subsection{Généricité}\label{instructions:nombres:genericite}

\subsection{Entraînement}\label{instructions:nombres:entrainement}

%-------------------------------------------------------------------------
\section{Figures : }\label{instructions:figures}
%-------------------------------------------------------------------------

\subsection{Objectif}\label{instructions:figures:objectif}
\begin{description}
\item[Principal : ] mettre en \oe uvre les instructions de base : affectation, tests, boucles.
\item[Secondaire :] .
\end{description}


\subsection{Syntaxe \python}\label{instructions:figures:python}

\subsection{Enoncé}\label{instructions:figures:enonce}

\subsection{Méthode}\label{instructions:figures:methode}

\subsection{Résultat}\label{instructions:figures:resultat}

\subsection{Vérification}\label{instructions:figures:verification}

\subsection{Généricité}\label{instructions:figures:genericite}

\subsection{Entraînement}\label{instructions:figures:entrainement}

%-------------------------------------------------------------------------
\section{Mathématiques : zéro d'une fonction}\label{instructions:maths}
%-------------------------------------------------------------------------

\subsection{Objectif}\label{instructions:maths:objectif}
\begin{description}
\item[Principal : ] mettre en \oe uvre les instructions de base : affectation, tests, boucles.
\item[Secondaire :] .
\end{description}


\subsection{Syntaxe \python}\label{instructions:maths:python}

\subsection{Enoncé}\label{instructions:maths:enonce}

\subsection{Méthode}\label{instructions:maths:methode}

\subsection{Résultat}\label{instructions:maths:resultat}

\subsection{Vérification}\label{instructions:maths:verification}

\subsection{Généricité}\label{instructions:maths:genericite}

\subsection{Entraînement}\label{instructions:maths:entrainement}

%-------------------------------------------------------------------------
\section{Physique : }\label{instructions:physique}
%-------------------------------------------------------------------------

\subsection{Objectif}\label{instructions:physique:objectif}
\begin{description}
\item[Principal : ] mettre en \oe uvre les instructions de base : affectation, tests, boucles.
\item[Secondaire :] .
\end{description}


\subsection{Syntaxe \python}\label{instructions:physique:python}

\subsection{Enoncé}\label{instructions:physique:enonce}

\subsection{Méthode}\label{instructions:physique:methode}

\subsection{Résultat}\label{instructions:physique:resultat}

\subsection{Vérification}\label{instructions:physique:verification}

\subsection{Généricité}\label{instructions:physique:genericite}

\subsection{Entraînement}\label{instructions:physique:entrainement}

%-------------------------------------------------------------------------
\section{Informatique : }\label{instructions:informatique}
%-------------------------------------------------------------------------

\subsection{Objectif}\label{instructions:informatique:objectif}
\begin{description}
\item[Principal : ] mettre en \oe uvre les instructions de base : affectation, tests, boucles.
\item[Secondaire :] .
\end{description}


\subsection{Syntaxe \python}\label{instructions:informatique:python}

\subsection{Enoncé}\label{instructions:informatique:enonce}

\subsection{Méthode}\label{instructions:informatique:methode}

\subsection{Résultat}\label{instructions:informatique:resultat}

\subsection{Vérification}\label{instructions:informatique:verification}

\subsection{Généricité}\label{instructions:informatique:genericite}

\subsection{Entraînement}\label{instructions:informatique:entrainement}


%-------------------------------------------------------------------------
\section{Retours d'expériences}\label{instructions:retours}
%-------------------------------------------------------------------------

%-------------------------------------------------------------------------
\subsection{Algorithmes}\label{instructions:retours:algorithmes}

%-------------------------------------------------------------------------
\subsection{\mrv : méthode}\label{instructions:retours:methode}

%-------------------------------------------------------------------------
\subsection{\mrv : résultat}\label{instructions:retours:resultat}

%-------------------------------------------------------------------------
\subsection{\mrv : vérification}\label{instructions:retours:verification}
