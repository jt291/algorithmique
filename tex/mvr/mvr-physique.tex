% mrv-physique.tex

%-------------------------------------------------------------------------
\documentclass[11pt,a4paper,colorlinks,breaklinks]{article}
%-------------------------------------------------------------------------

%-------------------------------------------------------------------------
\RequirePackage{calc}
\RequirePackage[text={16cm,23cm},centering=true,showframe=false]{geometry}
\RequirePackage{fancybox,fancyvrb,fancyhdr,lastpage,lineno}
\RequirePackage{longtable,multirow}
\RequirePackage{xcolor,graphics,xmpmulti,pgf,pgfpages,tikz}
\RequirePackage{colortbl,color}
\RequirePackage{amsmath,amssymb,amsfonts}
\RequirePackage{hyperref,multimedia,rotating}
\RequirePackage{listings}
\RequirePackage{fontspec}
\RequirePackage[french]{babel}
\RequirePackage[french]{nomencl}
\RequirePackage{eurosym}
\RequirePackage{pifont}

%-------------------------------------------------------------------------
\input{sigle.tex}

%-------------------------------------------------------------------------
\lstset
{
language=Python,
basicstyle=\ttfamily,
identifierstyle=\ttfamily,
keywordstyle=\color{blue}\ttfamily,
commentstyle=\color{gray}\ttfamily,
stringstyle=\color{green}\ttfamily,
showstringspaces=false,
extendedchars=true,
numbers=left, 
numberstyle=\color{blue}\tiny,
frame=lines,
linewidth=0.95\textwidth,
xleftmargin=5mm
} 
%-------------------------------------------------------------------------

%-------------------------------------------------------------------------
\pgfdeclareimage[width=3cm,interpolate=true]{logo-enib}{logo-enib}
%-------------------------------------------------------------------------

\def\python{\textsc{Python}}

%-------------------------------------------------------------------------
\pagestyle{fancy}
\fancyhead{}
\fancyhead[L]{\hspace*{-3em}\begin{minipage}{3cm}\pgfuseimage{logo-enib}\end{minipage}}
\fancyhead[C]{\textsc{Mrv--Physique} : \thesection}
\fancyhead[R]{\thepage/\pageref{LastPage}}
\fancyfoot{}
\fancyfoot[L]{}
\fancyfoot[C]{}
\fancyfoot[R]{}
\setlength{\headheight}{51pt}
\setlength{\footskip}{38pt}
\renewcommand{\headrulewidth}{0pt}
\renewcommand{\footrulewidth}{0pt}
%-------------------------------------------------------------------------

%-------------------------------------------------------------------------
\begin{document}
%-------------------------------------------------------------------------

%-------------------------------------------------------------------------
\newpage
\setcounter{section}{1}
\section{Des n\oe uds aux kilomètres par heure}\label{mrv:noeuds}
%-------------------------------------------------------------------------
\paragraph{Objectif:} Mettre en \oe uvre l'instruction d'affectation.

\paragraph{Syntaxe \python:} \texttt{variable = expression}

\paragraph{}\mbox{}

\noindent\fbox{\begin{minipage}{0.985\textwidth}
\paragraph{Enoncé:}
On veut convertir une certaine quantité $n1$ de vitesse exprimée en n\oe uds 
(miles nautiques par heure) en la quantité équivalente $n_2$ exprimée en 
kilomètres par heure (km/h).\\
Proposer une instruction de type « affectation » qui réalise cette conversion.
\end{minipage}}

\paragraph{Méthode:}
On cherche ici à convertir $n_1\cdot u_1$ en $n_2\cdot u_2$ 
où $u_1$ et $u_2$ sont des unités physiques compatibles 
qui dérivent de la même unité de base $u_b$ du \href{http://www.bipm.org/fr/si/}{Système international d'unités}.
$$\left|\begin{array}{l@{\ =\ }r}
u_1 & a_1\cdot u_b \\
u_2 & a_2\cdot u_b
\end{array}\right.
\ \Rightarrow\ \ 
\left|\begin{array}{l@{\ =\ }r@{\ =\ }r}
n_1\cdot u_1 & n_1\cdot (a_1\cdot u_b) & (n_1\cdot a_1)\cdot u_b\\
n_2\cdot u_2 & n_2\cdot (a_2\cdot u_b) & (n_2\cdot a_2)\cdot u_b
\end{array}\right.
\ \Rightarrow\ \ \frac{n_1\cdot u_1}{n_2\cdot u_2} = \frac{n_1\cdot a_1}{n_2\cdot a_2}$$
Comme on cherche $n_2$ tel que $n_1\cdot u_1 = n_2\cdot u_2$, on a donc :
$$\frac{n_1\cdot u_1}{n_2\cdot u_2} = \frac{n_1\cdot a_1}{n_2\cdot a_2} = 1
\ \Rightarrow\ \ n_2 = n_1 \cdot \frac{a_1}{a_2}$$
où les coefficients $a_i$ sont documentés dans le Système international d'unités par le
\href{http://www.bipm.org/}{Bureau international des poids et mesures}.

Une fois connus les coefficients $a_i$, on détermine la quantité $n_2$ de l'unité $u_2$
par une affectation simple : \texttt{n2 = n1*a1/a2} .

\paragraph{Résultat:} On applique la méthode précédente à la conversion
proposée dans l'énoncé où $u_1$ représente les n\oe uds (miles nautiques par heure), 
$u_2$ les kilomètres par heure (km/h) et $u_b$ les mètres par seconde (m/s).
Le Système international d'unités fournit par ailleurs les facteurs 
de conversion $a_1$ (nd $\rightarrow$ m/s) et $a_2$ (km/h $\rightarrow$ m/s) :
$a_1 = 1852/3600$ et $a_2 = 1000/3600$.


\noindent\begin{minipage}[t]{7cm}
Compte-tenu de ces valeurs, le code ci-contre
permet de calculer le nombre $n_2$ de kilomètres par heure
en fonction du nombre $n_1$ de n\oe uds.
\end{minipage}
\hfill
\begin{minipage}[t]{8cm}
\begin{lstlisting}
a1 = 1852/3600
a2 = 1000/3600
n2 = n1*a1/a2
\end{lstlisting}
\end{minipage}

Remarque : on n'a pas cherché à effectuer « à la main » les calculs numériques :
\python{} les fera mieux que nous; et surtout, on n'a pas cherché non plus à 
particulariser la $3^{\grave eme}$ ligne du code en \texttt{n2 = n1*1852/1000} :
la forme plus abstraite \texttt{n2 = n1*a1/a2} restera identique pour convertir 
des parsecs en années-lumière (longueurs), des gallons en barils (volumes) ou 
encore des électron-volts en frigories (énergies), seules les valeurs des coefficients $a_i$ changeront (lignes \texttt{1} 
et \texttt{2} du code).

\paragraph{Vérification:}
Pour tester le résultat précédent, on peut comparer les valeurs obtenues par le calcul 
avec celles de quelques valeurs caractéristiques facilement évaluables « à la main »
(exemples : $n_1 = 1\ \rm{nd} \Rightarrow{} n_2 = 1.852\ \rm{km/h}$ ou
$n_1 = 1/1852\ \rm{nd} \Rightarrow{} n_2 = 1/1000\ \rm{km/h}$).
$$\begin{minipage}[t]{7.5cm}\footnotesize
\begin{Verbatim}
>>> n1 = 1
>>> a1, a2 = 1852/3600, 1000/3600
>>> n2 = n1*a1/a2
>>> n2
1.852
\end{Verbatim}
\end{minipage}
\hfill
\begin{minipage}[t]{7.5cm}\footnotesize
\begin{Verbatim}
>>> n1 = 1/1852
>>> a1, a2 = 1852/3600, 1000/3600
>>> n2 = n1*a1/a2
>>> n2
0.0010000000000000002
\end{Verbatim}
\end{minipage}$$
On obtient bien par le calcul les résultats escomptés.

%-------------------------------------------------------------------------
\end{document}
%-------------------------------------------------------------------------
