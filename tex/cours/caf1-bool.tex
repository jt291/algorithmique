% cbool.tex
%-----------------------------------------------------------------------------
\documentclass[12pt]{article}
%-----------------------------------------------------------------------------
\usepackage[latin1]{inputenc}
\usepackage{epsfig}
\pagestyle{myheadings}
\markright{{E.N.I. Brest}\hfill{Informatique 1A}\hfill{Objectif 1 -- Page\ }}

\setlength{\textheight}{23cm}
\setlength{\textwidth}{16cm}
\setlength{\parindent}{0cm}

\voffset=-1.5cm
\hoffset=-1.3cm

\def\entete#1{\framebox[13.5cm][l]{Nom : \hspace*{4cm} Pr\'enom : \hspace*{3.5cm}Groupe : $\rule[-0.3cm]{0cm}{1cm}$}
\hfill
\framebox[0.7cm]{\bf \ 0 $\rule[-0.3cm]{0cm}{1cm}$}
\framebox[0.7cm]{\bf \ 1 $\rule[-0.3cm]{0cm}{1cm}$}
\framebox[0.7cm]{\bf \ 2 $\rule[-0.3cm]{0cm}{1cm}$}\\[1mm]
{\footnotesize\sc Durée: 30'\hfill 
Documents, téléphones, calculettes et ordinateurs interdits.}\\[2mm]
\centerline{\bf #1}}

%-----------------------------------------------------------------------------
\begin{document}
%-----------------------------------------------------------------------------

%-----------------------------------------------------------------------------
\entete{Instructions de base : contrôle {\em a priori}}
%-----------------------------------------------------------------------------
\section{Développement d'une expression booléenne}
Développer l'expression booléenne suivante : 
$$t = \overline{a \cdot b \cdot c \cdot d}$$
$$\framebox[13.5cm]{
$\begin{array}{llll}
t &=& \overline{a \cdot b \cdot c \cdot d} & \\
  &=& \overline{(a \cdot b) \cdot (c \cdot d)} & \mbox{associativité}\\ 
  &=& \overline{a \cdot b} + \overline{c \cdot d} & \mbox{De Morgan}\\ 
  &=& (\overline{a} + \overline{b}) + (\overline{c} + \overline{d}) & \mbox{De Morgan}\\
  &=& \overline{a} + \overline{b} + \overline{c} + \overline{d} & \mbox{associativité}
\end{array}$}$$

%-----------------------------------------------------------------------------
\section{Table de vérité d'une expression booléenne}
Etablir la table de vérité de l'expression booléenne suivante :
$$t = ((a \Rightarrow b) \cdot (b \Rightarrow c)) \Rightarrow (\overline{c} \Rightarrow \overline{a})$$
$$\framebox[13.5cm]{
\begin{minipage}{13cm}
On pose $u = (a \Rightarrow b) \cdot (b \Rightarrow c)$ :

$$\begin{array}{|ccc|ccc|ccc|c|}
\hline
a & b & c & (a \Rightarrow b) & (b \Rightarrow c) & u & \overline{c} & \overline{a} & (\overline{c} \Rightarrow \overline{a}) & t \\
\hline
0 & 0 & 0 &   1 & 1 & 1 &   1 & 1 & 1 &   1\\
0 & 0 & 1 &   1 & 1 & 1 &   0 & 1 & 1 &   1\\
0 & 1 & 0 &   1 & 0 & 0 &   1 & 1 & 1 &   1\\
0 & 1 & 1 &   1 & 1 & 1 &   0 & 1 & 1 &   1\\
1 & 0 & 0 &   0 & 1 & 0 &   1 & 0 & 0 &   1\\
1 & 0 & 1 &   0 & 1 & 0 &   0 & 0 & 1 &   1\\
1 & 1 & 0 &   1 & 0 & 0 &   1 & 0 & 0 &   1\\
1 & 1 & 1 &   1 & 1 & 1 &   0 & 0 & 1 &   1\\
\hline
\end{array}$$
\end{minipage}}$$

%-----------------------------------------------------------------------------
\section{Table de vérité d'un circuit logique}
On considère les conventions graphiques traditionnelles pour les opérateurs logiques $\cdot$, $+$ et $\oplus$ :
$$\begin{tabular}{ccc}
$a \cdot b$ & $a + b$ & $a \oplus b$ \\
\epsfig{figure=et.eps} & \epsfig{figure=ou.eps} & \epsfig{figure=xor.eps} 
\end{tabular}$$

Etablir la table de vérité du circuit logique ci-dessous où $a$, $b$
et $c$ sont les entrées, $x$ et $y$ les sorties.

$$\epsfig{figure=add3.eps}$$
$$\framebox[13.5cm]{
\begin{minipage}{13cm}

$$\begin{array}{|ccc|ccc|cc|}
\hline
a & b & c & (b \cdot c) & (b \oplus c) & (a \cdot (b \oplus c)) & x & y \\
\hline
0 & 0 & 0 &   0 & 0 & 0 &   0 & 0 \\
0 & 0 & 1 &   0 & 1 & 0 &   1 & 0 \\
0 & 1 & 0 &   0 & 1 & 0 &   1 & 0 \\
0 & 1 & 1 &   1 & 0 & 0 &   0 & 1 \\
1 & 0 & 0 &   0 & 0 & 0 &   1 & 0 \\
1 & 0 & 1 &   0 & 1 & 1 &   0 & 1 \\
1 & 1 & 0 &   0 & 1 & 1 &   0 & 1 \\
1 & 1 & 1 &   1 & 0 & 0 &   1 & 1 \\
\hline
\end{array}$$

Il s'agit de l'addition des 3 bits $a$, $b$ et $c$ : $x$ est la somme et
$y$ la retenue

$$\begin{tabular}{r}
$a$ \\
$b$ \\
$c$ \\
\hline
$yx$
\end{tabular}
\hspace*{5mm}
\begin{tabular}{r}
0 \\
0 \\
0 \\
\hline
00
\end{tabular}
\hspace*{2mm}
\begin{tabular}{r}
0 \\
0 \\
1 \\
\hline
01
\end{tabular}
\hspace*{2mm}
\begin{tabular}{r}
0 \\
1 \\
0 \\
\hline
01
\end{tabular}
\hspace*{2mm}
\begin{tabular}{r}
0 \\
1 \\
1 \\
\hline
10
\end{tabular}
\hspace*{2mm}
\begin{tabular}{r}
1 \\
0 \\
0 \\
\hline
01
\end{tabular}
\hspace*{2mm}
\begin{tabular}{r}
1 \\
0 \\
1 \\
\hline
10
\end{tabular}
\hspace*{2mm}
\begin{tabular}{r}
1 \\
1 \\
0 \\
\hline
10
\end{tabular}
\hspace*{2mm}
\begin{tabular}{r}
1 \\
1 \\
1 \\
\hline
11
\end{tabular}$$

\end{minipage}}$$
%-----------------------------------------------------------------------------

%-----------------------------------------------------------------------------
\label{fini}
\end{document}
%-----------------------------------------------------------------------------
