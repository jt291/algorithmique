% algoCorrA.tex
%-----------------------------------------------------------------------------
\documentclass[12pt]{article}
%-----------------------------------------------------------------------------
\usepackage[latin1]{inputenc}
\usepackage{epsfig}
\pagestyle{myheadings}
\markright{{E.N.I. Brest}\hfill{Informatique S1}\hfill{Page\ }}

\setlength{\textheight}{23cm}
\setlength{\textwidth}{16cm}
\setlength{\parindent}{0cm}

\voffset=-1.5cm
\hoffset=-1.3cm

\def\entete{\framebox[12.6cm][l]{Nom : \hspace*{3.85cm} Prénom :
\hspace*{3cm}Groupe : {\bf A}$\rule[-0.3cm]{0cm}{1cm}$}
\hfill
\framebox[0.7cm]{\bf \ 3 $\rule[-0.3cm]{0cm}{1cm}$}
\framebox[0.7cm]{\bf \ 2 $\rule[-0.3cm]{0cm}{1cm}$}
\framebox[0.7cm]{\bf \ 1 $\rule[-0.3cm]{0cm}{1cm}$}
\framebox[0.7cm]{\bf \ 0 $\rule[-0.3cm]{0cm}{1cm}$}\\[1mm]
{\footnotesize\sc Durée: 80'\hfill 
Documents, téléphones, calculettes et ordinateurs interdits.}}

\newenvironment{py}[1]{\begin{minipage}[t]{#1}\footnotesize}{\end{minipage}}

%-----------------------------------------------------------------------------
\begin{document}
%-----------------------------------------------------------------------------

%-----------------------------------------------------------------------------
\entete
\vspace*{1mm}
\centerline{\bf DS : algorithmique}
%-----------------------------------------------------------------------------

%-----------------------------------------------------------------------------
\section{Calcul de $\pi$}
%-----------------------------------------------------------------------------
Définir une fonction qui calcule $\pi$ à l'ordre $n$ selon la formule :
	$$\pi = 2\cdot
	\frac{4}{3}\cdot\frac{16}{15}\cdot\frac{36}{35}\cdot\frac{64}{63}\cdots =
	      2\prod_{k=1}^n\frac{4k^2}{4k^2 - 1}$$

$$\begin{minipage}[t]{12cm}
\begin{verbatim}
def pi(n):
    """
    y = pi(n)
    calcule pi à l'ordre n
    >>> abs(pi(1) - math.pi) < 1.
    True
    >>> abs(pi(100) - math.pi) < 1./100
    True
    >>> abs(pi(10000000) - math.pi) < 1.e-7
    True
    """
    assert type(n) is int and n > 0
    y = 2.
    for k in range(1,n+1):
        u = 4*k*k
        y = y*u/(u-1)
    return y
\end{verbatim}
\end{minipage}$$

%-----------------------------------------------------------------------------
\section{Conversion décimal $\rightarrow$ base $b$}
%-----------------------------------------------------------------------------
Définir une fonction qui calcule le code $t$ en base $b$ sur $k$ chiffres
d'un entier $n$.
Exemples pour $n=23$ : 
\begin{tabular}[t]{ll@{\ $\rightarrow$\ }l}
{\tt b = 2}  & {\tt k = 7} & {\tt t = [0, 0, 1, 0, 1, 1, 1]}\\
{\tt b = 5}  & {\tt k = 5} & {\tt t = [0, 0, 0, 4, 3]}\\
{\tt b = 21} & {\tt k = 2} & {\tt t = [1, 2]}\\
{\tt b = 25} & {\tt k = 6} & {\tt t = [0, 0, 0, 0, 0, 23]}
\end{tabular}

$$\begin{minipage}[t]{12cm}
\begin{verbatim}
def code(n,b,k):
    """
    c = code(n,b,k)
    code n en base b sur k chiffres
    >>> code(23,2,8)
    [0, 0, 0, 1, 0, 1, 1, 1]
    >>> code(23,2,7)
    [0, 0, 1, 0, 1, 1, 1]
    >>> code(23,5,5)
    [0, 0, 0, 4, 3]
    >>> code(23,21,2)
    [1, 2]
    >>> code(23,25,6)
    [0, 0, 0, 0, 0, 23]

    """
    assert type(n) is int and n >= 0
    assert type(b) is int and b > 1
    assert type(k) is int and k > 0
    c = []
    for i in range(k): c.append(0)
    q = n
    i = k-1
    while q > 0 and i >= 0:
        r = q%b
        q = q/b
        c[i] = r
        i = i - 1
    return c
\end{verbatim}
\end{minipage}$$

%-----------------------------------------------------------------------------
\section{Quinconce}
%-----------------------------------------------------------------------------
Définir une procédure qui dessine $n\times m$ cercles de rayon $r$
disposés en quinconce sur $n$ rangées de $m$ cercles chacune.
On utilisera les instructions de tracé {\em à la Logo}.

$$\begin{minipage}[t]{12cm}
\begin{verbatim}
def quinconce(n,m,r):
    """
    quinconce(n,m,r)
    trace n rangées de m cercles de rayon r
    disposées en quinconce
    >>> quinconce(5,10,10)
    """
    assert type(n) is int and n > 0
    assert type(m) is int and m > 0
    assert type(r) is int and r > 0
    for i in range(n) :
        x0 = r*(i%2)
        y0 = 2*i*r
        for j in range(m) :
            up()
            goto(x0+2*j*r,y0)
            down()
            circle(r)           
    return
\end{verbatim}
\end{minipage}$$

%-----------------------------------------------------------------------------
\section{Coefficients de Kreweras}
%-----------------------------------------------------------------------------
\begin{minipage}[t]{7cm}
On consid\`ere la fonction {\tt g} ci-contre :
\begin{enumerate}
\item Calculer toutes les valeurs possibles
	de $g(n,m)$ pour $n \in [0,6]$.
\item Vérifier que {\tt 12.*g(5,5)/g(6,6)} est une bonne approximation de
	$\pi$.
\end{enumerate}
\end{minipage}
\hfill
\begin{minipage}[t]{7cm}\footnotesize
\begin{verbatim}
#----------------------------------
def g(n,m):
#----------------------------------
    assert type(n) is int
    assert type(m) is int
    assert 0 <= m and m <= n
    if n == 0 and m == 0:
        c = 1
    else:
        if m == 0: c = 0
        else:
            c = 0
            for i in range(1,m+1):
                c = c + g(n-1,n-i)
    return c
#---------------------------------
\end{verbatim}
\end{minipage}

\noindent\begin{minipage}[t]{8cm}
\begin{verbatim}
>>> for n in range(7):
        print 'n =', n, ' :',
            for m in range(n+1):
                print kreweras(n,m),
        print

	
n = 0  : 1
n = 1  : 0 1
n = 2  : 0 1 1
n = 3  : 0 1 2 2
n = 4  : 0 2 4 5 5
n = 5  : 0 5 10 14 16 16
n = 6  : 0 16 32 46 56 61 61

>>> 2.*6*g(5,5)/g(6,6)
3.1475409836065573
>>> 2.*12*g(11,11)/g(12,12)
3.1416005461074121

\end{verbatim}
\end{minipage}

%-----------------------------------------------------------------------------
\section{Portée des variables}
%-----------------------------------------------------------------------------
On considère les fonctions {\tt f}, {\tt g} et {\tt h} suivantes :
\begin{center}
\begin{py}{4cm}
\begin{verbatim}
def f(x):
    x = 2*x
    print 'f', x
    return x
\end{verbatim}
\end{py}\hspace*{1cm}
\begin{py}{4cm}
\begin{verbatim}
def g(x):
    x = 4*f(x)
    print 'g', x
    return x
\end{verbatim}
\end{py}\hspace*{1cm}
\begin{py}{4cm}
\begin{verbatim}
def h(x):
    x = 3*g(f(x))
    print 'h', x
    return x
\end{verbatim}
\end{py}
\end{center}

Qu'affichent les appels suivants ?
\vspace*{2mm}

\begin{minipage}{7cm}
\begin{enumerate}
\item 

\begin{py}{3cm}
\begin{verbatim}
>>> x = 2
>>> print x
2
>>> y = f(x)
f 4
>>> print x
2
>>> z = g(x)
f 4
g 16
>>> print x
2
>>> t = h(x)
f 4
f 8
g 32
h 96
>>> print x
2
\end{verbatim}
\end{py}
\end{enumerate}
\end{minipage}
\hfill
\begin{minipage}{7cm}
\begin{enumerate}

\item

\begin{py}{4cm}
\begin{verbatim}
>>> x = 2
>>> print x
2
>>> x = f(x)
f 4
>>> print x
4
>>> x = g(x)
f 8
g 32
>>> print x
32
>>> x = h(x)
f 64
f 128
g 512
h 1536
>>> print x
1536
\end{verbatim}
\end{py}

\end{enumerate}
\end{minipage}

%-----------------------------------------------------------------------------
\section{Exécution d'une fonction itérative}
%-----------------------------------------------------------------------------
\begin{minipage}[t]{7cm}
On considère la procédure {\tt f} définie ci-contre.
\begin{enumerate}
\item Qu'affiche l'instruction sui\-vante ?
\begin{verbatim}
>>> for n in range(7): f(n)

1
1 1
1 2 1
1 3 3 1
1 4 6 4 1
1 5 10 10 5 1
1 6 15 20 15 6 1
\end{verbatim}
\item Que représente {\tt c} à la fin de chaque itération sur {\tt p} ?

	Le $p^{i\grave eme}$ coefficient du binôme $(x+y)^n$.
\end{enumerate}
\end{minipage}
\hfill
\begin{py}{8cm}
\begin{verbatim}
#----------------------------------------
def f(n):
#----------------------------------------
    for p in range(n+1) :
        num = 1
        den = 1
        for i in range(1,p+1) :
            num = num*(n-i+1)
            den = den*i
        c = num/den
        print c,  #------------ affichage
    print         #------------ affichage
    return
#----------------------------------------
\end{verbatim}
\end{py}

%-----------------------------------------------------------------------------
\label{fini}
\end{document}
%-----------------------------------------------------------------------------
