% instrucEval.tex
%-----------------------------------------------------------------------------
\documentclass[12pt]{article}
%-----------------------------------------------------------------------------
\usepackage[latin1]{inputenc}
\usepackage{epsfig}
\pagestyle{myheadings}
\markright{{E.N.I. Brest}\hfill{Informatique 1A}\hfill{Page\ }}

\setlength{\textheight}{23cm}
\setlength{\textwidth}{16cm}
\setlength{\parindent}{0cm}

\voffset=-1.5cm
\hoffset=-1.3cm

\def\entete{\newpage\framebox[13.5cm][l]{Nom : \hspace*{4cm} Pr\'enom : \hspace*{3.5cm}Groupe : $\rule[-0.3cm]{0cm}{1cm}$}
\hfill
\framebox[0.7cm]{\bf \ 0 $\rule[-0.3cm]{0cm}{1cm}$}
\framebox[0.7cm]{\bf \ 1 $\rule[-0.3cm]{0cm}{1cm}$}
\framebox[0.7cm]{\bf \ 2 $\rule[-0.3cm]{0cm}{1cm}$}\\[1mm]
{\footnotesize\sc Durée: 1h20\hfill 
Documents, téléphones, calculettes et ordinateurs interdits.}\\[2mm]
\centerline{\bf Instructions de base : contrôle de synthèse}}

\newenvironment{py}[1]{\begin{minipage}[t]{#1}\footnotesize}{\end{minipage}}

%-----------------------------------------------------------------------------
\begin{document}
%-----------------------------------------------------------------------------

%-----------------------------------------------------------------------------
\entete
\section{Exécution d'une séquence d'instructions}
%-----------------------------------------------------------------------------
Qu'affiche la séquence d'instructions suivante ?
\vspace*{5mm}

\hspace*{5mm}\begin{py}{5cm}
\begin{verbatim}
a = 289
x = 1
z = a
y = 0
t = x
print a,x,z,t,y
while x <= a: x = x*4

print a,x,z,t,y
t = x
while x > 1:
  x = x/4
  t = t/2 - x
  if t <= z: 
    z = z - t
    t = t + x*2
  y = t/2
  print a,x,z,t,y

print a,x,z,t,y
\end{verbatim}
\end{py}\hfill
\begin{tabular}[t]{|c|c|c|c|c|}
\hline
\makebox[1cm]{a} & \makebox[1cm]{x} & \makebox[1cm]{z} & \makebox[1cm]{t} & \makebox[1cm]{y} \\
\hline
289 & 1 & 289 & 1 & 0 \\
\hline
289 & 1024 & 289 & 1 & 0 \\
\hline
289 & 256 & 33 & 768 & 384 \\
289 & 64 & 33 & 320 & 160 \\
289 & 16 & 33 & 144 & 72 \\
289 & 4 & 33 & 68 & 34 \\
289 & 1 & 0 & 35 & 17 \\
\hline
289 & 1 & 0 & 35 & 17 \\
\hline
\end{tabular}
\vspace*{5mm}


Il s'agit du calcul de la racine carrée entière $y$ d'un nombre entier $a$ :
$y = 17 = \sqrt{289} = \sqrt{a}$.

%-----------------------------------------------------------------------------
\section{Calcul de $\pi$}
%-----------------------------------------------------------------------------
Dans cette section, on se propose de calculer $\pi$ selon la méthode
des rectangles.

Selon cette méthode, on calcule $\pi$ \`a partir de l'expression de la 
surface $S$ d'un cercle de rayon unité.
On approche la surface du quart de cercle par $n$ rectangles 
d'aire $A_i = y_i/n$.
	
	\begin{minipage}{9cm}
	Ecrire un algorithme qui calcule $\pi$ selon la méthode des rectangles
		à l'ordre $n$.
	\end{minipage}
	\hfill
	\begin{minipage}{4cm}
	\centerline{\epsfig{figure=rectangle.eps,width=4cm}}
	\end{minipage}

$$\framebox[14.5cm]{\begin{minipage}{14cm}
Les points du cercle de rayon unité ont pour coordonnées
	$\displaystyle x_k = \frac{k}{n}$ et $\displaystyle y_k = \sqrt{1-\frac{k^2}{n^2}}$
	et un rectangle de base a pour surface $\displaystyle s_k = \frac{1}{n}\sqrt{1-\frac{k^2}{n^2}}$.
	La surface du 1/4 de cercle s'écrit alors comme la somme des rectangles de base :
	$\displaystyle S = \frac{\pi}{4} = \sum_{k=0}^n s_k = \frac{1}{n}\sum_{k=0}^n \sqrt{1-\frac{k^2}{n^2}}$.
\vspace*{3mm}

Ce qui donne avec l'interpréteur {\tt python} :
$$\begin{py}{12cm}
\tt
>>> from math import *\\
>>> n = 20000000\\
>>> y = 0.\\
>>> for k in range(0,n+1) :\\
...\ \ \ \ y = y + sqrt(1 - (1.*k*k)/(n*n))\\
... \\
>>> pi - 4*y/n\\
-9.9986801060936159e-08\\
\mbox{}
\end{py}$$
\end{minipage}}$$ 

%-----------------------------------------------------------------------------

%-----------------------------------------------------------------------------
\section{Zéro d'une fonction}
%-----------------------------------------------------------------------------
Dans cette section, on recherche le zéro d'une fonction $f$ continue sur un 
intervalle $[a,b]$ telle que $f(a).f(b) < 0$ 
(il existe donc une racine de $f$ dans $]a,b[$ que nous supposerons 
unique). 

Ecrire un algorithme qui détermine le zéro de $\cos(x)$ dans $[1,2]$
	selon la méthode des tangentes.
	
	Indications : soit $x_n$ une approximation de la racine $c$ recherchée : 
	$f(c) = f(x_n) + (c-x_n)f'(x_n)$; comme $f(c) = 0$, on a : 
	$c = x_n - f(x_n)/f'(x_n)$. Posons $x_{n+1} = x_n - f(x_n)/f'(x_n)$ : 
	on peut considérer que $x_{n+1}$ est une meilleure approximation de $c$ que 
	$x_n$. On recommence le procédé avec $x_{n+1}$  et ainsi de suite jusqu'à ce 
	que $|x_{n+1}-x_n|$ soit inférieur à un certain seuil $s$.

$$\framebox[14.5cm]{\begin{minipage}{14cm}
$$\begin{py}{12cm}
\tt
>>> from math import *\\
>>> x1 = 1.\\
>>> x2 = 2.\\
>>> s = 1.e-9\\
>>> f = cos\\
>>> df = sin\\
>>> x = x2 - f(x2)/(-df(x2))\\
>>> while fabs(x-x2) > s:\\
...   x2 = x\\
...   x = x - f(x)/(-df(x))\\
... \\
>>> cos(x)\\
6.1230317691118863e-17\\
\mbox{}
\end{py}$$
\end{minipage}}$$ 

%-----------------------------------------------------------------------------
\section{Tableau d'Ibn al-Banna}
%-----------------------------------------------------------------------------
L'exercice suivant est inspir\'e du premier chapitre du
livre ``Histoire d'algo\-ri\-thmes''\footnote{{\bf Chabert J.-L. et al.}, 
{\em Histoire d'algorithmes : du caillou à la puce, Chapitre 1: algorithmes des op\'erations 
arithm\'etiques}, Editions Belin, Paris, 1994}.
On consid\`ere ici le texte d'Ibn al-Banna concernant la multiplication
\`a l'aide de tableaux.
$$\begin{minipage}{15cm}\footnotesize
Tu construis un quadrilat\`ere que tu subdivises verticalement et
horizontalement en autant de bandes qu'il y a de positions dans les
deux nombres multipli\'es. Tu divises diagonalement les carr\'es
obtenus, \`a l'aide de diagonale allant du coin inf\'erieur gauche au
coin sup\'erieur droit.

Tu places le multiplicande au-dessus du quadrilat\`ere, en faisant 
correspondre chacune de ses positions \`a une colonne\footnote{L'écriture
du nombre s'effectue de droite à gauche (exemple : 352 s'écrira donc 253).}. 
Puis, tu places le multiplicateur \`a gauche ou \`a droite du quadrilat\`ere,
de telle sorte qu'il descende avec lui en faisant correspondre \'egalement 
chacune de ses positions \`a une ligne\footnote{L'écriture
du nombre s'effectue de bas en haut (exemple : {\tiny$\begin{array}{c}3\\5\\2\end{array}$} 
s'écrira donc {\tiny$\begin{array}{c}2\\5\\3\end{array}$}).}. Puis, tu multiplies, 
l'une apr\`es l'autre, chacune des positions du multiplicande du carr\'e 
par toutes les positions du multiplicateur, et tu poses le r\'esultat 
partiel correspondant \`a chaque position dans le carr\'e o\`u se coupent 
respectivement leur colonne et leur ligne, en pla\c{c}ant les unit\'es 
au-dessus de la diagonale et les dizaines en dessous. Puis, tu
commences \`a additionner, en partant du coin sup\'erieur gauche :
tu additionnes ce qui est entre les diagonales, sans effacer, 
en pla\c{c}ant chaque nombre dans sa position, en transf\'erant 
les dizaines de chaque somme partielle \`a la diagonale suivante et
en les ajoutant \`a ce qui y figure. 

La somme que tu obtiendras sera le r\'esultat.
\end{minipage}$$

En utilisant la m\'ethode du tableau d'Ibn al-Banna, calculer $63247\times124$ 
($= 7842628$).

$$\framebox[14.5cm]{\begin{minipage}{14cm}
$$\epsfig{figure=ibnalbanna.eps,width=7cm}$$
\end{minipage}}$$ 

%-----------------------------------------------------------------------------
\label{fini}
\end{document}
%-----------------------------------------------------------------------------
