% typesCorrA.tex
%-----------------------------------------------------------------------------
\documentclass[12pt]{article}
%-----------------------------------------------------------------------------
\usepackage[latin1]{inputenc}
\usepackage{epsfig}
\pagestyle{myheadings}
\markright{{E.N.I. Brest}\hfill{Informatique S1}\hfill{Page\ }}

\setlength{\textheight}{23cm}
\setlength{\textwidth}{16cm}
\setlength{\parindent}{0cm}

\voffset=-1.5cm
\hoffset=-1.3cm

\def\entete#1{\framebox[12.6cm][l]{Nom : \hspace*{3.85cm} Prénom : \hspace*{3cm}Groupe : {\bf A}$\rule[-0.3cm]{0cm}{1cm}$}
\hfill
\framebox[0.7cm]{\bf \ 3 $\rule[-0.3cm]{0cm}{1cm}$}
\framebox[0.7cm]{\bf \ 2 $\rule[-0.3cm]{0cm}{1cm}$}
\framebox[0.7cm]{\bf \ 1 $\rule[-0.3cm]{0cm}{1cm}$}
\framebox[0.7cm]{\bf \ 0 $\rule[-0.3cm]{0cm}{1cm}$}\\[1mm]
{\footnotesize\sc Durée: 30'\hfill 
Documents, téléphones, calculettes et ordinateurs interdits.}\\[2mm]
\centerline{\bf #1}}


\newenvironment{py}[1]{\begin{minipage}[t]{#1}\footnotesize}{\end{minipage}}

%-----------------------------------------------------------------------------
\begin{document}
%-----------------------------------------------------------------------------

%-----------------------------------------------------------------------------
\entete{Contrôle d'autoformation : codage des nombres}
%-----------------------------------------------------------------------------
\section{Représentation en complément à 2}
\begin{enumerate}
\item Déterminer la plage de valeurs entières possibles lorsqu'un
	entier positif, négatif ou nul est codé en binaire 
	sur $k = 4$ chiffres dans la représentation en complément à 2.
	
	$$\framebox[13.5cm]{\begin{minipage}{13cm}
	Lors d'un codage en binaire sur $k$ bits, 
	seuls les nombres entiers relatifs $n$ 
	tels que $-2^{k-1} \leq n < 2^{k-1}$ peuvent être représentés 
	($\displaystyle n \in [-2^{k-1};2^{k-1}[$).
	\vspace*{3mm}
	
	Application numérique : $k=4 : [-2^{4-1};2^{4-1}[ \ = [-8;+8[$
	\end{minipage}}$$

\item Coder l'entier $n = (- 93)_{10}$ sur $k = 8$ chiffres en base $b = 2$
	en utilisant la représentation du complément à 2.
	
	$$\framebox[13.5cm]{
\begin{minipage}{13cm}
\begin{enumerate}
\item coder $|n|=(93)_{10}$ sur $(k-1) = 7$ chiffres :
	$(93)_{10} = (1011101)_2$ 
\item mettre le bit de poids fort ($k^{\grave eme}$ bit) à $0$ :
        $(01011101)_2$ 
\item inverser tous les bits obtenus :
        $(01011101)_2 \rightarrow (10100010)_2$
\item ajouter 1 : 
	$(10100010)_2 + (00000001)_2 = (10100011)_2$
\item conclusion :
	$n = (-93)_{10} = (10100011)_2$
\end{enumerate}
\end{minipage}}$$

\item Vérifier la réponse précédente en additionnant en base $b=2$,
	$n$ et $(-n)$, codés sur $k = 8$ chiffres dans la représentation
	du complément à 2.

	$$\framebox[13.5cm]{\begin{minipage}{13cm}
$\begin{array}{lrrrrr}
  & n &=& (- 93)_{10} &=& \ (10100011)_2\\[3mm]
+ & (-n) &=& (+ 93)_{10} &=& \ (01011101)_2\\[3mm]
\hline
\\[-2mm]
= & 0 &=& (0)_{10} &=& \ 1(00000000)_2
\end{array}$
\vspace*{3mm}

Le bit de poids fort (le $9^{\grave eme}$ bit à partir de la droite : {\tt 1})
est perdu.
\end{minipage}}$$
\end{enumerate}

%-----------------------------------------------------------------------------
\section{IEEE 754 simple précision}
Coder le nombre réel $x=-73.25$ selon la norme IEEE 754 en simple précision.
	
	$$\framebox[14.5cm]{
\begin{minipage}{14cm}
\begin{description}
\item[partie entière :] : $(73)_{10} = (1001001)_2$
\item[partie fractionnaire :] $(0.25)_{10} = (0.01)_2$
\item[mantisse normalisée :] $(1001001.01)_2 =
(1.00100101)_2 \cdot 2^6 = (1.00100101)_2 \cdot 2^{(00000110)_2}$
\item[exposant relatif à 127 :] $(01111111)_2 + (00000110)_2 = (10000101)_2$
\item[signe :] $(1)_2$
\item[norme IEEE 754 :]
\end{description}
\centerline{\epsfig{figure=ieee754A.eps,width=13.5cm}}
\end{minipage}
}$$

%-----------------------------------------------------------------------------
\label{fini}
\end{document}
%-----------------------------------------------------------------------------
