%-------------------------------------------------------------------------
% info-S1-exercices.tex
%-------------------------------------------------------------------------

Le planning prévisionnel de la section \ref{annexe:planning} 
page \pageref{annexe:planning} permet de visualiser
la répartition des 7 séances de travaux dirigées organisées
au cours des 15 semaines du cours d'Informatique S1 de l'ENIB.
Dans cette annexe, on précise pour chaque TD
\begin{enumerate}
\item les objectifs recherchés,
\item les exercices de TD à préparer avant la séance,
\item les exercices complémentaires pour s'exercer.
\end{enumerate}

%-------------------------------------------------------------------------
\section*{TD 1}\label{td1}
%-------------------------------------------------------------------------
\subsection*{Objectifs}
\begin{enumerate}
\item Prise en main de l'environnement informatique 
	(système d'exploitation {\sc Linux}, environnement de 
	programmation \python, site {\sc Web})\dotfill\ {\color{blue}1h30}\mbox{}\linebreak
\item Exploitation des instructions de base : 
	affectation et tests 
	(chapitre \ref{ch:instructions}, sections \ref{affectation} et 
	\ref{tests} page \pageref{tests})
	\dotfill\ {\color{blue}1h30}\mbox{}
\end{enumerate}

\subsection*{Exercices de TD}
\begin{description}
\item[\sc Affectation]\mbox{}\\
	\exercice{td:torr}{Unité de pression.}\\
	\exercice{td:suiteArit}{Suite arithmétique (1).}\\
	\exercice{td:permutation1}{Permutation circulaire (1).}\\
	\exercice{td:seq1}{Séquence d'affectations (1).}
\item[\sc Tests]\mbox{}\\
	\exercice{td:booleens1}{Opérateurs booléens.}\\
	\exercice{td:circuits}{Circuits logiques (1).}\\
	\exercice{td:prop}{Lois de De Morgan.}\\
	\exercice{td:max}{Maximum de 2 nombres.}\\
	\exercice{td:porte}{Fonction « porte ».}\\
	\exercice{td:guichet}{Ouverture d'un guichet.}\\
	\exercice{td:categorie}{Catégorie sportive.}
\end{description}

\subsection*{Exercices complémentaires}
\begin{description}
\item[\sc Comprendre]\mbox{}\\
	\exercice{td:al}{Unité de longueur.}\\
	\exercice{td:permutation2}{Permutation circulaire (2).}\\
	\exercice{td:seq2}{Séquence d'affectations (2).}\\
	\exercice{td:circuits2}{Circuits logiques (2).}\\
	\exercice{td:reciproque}{Alternative simple et test simple.}\\
	\exercice{td:trinome}{Racines du trinome.}\\
	\exercice{td:seq3}{Séquences de tests.}\\
\item[\sc Appliquer]\mbox{}\\
	\exercice{td:geometrieInstruc}{Figures géométriques.}\\
	\exercice{td:suites}{Suites numériques.}\\
	\exercice{td:vecteurs}{Calcul vectoriel.}\\
	\exercice{td:photocopie}{Prix d'une photocopie.}\\
	\exercice{td:impot}{Calcul des impôts.}\\
\item[\sc Analyser]\mbox{}\\
	\exercice{td:dessins}{Dessins géométriques.}\\
	\exercice{td:assurance}{Police d'assurance.}
\end{description}
%-------------------------------------------------------------------------


%-------------------------------------------------------------------------
\section*{TD 2}\label{td2}
%-------------------------------------------------------------------------
\subsection*{Objectifs}
\begin{enumerate}
\item Exploitation des instructions de base : boucles
	(chapitre \ref{ch:instructions}, section \ref{boucles})
	\dotfill\ {\color{blue}3h}\mbox{}
\end{enumerate}

\subsection*{Exercices de TD}
\begin{description}
\item[\sc Boucles]\mbox{}\\
	\exercice{td:etoile}{Dessin d'étoiles (1).}\\
	\exercice{td:factorielle}{Fonction factorielle.}\\
	\exercice{td:sinus}{Fonction sinus.}\\
	\exercice{td:euclide}{Algorithme d'Euclide.}\\
	\exercice{td:division}{Division entière.}\\
	\exercice{td:caractere}{Affichage inverse.}\\
	\exercice{td:parcours}{Parcours inverse.}\\
	\exercice{td:suiteArit2}{Suite arithmétique (2).}\\
	\exercice{td:etoile2}{Dessin d'étoiles (2).}\\
	\exercice{td:booleens2}{Opérateurs booléens dérivés (2).}\\
	\exercice{td:damier}{Damier.}\\
	\exercice{td:traceFactorielle}{Trace de la fonction factorielle.}\\
	\exercice{td:quinconce}{Figure géométrique.}
\end{description}

\subsection*{Exercices complémentaires}
\begin{description}
\item[\sc Comprendre]\mbox{}\\
	\exercice{td:racine}{Racine carrée entière.}\\
	\exercice{td:iterations}{Exécutions d'instructions itératives.}
\item[\sc Appliquer]\mbox{}\\
	\exercice{td:dev}{Développements limités.}\\
	\exercice{td:tablesVerite}{Tables de vérité.}
\item[\sc Analyser]\mbox{}\\
	\exercice{td:zero}{Zéro d'une fonction.}
\end{description}

%-------------------------------------------------------------------------
\section*{TD 3}\label{td3}
%-------------------------------------------------------------------------
\subsection*{Objectifs}
\begin{enumerate}
\item Exploitation des instructions de base : affectation, tests et boucles imbriqués
	(chapitre \ref{ch:instructions} en entier)
	\dotfill\ {\color{blue}3h}\mbox{}
\end{enumerate}

\subsection*{Exercices de TD}
{\footnotesize\noindent Il s'agit d'un TD récapitulatif sur les instructions de base, aussi les exercices à
préparer font partie des exercices complémentaires des TD précédents.}
\vspace*{3mm}

\noindent 
\exercice{td:dev}{Développements limités.}\\
\exercice{td:tablesVerite}{Tables de vérité.}\\
\exercice{td:zero}{Zéro d'une fonction.}


%-------------------------------------------------------------------------
\section*{TD 4}\label{td4}
%-------------------------------------------------------------------------
\subsection*{Objectifs}
\begin{enumerate}
\item Spécifier et implémenter des fonctions
	(chapitre \ref{ch:fonctions}, section \ref{definition})
	\dotfill\ {\color{blue}3h}\mbox{}
\end{enumerate}

\subsubsection*{Exercices de TD}
\noindent 
\exercice{td:codageEntier}{Codage des entiers positifs (1).}\\
\exercice{td:fraction}{Codage d'un nombre fractionnaire.}\\
\exercice{td:decoder}{Décodage base b $\rightarrow$ décimal.}\\
\exercice{td:codage}{Codage des entiers positifs (2).}\\
\exercice{td:implem}{Une spécification, plusieurs implémentations.}

\subsubsection*{Exercices complémentaires}
\begin{description}
\item[\sc Analyser]\mbox{}\\
	\exercice{td:addition2}{Addition binaire.}\\
	\exercice{td:complement2}{Complément à 2.}\\
	\exercice{td:ieee754}{Codage-décodage des réels.}\\
	\exercice{td:integration}{Intégration numérique.}\\
	\exercice{td:traces}{Tracés de courbes paramétrées.}

\end{description}

%-------------------------------------------------------------------------
\section*{TD 5}\label{td5}
%-------------------------------------------------------------------------
\subsection*{Objectifs}
\begin{enumerate}
\item Spécifier et implémenter des fonctions
	(chapitre \ref{ch:fonctions}, section \ref{definition})
	\dotfill\ {\color{blue}3h}\mbox{}
\end{enumerate}

\subsubsection*{Exercices de TD}
{\footnotesize\noindent Il s'agit d'un TD récapitulatif sur la définition des fonctions, aussi les exercices à
préparer font partie des exercices complémentaires des TD précédents.}
\vspace*{3mm}

\noindent 
\exercice{td:integration}{Intégration numérique.}\\
\exercice{td:traces}{Tracés de courbes paramétrées.}

%-------------------------------------------------------------------------
\section*{TD 6}\label{td6}
%-------------------------------------------------------------------------
\subsection*{Objectifs}
\begin{enumerate}
\item Appeler des fonctions itératives ou récursives
	(chapitre \ref{ch:fonctions}, section \ref{appel})
	\dotfill\ {\color{blue}3h}\mbox{}
\end{enumerate}

\subsubsection*{Exercices de TD}
\noindent
\exercice{td:swap}{Passage par valeur.}\\
\exercice{td:defaut}{Valeurs par défaut.}\\
\exercice{td:portee}{Portée des variables (1).}\\
\exercice{td:hanoi}{Tours de Hanoï {\em à la main}.}\\
\exercice{td:pgcd}{Pgcd et ppcm de 2 entiers.}\\
\exercice{td:somme}{Somme arithmétique.}\\
\exercice{td:fractal}{Courbes fractales.}

\subsubsection*{Exercices complémentaires}
\begin{description}
\item[\sc Comprendre]\mbox{}\\
	\exercice{td:passage}{Passage des paramètres.}\\
	\exercice{td:portee2}{Portée des variables (2).}

\item[\sc Appliquer]\mbox{}\\
	\exercice{td:geometrie}{Suite géométrique.}\\
	\exercice{td:puissance}{Puissance entière.}\\
	\exercice{td:binome}{Coefficients du binôme.}\\
	\exercice{td:ackerman}{Fonction d'Ackerman.}
\end{description}

%-------------------------------------------------------------------------
\section*{TD 7}\label{td7}
%-------------------------------------------------------------------------
\subsection*{Objectifs}
\begin{enumerate}
\item Manipulation de séquences
	(chapitre \ref{ch:listes}, section \ref{sequence})
	\dotfill\ {\color{blue}3h}\mbox{}
\end{enumerate}

\subsubsection*{Exercices de TD}
\begin{description}
\item[\sc N-uplets]\mbox{}\\
	\exercice{td:n-uplet}{Opérations sur les n-uplets.}\\
	\exercice{td:pgcdppcm}{Pgcd et ppcm de 2 entiers. (2)}
	
\item[\sc Chaînes de caractères]\mbox{}\\
	\exercice{td:chaine}{Opérations sur les chaînes de caractères.}\\
	\exercice{td:inverser}{Inverser une chaîne.}\\
	\exercice{td:wc}{Caractères, mots, lignes d'une chaîne.}
	
\item[\sc Listes]\mbox{}\\
	\exercice{td:listes1}{Opérations sur les listes. (1)}\\
	\exercice{td:listes2}{Opérations sur les listes. (2)}\\
	\exercice{td:collect}{Sélection d'éléments.}
	
\item[\sc Piles et files]\mbox{}\\
	\exercice{td:pile}{Opérations sur les piles.}\\
	\exercice{td:file}{Opérations sur les files.}
	
\item[\sc Listes multidimensionnelles]\mbox{}\\
	\exercice{td:matrices1}{Produit de matrices.}
\end{description}

\subsubsection*{Exercices complémentaires}
\begin{description}
\item[\sc Comprendre]\mbox{}\\
	\exercice{td:alea}{Génération de séquences.}\\
	\exercice{td:foreach}{Application d'une fonction à tous les éléments d'une liste.}\\
	\exercice{td:trishell}{Que fait cette procédure ?}  

\item[\sc Appliquer]\mbox{}\\
	\exercice{td:asciichaines}{Codes ASCII et chaînes de caractères.}\\
	\exercice{td:matrices2}{Opérations sur les matrices.}

\item[\sc Analyser]\mbox{}\\
	\exercice{td:motif}{Recherche d'un motif.}\\
	\exercice{td:occurs}{Recherche de toutes les occurences.}\\
	\exercice{td:bulles}{Tri bulles.}\\
	\exercice{td:gauss}{Méthode d'élimination de {\sc Gauss}.}  

\item[\sc Evaluer]\mbox{}\\
	\exercice{td:exectri2}{Comparaison d'algorithmes de recherche.}\\
	\exercice{td:exectri3}{Comparaison d'algorithmes de tri.}

\end{description}
