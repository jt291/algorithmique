% rechercheEvalA.tex
%-----------------------------------------------------------------------------
\documentclass[12pt]{article}
%-----------------------------------------------------------------------------
\usepackage[latin1]{inputenc}
\usepackage{epsfig}
\usepackage{listings}
\pagestyle{myheadings}
\markright{{E.N.I. Brest}\hfill{Informatique S1}\hfill{Page\ }}

\setlength{\textheight}{23cm}
\setlength{\textwidth}{16cm}
\setlength{\parindent}{0cm}

\voffset=-1.5cm
\hoffset=-1.3cm

\def\entete#1{\framebox[12.6cm][l]{Nom : \hspace*{3.85cm} Prénom :
\hspace*{3cm}Groupe : {\bf A}$\rule[-0.3cm]{0cm}{1cm}$}
\hfill
\framebox[0.7cm]{\bf \ 3 $\rule[-0.3cm]{0cm}{1cm}$}
\framebox[0.7cm]{\bf \ 2 $\rule[-0.3cm]{0cm}{1cm}$}
\framebox[0.7cm]{\bf \ 1 $\rule[-0.3cm]{0cm}{1cm}$}
\framebox[0.7cm]{\bf \ 0 $\rule[-0.3cm]{0cm}{1cm}$}\\[1mm]
{\footnotesize\sc Durée: 30'\hfill 
Documents, téléphones, calculettes et ordinateurs interdits.}\\[2mm]
\centerline{\bf #1}}

%-------------------------------------------------------------------------
\lstset
{
language=Python,
basicstyle=\ttfamily,
identifierstyle=\ttfamily,
keywordstyle=\color{blue}\ttfamily,
commentstyle=\color{gray}\ttfamily,
stringstyle=\color{green}\ttfamily,
showstringspaces=false,
extendedchars=true,
numbers=left, 
numberstyle=\tiny,
frame=lines,
linewidth=0.95\textwidth,
xleftmargin=5mm
} 
%-------------------------------------------------------------------------


%-----------------------------------------------------------------------------
\begin{document}
%-----------------------------------------------------------------------------

%-----------------------------------------------------------------------------
\entete{Contrôle d'autoformation : recherche d'un élément}
%-----------------------------------------------------------------------------

%-----------------------------------------------------------------------------
\section{Recherche d'une occurrence}
Définir une fonction {\tt rechercheKieme} qui recherche le rang {\tt r}
de la $k^{\grave eme}$ occurrence d'un élément {\tt x} à partir du début 
d'une liste {\tt t}.\\
Exemple : {\tt t = [7,4,3,2,4,4]}, {\tt k = 2}, {\tt x = 4} $\rightarrow$ {\tt r
= 4}

$$\framebox[14.5cm]{\begin{minipage}{14cm}
\begin{lstlisting}
def rechercheKieme(t,x,k):
    """
    recherche la kième occurrence de x dans le tableau t
    -> (found, index)
    (found == False and index == len(t)) or
    (found == True and 0 <= index < len(t))

    >>> rechercheKieme([],7,2)
    (False, 0)
    >>> rechercheKieme([1,2,3,2],7,2)
    (False, 4)
    >>> rechercheKieme([1,2,3,2],2,2)
    (True, 3)
    """
    assert type(t) is list
    assert type(k) is int
    found = False
    index = 0
    occur = 0
    
    while index < len(t) and not found:
        if t[index] == x:
            occur = occur + 1
            if occur == k: found = True
            else: index = index + 1
        else: index = index + 1

    return found, index
\end{lstlisting}
\end{minipage}
}$$

%-----------------------------------------------------------------------------
\section{Exécution d'une fonction}
\begin{minipage}[t]{7cm}
On considère la fonction {\tt g} ci-contre.

\begin{enumerate}
\item Qu'affiche l'appel suivant ?
$$\framebox[5.9cm]{
\begin{minipage}{5.75cm}\tt
>>> t = [5,2,5,3,5,2,5,2]\\
>>> g(t,2,3,0)
\vspace*{6.7cm}

\end{minipage}}$$
\end{enumerate}
\end{minipage}
\hfill
\begin{minipage}[t]{8cm}\footnotesize
\begin{verbatim}
#-----------------------------------------
def g(t,x,k,d):
#-----------------------------------------
    assert type(t) is list
    assert type(k) is int and k > 0
    assert d in [0,1]

    ok = False
    i  = (1-d)*(len(t)-1)
    n  = 0
    
    print i,n,ok #-------------- affichage
    while i in range(0,len(t)) and not ok:
        if t[i] == x:
            n = n + 1
            if n == k: ok = True
            else: i = i - (-1)**d
        else: i = i - (-1)**d
        print i,n,ok #---------- affichage

    print i,n,ok #-------------- affichage
    return ok,i
#-----------------------------------------
\end{verbatim}
\end{minipage}

\begin{enumerate}\setcounter{enumi}{1}
\item Compléterer la spécification de la fonction {\tt g}.
	$$\framebox[14.75cm]{\begin{minipage}{14cm}
	\begin{description}
	\item[Description :]\mbox{} \vspace*{2.5cm}
	
	\item[Jeu de tests :]\mbox{} \vspace*{7.5cm}
	\end{description}
	\end{minipage}}$$
\end{enumerate}

%-----------------------------------------------------------------------------
\label{fin}
\end{document}
%-----------------------------------------------------------------------------
